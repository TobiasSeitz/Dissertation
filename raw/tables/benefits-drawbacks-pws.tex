% Table generated by Excel2LaTeX from sheet 'Sheet1'
\begin{table}[H]
  \centering
  \caption{\label{table:rw:benefits_drawbacks_pws}Benefits and Drawbacks for different stakeholders in a password-based authentication scheme}
    \begin{tabular}{c|lr}
	\cmidrule{2-3}    \multicolumn{1}{r}{} & \textbf{Benefits} & \multicolumn{1}{l}{\textbf{Drawbacks}} \\
	\cmidrule{2-3}    \rowcolor[rgb]{ .949,  .949,  .949} \multirow{5}[1]{*}{\textbf{SPs}} & Low costs & \multicolumn{1}{l}{Large number of attack vectors} \\
	\rowcolor[rgb]{ .949,  .949,  .949}       & Easy to implement & \multicolumn{1}{l}{Anomaly detection costly} \\
	\rowcolor[rgb]{ .949,  .949,  .949}       & Replaceable when compromised & \multicolumn{1}{l}{Attacks are simple to carry out} \\
	\rowcolor[rgb]{ .949,  .949,  .949}       & Revokable by administrator & \multicolumn{1}{l}{Attack automation simple} \\
	\rowcolor[rgb]{ .949,  .949,  .949}       & Enforceable policies & \multicolumn{1}{l}{Attacks can have severe consequences} \\
	\rowcolor[rgb]{ .886,  .937,  .855} \multirow{6}[0]{*}{\textbf{Users}} & Fast entry on desktops & \multicolumn{1}{l}{Memory overload from too many passwords} \\
	\rowcolor[rgb]{ .886,  .937,  .855}       & Most users already familiarized & \multicolumn{1}{l}{Suboptimal coping strategies} \\
	\rowcolor[rgb]{ .886,  .937,  .855}       & Easy to learn & \multicolumn{1}{l}{Stronger passwords difficult to memorize} \\
	\rowcolor[rgb]{ .886,  .937,  .855}       & Sharable with others & \multicolumn{1}{l}{Entry on mobile devices difficult} \\
	\rowcolor[rgb]{ .886,  .937,  .855}       & High degree of control / freedom & \multicolumn{1}{l}{Mastery difficult} \\
	\rowcolor[rgb]{ .886,  .937,  .855}       &       & \multicolumn{1}{l}{Disliked by many users / perceived as burden} \\
	\rowcolor[rgb]{ .851,  .882,  .949} \multirow{2}[0]{*}{\textbf{Misc}} & Idenpendent of identification & \multicolumn{1}{l}{Weak passwords are a risk for users and SPs } \\
	\rowcolor[rgb]{ .851,  .882,  .949}       & Adjustable security level &  \\
	\end{tabular}%
\end{table}%