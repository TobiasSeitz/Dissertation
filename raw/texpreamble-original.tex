\newcommand{\mytitle}{Supporting Users in Password Authentication with Persuasive Design}
\newcommand{\mysubtitle}{}
\newcommand{\myname}{Tobias Seitz}
\newcommand{\ie}{i.e.,\ }
\newcommand{\eg}{e.g.,\ }


\definecolor{pinkaccent}{HTML}{F50057}
\definecolor{darkindigo}{HTML}{303F9F}
\definecolor{lightblue}{HTML}{c5cae9}
\definecolor{primaryblue}{HTML}{3f51b5}
\definecolor{verylightblue}{HTML}{B5BDEA}
\definecolor{bluegray}{HTML}{607d8b}
\definecolor{secondarygray}{HTML}{757575}
\definecolor{dividergray}{HTML}{bdbdbd}
\definecolor{almostwhite}{HTML}{eaeaea}



%Stats
\newcommand{\statslt}[3]{\textit{F\textsubscript{#1}}~$=$~#2, \textit{p}~$<$~#3}
\newcommand{\stats}[3]{\textit{F\textsubscript{#1}}~$=$~#2, \textit{p}~$=$~#3}
\newcommand{\statsgt}[3]{\textit{F\textsubscript{#1}}~$=$~#2, \textit{p}~$>$~#3}
\newcommand{\pval}[1]{\textit{p}~$=$~#1}
\newcommand{\pvallt}[1]{\textit{p}~$<$~#1}
\newcommand{\pvalgt}[1]{\textit{p}~$>$~#1}
\newcommand{\mean}[2]{\textit{m}~$=$~#1, \textit{sd}~$=$~#2}
\newcommand{\meant}[2]{\textit{m}$=$#1 \textit{sd}$=$#2}
\newcommand{\meanonly}[1]{\textit{m}~$=$~#1}
\newcommand{\median}[1]{\textit{median}~$=$~#1}
\newcommand{\average}{average:\xspace}
\newcommand{\Rsqadj}[1]{$R_{adj}^2$~$=$~#1}
%tables and figures
\newcommand*{\thead}[1]{\multicolumn{1}{c}{\bfseries #1}}
\newunicodechar{✓}{\ding{51}}
\newunicodechar{✘}{\ding{56}}
\newunicodechar{↗}{\ding{246}}
\newunicodechar{↘}{\ding{244}}

% emojis
\newlength\myheight
\newlength\mydepth
\settototalheight\myheight{Xygp}
\settodepth\mydepth{Xygp}
\setlength\fboxsep{0pt}
\newcommand*\inlinegraphics[1]{%
	\settototalheight\myheight{Xygp}%
	\settodepth\mydepth{Xygp}%
	\raisebox{-\mydepth}{\includegraphics[height=\myheight]{#1}}%
}
\newcommand{\emoji}[2][ios]{\inlinegraphics{../latex-emoji/#1/#2}}%
% this works on a more rudimentary level: \newcommand{\emoji}[1]{\includegraphics[width=1em]{../latex-emoji/ios/#1}}%

\newcommand{\citep}[1]{\cite{#1}} %convenience for texmaker, but texstudio doesn't have this problem. % natbib package

\graphicspath{{figures/}}

%SHORTCUTS
\label{Shortcuts}
\usepackage{xspace}
\newcommand{\percent}{\%\xspace}
\newcommand{\seefig}{see Figure\xspace}
\newcommand{\red}[1]{\textcolor{red}{#1}}
\newcommand{\access}[1]{\textit{(last accessed #1)}}
\newcommand{\la}[1]{\textit{(last accessed #1)}}
\newcommand{\lastaccess}[1]{\textit{(last accessed #1)}}
\newcommand{\footurl}[2]{\footnote{\url{#1} \lastaccess{#2}}}
%alternative approach: try if time:
%\newcommand{\footurl}[2]{
%	\bibitem{#1}~, \url{http://en.wikipedia.org/wiki/Business_logic_layer}, #2.
%	\cite{#1}
%}


\newcommand{\etal}{et al.\xspace}
\newcommand{\todo}[1]{\textcolor{pinkaccent}{\textbf{@TODO} {#1}}}
\newcommand{\ar}{\textcolor{pinkaccent}{\textbf{@ref}}}
\newcommand{\vspc}{\vspace*{2cm}}

\usepackage[light]{roboto}  %% Option 'sfdefault' only if the base font of the document is to be sans serif

\renewcommand{\floatpagefraction}{.8}% this avoids putting larger images onto separate pages.

% Footnote style.
\makeatletter
\renewcommand{\@makefntext}[1]{%
	\setlength{\parindent}{0pt}%
	\begin{list}{}{\setlength{\labelwidth}{1em}%
			\setlength{\leftmargin}{\labelwidth}%
			\setlength{\labelsep}{3pt}%
			\setlength{\itemsep}{0pt}%
			\setlength{\parsep}{0pt}%
			\setlength{\topsep}{0pt}%
			%   \setlength{\rightmargin}{0.2\textwidth}%
			\footnotesize}%
		\item[\@makefnmark\hfil]#1%
	\end{list}%
}
\makeatother

%%% SARAH's original:

% TEST :: BEGIN ========
\clubpenalty=10000
\widowpenalty=10000
\displaywidowpenalty=10000
%\interlinepenalty=5000
% TEST :: END ==========

% Quotes
\label{Quotes}
\newcommand{\sciencequote}[3]{
	\vskip 5pt
	\begin{flushright}
		\begin{minipage}{#3\textwidth}
			\begin{flushright}
				\fontsize{14}{16}\rmfamily\textit{#1}
			\end{flushright}
		\end{minipage}
		\vskip 4pt
		\begin{minipage}{#3\textwidth}
			\begin{flushright}
				\fontsize{12}{14}\rmfamily\textbf{-- #2 --}
			\end{flushright}
		\end{minipage}
	\end{flushright}
	\vskip 30pt
}


\makeatletter
\renewenvironment{theindex}
{\if@twocolumn
	\@restonecolfalse
	\else
	\@restonecoltrue
	\fi
	\setlength{\columnseprule}{0pt}
	\setlength{\columnsep}{20pt}
	\begin{multicols}{2}[\chapter*{\indexname}]
		\markboth{\MakeUppercase\indexname}%
		{\MakeUppercase\indexname}%
		\thispagestyle{plain}
		\setlength{\parindent}{0pt}
		\setlength{\parskip}{0pt plus 0.3pt}
		\relax
		\let\item\@idxitem}%
	{\end{multicols}\if@restonecol\onecolumn\else\clearpage\fi}
\makeatother

\AtBeginDocument{%
	\definecolor{gray}{rgb}{0.8,0.8,0.8}
	\definecolor{darkgray}{RGB}{102,102,102}
}% end \AtBeginDocument

\newcommand{\clearemptydoublepage}{%
	\ifthenelse{\boolean{@twoside}}{\newpage{\pagestyle{empty}\cleardoublepage}}%
	{\clearpage}}

\ifpdf

  \definecolor{darkred}{rgb}{.25,0,0}
  \definecolor{darkgreen}{rgb}{0,.2,0}
  \definecolor{darkmagenta}{rgb}{.2,0,.2}
  \definecolor{darkcyan}{rgb}{0,.15,.15}
  \definecolor{headings}{rgb}{0,0,.3}
  
  \definecolor{darkgray}{rgb}{.4,.4,.4}
  \definecolor{lightgray}{rgb}{.85,.85,.85}
  % plainpages=false to avoid warnings about duplicate page labels
  %%% ONLINE VERSION:
   
%  \usepackage[colorlinks=true,linkcolor=black,urlcolor=darkindigo,citecolor=bluegray]{hyperref}
%	\pdfcompresslevel=9
%	\pdfimageresolution=220
  %%% PRINT VERSION:
  \usepackage[colorlinks=true,linkcolor=black,urlcolor=black,citecolor=black]{hyperref}
  \pdfcompresslevel=9
  \pdfimageresolution=300
  \renewcommand{\\}{}
  \pdfinfo{
    /Author(\myname)
    /Title(\mytitle)
    /Subject(Dissertation of \myname)git
    }
    \hypersetup {
    pdfpagemode = {UseOutlines}, % prevents the automatic announcement of bookmarks
    pdftitle = {\mytitle},
    pdfsubject = {Dissertation of \myname},
    pdfauthor = {\myname},
    pdfdisplaydoctitle = true
  	}
\else
  \usepackage[draft]{hyperref}
\fi

% Look for images in subdirectory
%\graphicspath{{./images/}} = OLD

% Sections etc with different font settings
%\renewcommand{\sfdefault}{hls} % Lucida Bright font for sans serif
%\renewcommand{\sfdefault}{phv} % Helvetica font for sans serif
%\renewcommand{\sfdefault}{fvs} % Bera sans
%\usepackage[scaled=0.85]{berasans}
%\renewcommand{\sfdefault}{cmbr} % Computer Modern bright
\renewcommand{\sfdefault}{qhv} % Bera sans
\allsectionsfont{\textbf\sfdefault}
%\iffalse
\makeatletter

% Chapter Head
\label{Chapter Head}
\def\@makechapterhead#1{%
\thispagestyle{empty}
  \vspace*{-70\p@}%
  {\parindent \z@ \raggedright \normalfont
    \ifnum \c@secnumdepth >\m@ne
      \if@mainmatter
      	%	\par\vspace*{\fill}
	      	\par\vspace*{120\p@}
      		\IfSubStringInString{: }{#1}{%
			\begin{flushright}
			\fontsize{50}{40}\fontfamily{qhv}\textbf{\thechapter }\par\nobreak
			\vspace{10pt}
			\fontsize{20}{80}\fontfamily{qhv}\textbf{\BeforeSubString{: }{#1}}\par\nobreak
			%\fontsize{22}{26}\mdseries\textrm{-- \BehindSubString{: }{#1} --}\par\nobreak
			\fontsize{20}{80}\fontfamily{qhv}\textbf{\BehindSubString{: }{#1}}\par\nobreak
			  \end{flushright}
		}{
				
				\begin{flushright}
				\fontsize{50}{40}\fontfamily{qhv}\textbf{\thechapter }\par\nobreak
				\vspace{10pt}
				\fontsize{20}{80}\fontfamily{qhv}\textbf{#1}\par\nobreak
			     \end{flushright}
				\par\nobreak
        }
      \fi
  
        \vskip 30\p@
    %\vskip 40\p@
    
  }}
  
% Synopsis
\label{Chapter Synopsis}  
\newcommand{\shortsynopsis}[1]{		
%		\rule{\linewidth}{1pt}
%		\textit{\textbf{Synopsis.}}
		{\itshape #1}
%		\rule{\linewidth}{1pt}
		\newpage
}

%Subsections
\label{Subsections}
\renewcommand\section{\@startsection {section}{1}{\z@}%
                                   {-3.5ex \@plus -1ex \@minus -.2ex}%
                                   {2.3ex \@plus.2ex}%
                                   {\fontfamily{qhv}\Large\bfseries}}
\renewcommand\subsection{\@startsection{subsection}{2}{\z@}%
                                     {-3.25ex\@plus -1ex \@minus -.2ex}%
                                     {1.5ex \@plus .2ex}%
									%{\normalfont\Large\mdseries\rmfamily}
                                     {\fontfamily{qhv}\large\bfseries}}
\renewcommand\subsubsection{\@startsection{subsubsection}{3}{\z@}%
                                     {-1.75ex\@plus -1ex \@minus -.2ex}%
                                     {0.1ex \@plus .1ex}%
                                     {\fontfamily{qhv}\normalsize\bfseries}}
                                     %{\normalfont\large\itshape\rmfamily}}
\renewcommand\paragraph{\@startsection{paragraph}{4}{\z@}%
                                    {1.75ex \@plus1ex \@minus.2ex}%
                                    {-1em}%
                                    {\fontfamily{qhv}\small\bfseries}}
\renewcommand\subparagraph{\@startsection{subparagraph}{5}{\parindent}%
                                       {3.25ex \@plus1ex \@minus .2ex}%
                                       {-1em}%
                                      {\normalfont\normalsize\sffamily}}
                                      
                                      
%Table Font Size
\let\oldtabular\tabular
\renewcommand{\tabular}{\small\oldtabular}                                      
                                                                     
                                      
% Part Head
\def\@part[#1]#2{%
    \ifnum \c@secnumdepth >-2\relax
      \refstepcounter{part}%
      \addcontentsline{toc}{part}{\thepart\hspace{1em}#1}%
    \else
      \addcontentsline{toc}{part}{#1}%
    \fi
    \markboth{}{}%
    {\centering
     \interlinepenalty \@M
     \normalfont
     \ifnum \c@secnumdepth >-2\relax
       \textcolor{gray}{\fontsize{240}{288}\mdseries\textrm{\thepart}}%
       \par
     \fi
     \centering \normalfont
     \fontsize{30}{36}\mdseries\scshape\textrm{#2}\par}%
    \@endpart}
\def\@spart#1{%
    {\centering
     \interlinepenalty \@M
     \normalfont \Huge \bfseries \SS@parttitlefont {#1}\par}%
    \@endpart}

%\fi


% Bold captions
\makeatletter
\long\def\@makecaption#1#2{%
  \vskip\abovecaptionskip
  \vskip2mm
  \bfseries% added
  \centering% added
  \sbox\@tempboxa{#1: #2}%
%  \ifdim \wd\@tempboxa >\hsize
	\ifdim \wd\@tempboxa >0.95\textwidth
	  \begin{minipage}{0.95\textwidth}
    	\bfseries\small{#1:} \normalfont\small #2\par
    \end{minipage}
  \else
    \global \@minipagefalse
    %\hb@xt@\hsize{\hfil\box\@tempboxa\hfil}%
    \bfseries\small{#1:} \normalfont\small #2\par
  \fi
  \vskip\belowcaptionskip}
\makeatother
% END LATEX

% Make margins smaller
\addtolength{\oddsidemargin}{-.5in}
\addtolength{\evensidemargin}{-.5in}
\addtolength{\textwidth}{1in}
\addtolength{\topmargin}{-.25in}
\addtolength{\textheight}{.5in}

% Avoid lots of overfull \vbox messages with fancyhdr
\addtolength{\headheight}{3pt}
\pagestyle{fancy}
\sloppy

%\setlength{\abovecaptionskip}{.5ex}
\setlength{\abovecaptionskip}{1ex}
\setlength{\belowcaptionskip}{1ex}
%\setlength{\textfloatsep}{3ex}
%\renewcommand{\baselinestretch}{0.971}
% http://people.cs.uu.nl/piet/floats/node1.html
\renewcommand{\textfraction}{.15}
\renewcommand{\topfraction}{.8}     % vorher: .7
\renewcommand{\bottomfraction}{.8}  % vorher: .3
\renewcommand{\floatpagefraction}{.75}


% BEGIN LATEX
%% itemize mit kleinerem bullet
%\renewenvironment{itemize}{%
%  \begin{list}{\raisebox{.3ex}{\scriptsize$\bullet$}}{}%
%}{%
%  \end{list}%
%}
%% END LATEX
%
%% itemize mit kleinerem bullet und kleinerem Zeilenabstand
%\newenvironment{itemizex}{%
%  \begin{list}{\raisebox{.3ex}{\scriptsize$\bullet$}}%
%              {\setlength{\itemsep}{0pt}\setlength{\parsep}{0pt}}%
%}{%
%  \end{list}%
%}
%HEVEA \renewenvironment{itemizex}{\begin{itemize}}{\end{itemize}}
%HEVEA \renewcommand{\hfill}{ }

% enumerate with smaller parskip
%\newenvironment{enumeratex}{%
%  \begin{enumerate}\setlength{\itemsep}{0pt}\setlength{\parsep}{0pt}%
%}{%
%  \end{enumerate}%
%}

% enumerate with \item[somearg] - somearg is shown instead of the number
%\newenvironment{enumeratewithlabel}{\begin{enumerate}}{\end{enumerate}}
% HEVEA \renewenvironment{enumeratewithlabel}{\begin{description}}{\end{description}}

% description with smaller parskip
%\newenvironment{descriptionx}{%
%  \begin{description}\setlength{\itemsep}{0pt}\setlength{\parsep}{0pt}%
%}{%
%  \end{description}%
%}

\newenvironment{chapterdescription}[1]{%
	%\setlength{\parskip}{-2pt plus 2pt minus 1pt}%
	%\setlength{\parskip}{3pt}%
	\vskip 2pt
	\leftskip=8mm
	\rightskip=\leftskip
	%\begin{center}%
		%\begin{minipage}{0.85\textwidth}%
		%\begin{minipage}{150mm}%
			%\noindent
			\textbf{#1:}%
}{%
		%\end{minipage}%
	%\end{center}%
	\vskip 2pt
	\par
	\leftskip=0mm
	\rightskip=\leftskip
	%\setlength{\parskip}{7pt}
	%\setlength{\parskip}{7pt plus 2pt minus 1pt}
}

\newenvironment{contribution}{%
	\begin{center}
		\setlength{\arrayrulewidth}{5pt}
		\renewcommand{\arraystretch}{1.0}
		\arrayrulecolor{gray}
		\begin{tabular}{ | m{0.85\textwidth} }
			%\textsc{\textbf{Contribution Statement:}}
			\textbf{Contribution Statement:}
}{%
		\end{tabular}
		\setlength{\arrayrulewidth}{1pt}
		\renewcommand{\arraystretch}{0.0}
		\arrayrulecolor{black}
	\end{center}
}

\newenvironment{definition}[1]{%
	\begin{center}
		\setlength{\arrayrulewidth}{1pt}
		%\renewcommand{\arraystretch}{2.0}
		\arrayrulecolor{darkgray}
		\begin{tabular}{ | m{0.85\textwidth} | }
			\hline%
			\rowcolor{lightgray}\textbf{#1:}
}{%
			\\
			\hline%
		\end{tabular}
		\setlength{\arrayrulewidth}{1pt}
		%\renewcommand{\arraystretch}{0.0}
		\arrayrulecolor{black}
	\end{center}
}

\newenvironment{indented}[1]{%
	\leftskip=#1
	\rightskip=\leftskip
}{%
	\par
	\leftskip=0mm
	\rightskip=\leftskip
}


% -- Summary ---------------------------------------------------------------
% Creates a nicely indented summary paragraph
% Always starts with the word summary printed bold
\newcommand{\summary}[1]{
	\vspace{2mm}
	%\hspace*{5mm}
	\begin{center}
		\begin{minipage}{0.85\linewidth}
				\textbf{Summary:} #1
		\end{minipage}
	\end{center}
}	
% -- Summary ---------------------------------------------------------------


% \includegraphics of png/jpeg files, hevea-aware
\newcommand{\includepng}[2][]{\includegraphics[#1]{#2.png}\\}
% HEVEA \renewcommand{\includepng}[2][]{\imgsrc{#2.png}\\}
\newcommand{\includejpeg}[2][]{\includegraphics[#1]{#2.jpeg}\\}
% HEVEA \renewcommand{\includejpeg}[2][]{\imgsrc{#2.jpeg}\\}
\newcommand{\includepngx}[2][]{\includegraphics[#1]{#2.png}} % no newline
% HEVEA \renewcommand{\includepngx}[2][]{\imgsrc{#2.png}\\}
\newcommand{\includepdfpng}[2][]{\includegraphics[#1]{#2.pdf}\\} % PDF for LaTeX
% HEVEA \renewcommand{\includepng}[2][]{\imgsrc{#2.png}\\} % PNG for HEVEA

% Definition of page heading
\label{Page Heading}
%\pagestyle{fancyplain}
\renewcommand{\chaptermark}[1]{\markboth{ #1}{}}
\renewcommand{\sectionmark}[1]{\markright{ #1}}
\newcommand{\phstyle}{\fontfamily{qhv}\fontsize{10}{14}\bfseries}
%\lhead[\fancyplain{}{\bfseries\thepage}]{\fancyplain{}{\bfseries\rightmark}}
%\rhead[\fancyplain{}{\bfseries\leftmark}]{\fancyplain{}{\bfseries\thepage}}
\fancyhf{}
%EXAMPLE: \lhead[lh-even]{lh-odd}
\lhead[\phstyle]{}
\lfoot[\fancyplain{}\phstyle\thepage]{}
\rfoot[]{\fancyplain{}\phstyle\thepage}
\rhead[]{\fancyplain{}\phstyle\leftmark}
\cfoot{\fancyplain{}}


\author{\myname\\\small tobias.seitz@ifi.lmu.de}
\title{\mytitle\\
	\mysubtitle\\}
\date{\today}


%--Inline Comments-------------------------------------------------------------
\newcommand{\inlinecomment}[3]{
	\begin{center}
    \fbox{
      \begin{minipage}{.85\linewidth}
        \textcolor{#1} {{\textbf #2:} \textsf{#3}}
      \end{minipage}
      }
  \end{center}	
}
%--Inline Comments-------------------------------------------------------------

%--Personalized Comments-------------------------------------------------------
%\newcommand{\patrick}[1]{\inlinecomment{red}{Patrick}{#1}}
%\newcommand{\tico}[1]{\inlinecomment{green}{Tico}{#1}}
\newcommand{\comment}[1]{\inlinecomment{red}{\textbf{COMMENT}}{#1}}
\newcommand{\story}[1]{\inlinecomment{blue}{\textbf{STORY}}{#1}}
%--Personalized Comments-------------------------------------------------------

\newcommand{\TCop}{\textsuperscript{\textcopyright}}

%--Chapter
%\newcommand\Chapter[2]{
% \chapter[#1: {\itshape#2}]{#1\\[1pt]\Large\itshape#2}
%}
%\titlespacing*{\chapter}
%{0pt}{5.5ex plus 1ex minus .2ex}{4.3ex plus .2ex}
%--Summary----------------------------------------------------------------------

\newcommand{\summarize}[1]{
	\vspace{2mm}
	%\hspace*{5mm}
	%\begin{center}
	\begin{flushright}
		\begin{minipage}{0.85\linewidth}
			\begin{flushright}	
				\color{darkgray}\fontsize{16}{16}\textbf{-- Summary --\\} \fontsize{14}{16}\rmfamily\textit{#1}
			\end{flushright}
		\end{minipage}
	\end{flushright}
	%\end{center}
}

%--Synopsis---------------------------------------------------------------------

\newcommand{\synopsis}[1]{
%\textcolor{darkgray}{\hrule }
\hrule
\vspace{0.5cm}

	\section*{Synopsis}		
	\textit{#1}

\vspace{0.7cm}
\hrule

%\vspace{1cm}

\clearpage
}



\newcommand{\introstory}[1]{
	\noindent\makebox[\textwidth][c]{
		\begin{minipage}{.93\textwidth}
			\textit{#1}
		\end{minipage}
	}
}

\newcommand{\researchquestion}[2]{
	\vspace{0.5cm}
	\noindent\makebox[\textwidth][c]{
		\begin{minipage}{.9\textwidth}
			\textbf{#1}\\
					
			\textbf{#2}			
		\end{minipage}
	}
}


\newcommand{\cmark}{\ding{51}}%
\newcommand{\xmark}{\ding{55}}%

\makeatletter
\newcommand*{\compress}{\@minipagetrue}
\makeatother

%ANECDOTE
\label{Anecdote}
\newcommand{\anecdote}[2]{
\par
\begingroup
\leftskip2em
\rightskip\leftskip
\textit{#1 }
\begin{flushright}
\rightskip2em
\textit{#2}
\end{flushright}

\par
\endgroup

}