% DISCLAIMER

\section*{Disclaimer}

\subsubsection{Personal Contribution Statement}
In this thesis, I discuss projects that I carried out in collaboration with several students and colleagues. Most of them were accomplished as part of the bachelor or master theses of talented students that I was lucky to have supervised. In the following, I declare my personal contribution to each of these projects. 

\textbf{Chapter \ref{chap:policies_reuse}} is based on project work for the course Advanced Topics in HCI (in 2016). I developed the idea and project scope, and the work was conducted in collaboration with Manuel Hartmann, Jakob Pfab, and Samuel Souque. In regular meetings, each step was jointly discussed and agreed upon. I set the general course, and the three students carried out the audit of the policies. After the initial analysis, I evaluated further details of the dataset. The work was published as an extended abstract at the \textit{ACM CHI Conference on Human Factors in Computing Systems} (CHI '17) \cite{Seitz2017PoliciesReuse} with Manuel, Jakob, and Samuel as co-authors. I reworked their initial draft several times and revised the paper based on the reviewers' feedback. 

\textbf{Chapter \ref{chap:pws_and_personality}} reports on three studies that were carried out as part of the bachelor theses of three students: Timo Erdelt \cite{Erdelt2017BA}, Paul Huber \cite{Huber2016BA}, and Aline Neumann \cite{Neumann2017BA}. I developed the idea to study personality in the domain of password authentication. In regular meetings, we discussed each step. However, I provided guidance and was responsible for the key decisions about the study design, execution, and evaluations. For statistical analyses, we also consulted the LMU's internal statistic consultancy (StabLab), as well as Clemens Stachl from the psychology department. I validated and analyzed the data set independently from the students.

\textbf{Chapter \ref{chap:mental_models_pwm}} is based on a project that was part of Martin Prinz' Master thesis \cite{Prinz2017Thesis}. I developed the research questions, project roadmap, and methodology. Martin then carried out the interviews and created a large part of the mental model structure, whereas I connected the dots in broader analyses. We met regularly to discuss the progress. The results were published as an extended abstract at the Symposium on Usable Privacy and Security (SOUPS '17) \cite{Prinz2017MentalModel}. I revised Martin's draft in several iterations and redacted the paper based on the reviewers' feedback. 

\textbf{Chapter \ref{chap:feedback_modalities}} encompasses two projects from the bachelor theses of Caroline Olsienkiewicz \cite{Olsienkiewicz2016BAThesis} and Katharina Schwarz \cite{Schwarz2016BAThesis}. For both projects, I developed the initial ideas, provided guidance, and made the key decisions about the scope, methodology, and analyses. The specifics were always jointly discussed and agreed upon. Both Caroline and Katharina crafted wireframes, implemented prototypes, and executed the studies. I analyzed the data independently and derived broader concepts and paradigms. 

\textbf{Chapter \ref{chap:decoy}} is partially based on Stefanie Meitner's bachelor thesis \cite{Meitner2016BADecoy}. I had already developed the concepts and especially the choice architectures (in part with Isabel Schönewald) before Stefanie joined the project. I provided guidance, and defined the project scope and methodology. Stefanie implemented the prototype and executed the data collection before I performed the data analyses. The results have been published at the European Workshop on Usable Security (EuroUSEC '16) with Emanuel von Zezschwitz, Stefanie Meitner, and Heinrich Hußmann as co-authors. I wrote the main corpus of the paper and revised the submission based on the feedback from the reviewers, especially our shepherd Karen Renaud. 

\textbf{Chapter \ref{chap:emojipasswords}} is based on Florian Mathis' bachelor thesis \cite{Mathis2017BA}. I developed the original idea, provided guidance through the project, and made the key decisions. Specific aspects were always jointly discussed and agreed upon. Florian implemented the prototype and I performed source code reviews. He also was the experimenter and I shadowed him in user sessions. Each of us independently coded qualitative aspects and created the codebook. I also performed further quantitative analyses with the dataset. The results have been published at the Australian Conference on Human-Computer Interaction (OzCHI '17) with Florian Mathis and Heinrich Hußmann as co-authors. I wrote the paper and revised it based on feedback from my co-authors before and from the reviewers after the submission. Moreover, I redacted the paper for the final submission. 

\textbf{Chapter \ref{chap:perdespassup}} includes a design exercise that was carried out together with Magdalena Siferlinger and Martin Prinz as part of their theses \cite{Prinz2017Thesis, Siferlinger2017BAThesis}. I developed the idea for the exercise and defined the goals and scope of the project. I provided guidance and discussed all steps with them in weekly meetings. The students executed the interviews and created the prototypes, while I independently analyzed the data and drew inferences from it.