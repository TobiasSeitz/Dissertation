%!TEX root = ../../diss.tex

\chapter[Introduction]{Introduction}\label{chap:intro}


\section{Motivation}
personalized systems


everybody saves data online, sensitive and intimate documents that should be protected.
if you don't store data online yourself, you can be sure that there is some other party
that stores data about you, e.g. doctors, insurance agencies, airlines, whatever. 

for most consumer oriented services, providers usually ask people to sign up to use
all of what the service has to offer. in the top 100 websites in germany 83 offer public sign ups and xx have an offline activation workflow but still offer a kind of account (e.g. banks, phone network carriers, pay tv providers, insurance companies). 

most of the time users are asked to create a password to access their account, make changes to their personal data or generally use a service. 




\subsection{Authentication Costs Time}
if you log into a service every day and it takes 5 seconds to type in your username and password ()


\subsection{Weak Passwords Cost Money}

\section{Research Objectives}
\subsection{Problem Statement}
\subsection{Better Understanding of Password Usability}
\subsection{Making Users' Lives a Little Easier}

\section{Agenda: Claims to Support in this Thesis}

\begin{description}
\item[Perception of Password Strength] Over the past decade, users received many hints and advice to construct strong passwords. Their understanding of a secure password has changed and is sometimes wrong. We show that this is the case in section @TODO REF SEC

\item[Password Composition Policies] As many web sites require or allow some kind of registration, their operators implement different password composition policies. We show that the criteria are manifold and largely inconsistent. Consequently, users approach enrollment with their preferred password, and are forced to apply heuristics to modify the password, depending on the policy in use. 

\item[Password Value] We present a framework to assess the value a user associates with a specific password. The users might not realize that they re-use the password for accounts with different values. Knowledge about a password's value is important to design persuasive strategies to protect it, e.g. by discouraging its usage on low value accounts. (See PSST).
\end{description}


\section{Main Contributions}
\subsection{Insights into the Psychology of Passwords}
\subsection{Designing Around Password Reuse}
\subsection{When and Why to Apply Nudges}
\subsection{Holistic Password Support}


\section{Thesis Overview}
\textit{Chapter 1:}

\textit{Chapter 2:}

\textit{Chapter 3:}

\textit{Chapter 4:}

\textit{Chapter 5:}

\textit{Chapter 6:}

\textit{Chapter 7:}




