%!TEX root = ../../diss.tex
\chapter[Introduction]{Introduction}\label{chap:intro}
% INTRO -- 8-10 pages incl. overview. 

% a day in the life of a password user, i.e. everyone. 
% hassle free.
%We may not realize it anymore, but most of us need multiple passwords every day. 
We have become accustomed to using multiple passwords every day: We enter a four-digit PIN to unlock our phone to check incoming messages even before breakfast. Paywalls shield the news of the day, so they force us to log into our favorite news website before we can read them on the way to work. Once arrived, our fingers automatically type the password to unlock the computer. A colleague requests access to a client's platform, so we write down the credentials or share them via a password manager that itself is password protected. When we come home, we find the kids have used the family's tablet to log into their social media accounts, so we need to re-authenticate if we want to check our own. Interacting with systems relying on password authentication has become ubiquitous, and we merely carry out the task. Passwords are the de facto standard when it comes to controlling access to resources on the Internet. While it may seem straightforward to deal with passwords because we are so acquainted with them, there is a wide range of problems they entail:
% all the hassle.
The phone requests the PIN whenever we take it out of our pocket, so we spend considerable time per day authenticating \cite{Harbach2016HardLockLife}. The news website informs us that our password has expired, so we need to pick a new one that is not part of the ones we had used before. The work PC requires not only our credentials but also a one-time password that our phone generates for us every sixty seconds. At one point, the colleague leaves the company, so we have to request a new set of credentials from the client to maintain confidentiality. And the kids used the tablet at home to shop online because we had stored the password in the browser for convenience. So, we delete the stored password but fail to recall it a week later, because we had not typed it for a very long time. These and other situations lead to users feeling a considerable burden generated by password authentication. 

\section{The State of the World}
In a wide sense, password authentication encompasses digit-only \glspl{PIN}, graphical schemes like the Android screen lock pattern (see Section \ref{sec:rw:graphical_pws}), and alphanumeric passwords consisting of letters and digits. Throughout this thesis, however, we focus on the latter because they still are the go-to method on the Internet: In 2007, users had around 25 accounts and roughly six distinct passwords on average \cite{Florencio2007LargeScaleStudyPasswordHabits}, while more recent numbers suggest that users keep twelve unique passwords for different purposes \cite{Wash2016UnderstandingPasswordChoices}. With the growing number of accounts the issues around passwords amplify: users pick weak passwords, write them down in unprotected locations, reuse many of them, forget the exotic ones, and share credentials with other people \cite{Stobert2014PasswordLifeCycle}.  

Combating the risks entailed by these issues, researchers and practitioners have developed numerous approaches to support users in password authentication. Four central themes have emerged in their efforts: education, enforcement, assistance, and persuasion. 

\subsubsection{Education}
Weak password practices were attributed to a lack of understanding of their consequences for the longest time \cite{Sasse2001WeakestLink}. Thus, first approaches to mitigate the situation leaned on educating users through trainings, and textual instructions \cite{Herley2009SoLongThanksExternalities}. Explaining all risks and defense strategies to users in lengthy prose is doomed to fail, because password security is a secondary goal that is dominated by a primary task, e.g., reading emails \cite{Whitten1999WhyJohnnyCantEncrypt}. Nevertheless, carefully crafted advice and explanations may be a viable strategy for those who actively seek information \cite{Sasse2005UsableSecurityPosition, Ur2017DataDrivenPWMeter}. 

\subsubsection{Enforcement}
Since the inception of passwords in the 1960s, users have tried to create memorable passwords that are often based on simple dictionary words. To combat low complexity, enforcing password rules through policies is commonplace. Passwords must meet length and complexity requirements, e.g., a certain number of uppercase letters, digits, or symbols. The choice of a specific policy is not trivial for service providers, and many organizations make poor tradeoffs \cite{Florencio2010WhereDoPoliciesComeFrom, Shay2016DesigningPasswordPolicies}. The plethora of policies even fosters insecure password practices, because users try to find easy ways to comply with the rules, which they do in very predictable ways \cite{Inglesant2010TrueCostOfUnusablePolicies, Komanduri2011OfPasswordsAndPeople, Ur2015PWCreationLab}. 

\subsubsection{Assistance}
Users have to deal with a high number of passwords that also interfere with each other, so people reuse passwords and write them down on paper or digital files. To lower the risks entailed by these coping strategies, assistive tools like \glspl{PWM} and password generators automate certain interactions to boost both security and usability. However, they come at a price, e.g., lock-in effects to a particular vendor that make it difficult to move to a different tool in the future. Thus, many users rationally refrain from adopting them \cite{CSID2012PasswordHabits}. 

\subsubsection{Persuasion}
Finally, the youngest strategy to support users in password authentication is based on principles of \textit{persuasive technology}. Fogg defined this paradigm as ``any interactive computing system designed to change people's attitudes and behaviors'' \cite[p. 1]{Fogg2002Persuasive}. While assistive systems often that goal, persuasive technology tries to create sustainable impact even when the assistive trigger is absent. Therefore, such interventions often come as \textit{behavior-change support system} \cite{Oinas-Kukkonen2013BCSS}. In the realm of authentication, password meters are the most representative form of persuasive interventions. Those \gls{UI} elements often appear on registration pages of web services, and give feedback on the strength of a user-selected password. They often implement various \textit{nudges}, i.e. a small, transparent attempt to change behaviors \cite[p. 4]{Thaler2008Nudge}, to convince the user to pick a more suitable password. The user stays in control and is free to move on without following the advice, which is perhaps the biggest challenge in the design of persuasive password support. So far, the use of nudges has shown mixed results in empirical research \cite{Egelman2013DoesMyPasswordGoUpToEleven, Renaud2017LessonsLearnedNudges, Ur2017DataDrivenPWMeter}, and the spectrum of persuasive interventions in the wild is fairly narrow. 

\section{Problem Statement and Research Objectives}
Password-related challenges and risks for the users are at the heart of this dissertation. The balance between usability and security, especially for passwords, has been under investigation for several decades. This allows us to observe tectonic shifts in the hassles that users have to deal with. 
% problem statement:
% central problem theme: mental models are not well understood
% why this is important: ``form follows function'' if we don't get the function right, the form is negligible. 
% approach: different aspects: mm of pw strength and personality influence, mm of coping strategies
The notion of the inconsiderate user, who is the ``weakest link'' in the authentication chain and notoriously refuses security measures, has started to crumble: There is considerable evidence that many users want more control over their security and that they are willing to sacrifice usability for it \cite{Kessem2018IBMFutureIdentity}. This might not even be necessary, or lie in their best interest, which I illustrate below. 
% -- could be due to raised awareness about data breaches etc.
\subsection{Balancing the Costs}
Changing the status quo of the authentication world is a game-theoretic problem \cite{Bonneau2015ImperfectAuthentication}. It involves risks and opportunities that need to be balanced in terms of their costs for different ``players''. 

\subsubsection{Usable but Risky Password Practices}
Fortunately, the costs of being attacked remain hypothetical for many users \cite{Herley2015Counterfactuals}. However, there is a growing number of people who have experienced an attack with all its consequences \cite{BKA2016Bundeslagebild}. 
% money loss
First, weak password practices can cause financial damage for end-users, if an attacker manages to impersonate them and, e.g., withdraws money from bank accounts. At the moment, attacks on digital wallets containing cryptocurrency are highly lucrative, so protecting these assets with strong passwords is vital\footurl{https://blog.dashlane.com/cryptocurrency-exchange-password-power-rankings-2018/}{24.03.2018}. Although social engineering (see Section \ref{sec:rw:attack_vectors}) is a central threat for companies, weak password practices among employees also generate considerable financial losses\footurl{https://www.helpnetsecurity.com/2017/09/19/infosec-weakest-links/}{09.04.2018}. 
% identity theft
Herley \etal noted that \textit{``money is the most obvious loss, but time, frustration and reputation are also at stake''} \cite{Herley2012PersistenceOfPasswords}, beside the emotional distress \cite{Shay2014ReligiousAunt}. Accounts that have a weak password are more likely to be hijacked \cite{Wang2016TargetedGuessingUnderestimated}, and it often takes victims painful effort to recover from identity theft\footurl{https://www.businesswire.com/news/home/20151006006149/en/Latest-Data-Breach-Spotlights-Identity-Restoration}{11.01.2018}.

\subsubsection{Striving for Strong Passwords}
At the other end of the security-usability spectrum, strong passwords can be a large burden for users, too. Typing a long complex password occupies much time and is error prone \cite{Shay2014CanLongPasswordsBeSecureAndUsable}. They are especially tedious to enter on mobiles or devices that were not designed for a lot of text input, e.g., smart TV sets \cite{Melicher2016UsabilityMobileTextPasswords}. Striving for more security, service providers often force users to reset passwords after a given period. Password expiration is responsible for many support desk calls whose resolutions are costly \cite{Adams1999UsersEnemy, Sasse2005UsableSecurityPosition}. Moreover, most people are incapable of creating and memorizing a strong, unique password for every single account, so it is unrealistic to expect that they will do so without the use of external aids like handwritten notes or password managers. The cost of those is a dependency that the users did not ask for. Finally, strong passwords are no panacea in boosting online security. They do not stand a chance in fending off social engineering attacks where the victim unwillingly surrenders the password in plain text. Imposing hard rules on users may blur their view for such risks. 

\subsection{Improving an Innately Annoying Interaction}
% THIS IS THE MAIN POINT ACTUALLY THAT DIRECTLY LEADS TO THE RESEARCH QUESTIONS
Passwords are annoying for users, and there is no easy solution to alleviate this situation. It is impossible to remove all usability pain points \cite{Bonneau2012ReplacePasswords}. No alternative can fully replace passwords, either (see Section \ref{sec:rw:authentication_without_pws}). Thus, Herley and Van Oorschot point out that \textit{``supporting passwords better is a vast opportunity for improvement''} \cite{Herley2012PersistenceOfPasswords}, because making even a small change can have an impact on a vast number of users. 

So far, persuasive strategies have produced mixed results regarding their efficacy in supporting users with passwords. This could be due to several issues: Mental models and coping strategies evolve over time, which has not been considered in persuasive interventions. Much of the past research dealt with one-shot triggers in isolation, but many context factors were left out. In an analogy to the design principle ``form follows function'', a better understanding of the functions of user support can help design assistive and persuasive solutions (forms). Moreover, many highly effective nudges from other domains have not been adapted for password support, so we simply may not have found the best intervention, yet. 
% it is hard to study passwords

\subsection{Research Questions}
The aforementioned gaps in the literature are the foundation of our research questions:
\begin{itemize}
	\item[\textbf{RQ1}] What is the role of psychological factors and mental models for password selection and coping strategies?
	\item[\textbf{RQ2}] How can password authentication be simplified for users? 
	\item[\textbf{RQ3}] How can we design persuasive strategies to support users in any password-related tasks?
\end{itemize}

\section{Main Contributions}
% see summary, but describe the abstract results / implications rather than rolling them up all over again.
% around one page.
\subsubsection{A Better Understanding of Password Psychology}

\subsubsection{Persuasion Through Simplification}

\subsubsection{A Structured Process for the Design of Persuasive Password Support}



\section{Thesis Structure}
% can be up to two pages.
\paragraph{Part I: Foundations of Usable Authentication}

\textit{Chapter 2:} %Foundations and general background on passwords
How have we come to this place?
What are the risks of using passwords?
What is a strong, what is a weak password?
Why do we still need passwords when we have all these other authentication schemes?

\textit{Chapter 3:} % User side
How do we conduct experiments with humans and ethics in mind?
How do users cope with passwords? What makes their behavior risky?
What can we do to steer people away from risky behavior? 

\textit{Chapter 4:} 
%Summary of related work and derivation of open questions.
What are the main issues that have been previously identified?
What are the things that are not well understood and need more work?


\paragraph{Part II: Evaluating the Problem}
\textit{Chapter 5:} %Mental models of password strength (PASDJO)
How well can users gauge password strength?
Can a game tell us this?
How much data is necessary?
How well do people learn from playing the game?

\textit{Chapter 6:} %An audit of password policies in terms of reusability
What kind of passwords can be reused across the board, and which can't?

\textit{Chapter 7:} % Password personality
Does personality have any association with password related behavior, attitudes, and mental models?

\textit{Chapter 8:} %Mental Models of password managers (small scale)
Why are people (not) using password managers?
How do they make sense of their functionality?
%What kinds of risks are involved?


\paragraph{Part III: Persuasive Design Strategies}
\textit{Chapter 9:} %Exploring needs in persuasive feedback (enriching understanding of mental models)
What are users' mental models of current persuasive feedback?
What are users' expectations and needs for future persuasive feedback?

\textit{Chapter 10:} %Decoy effect
Does the decoy effect work to make stronger passwords more attractive?
How effective is feed-forward in combination with feedback?

\textit{Chapter 11:} %Empowerment through emojis. 
How usable are emoji-passwords?
What are the risks and potentials of emoji passwords?

\paragraph{Part IV: Synthesis}
\textit{Chapter 12:} %P4P
How should we design future persuasive password support?
Can we get some structure into the process? 

\textit{Chapter 13:} %Summary
What have we learned? 
What are the caveats?

\textit{Chapter 14:} %The End
What will the future bring?
What needs to change to make users' lives easier, to reach them more effectively, and to ease the transition into a password-less world?

\section{Style Choices}
\begin{itemize}
\item \textbf{Singular They}\footurl{https://en.wikipedia.org/wiki/Singular_they}{06.01.2018} Throughout this dissertation pronouns are used in the plural even when speaking about an individual, e.g. ``the user'' is mostly referred to as ``they'' instead of ``he'' or ``she'', to avoid discrimination of certain demographic groups. 
\item As is common in HCI literature, the author also utilizes ``We'' instead of ``I'' to acknowledge the work of any collaborators. In later more opinionated parts, the explicit usage of ``I'' intends to communicate the subjective nature of thoughts and interpretations.
\item Throughout the thesis, footnotes are excessively used to link to web content. Publications in scientific archives appear in the list of references at the end of the thesis.
\end{itemize}
