%!TEX root = ../../diss.tex
\chapter[Introduction]{Introduction}\label{chap:intro}
% INTRO -- 8-10 pages incl. overview. 

% a day in the life of a password user, i.e. everyone. 
% hassle free.
%We may not realize it anymore, but most of us need multiple passwords every day. 
We have become accustomed to using multiple passwords every day: We enter a four-digit PIN to unlock our phone to check incoming messages even before breakfast. Paywalls shield the news of the day, so they force us to log into our favorite news website before we can read them on the way to work. Once arrived, our fingers automatically type the password to unlock the computer. A colleague requests access to a client's platform, so we write down the credentials or share them via a password manager that itself is password protected. When we come home, we find the kids have used the family's tablet to log into their social media accounts, so we need to re-authenticate if we want to check our own. Interacting with systems relying on password authentication has become ubiquitous, and we merely carry out the task. Passwords are the de facto standard when it comes to controlling access to resources on the Internet. While it may seem straightforward to deal with passwords because we are so acquainted with them, there is a wide range of problems they entail:
% all the hassle.
The phone requests the PIN whenever we take it out of our pocket, so we spend considerable time per day authenticating \cite{Harbach2016HardLockLife}. The news website informs us that our password has expired, so we need to pick a new one that is not part of the ones we had used before. The work PC requires not only our credentials but also a one-time password that our phone generates for us every sixty seconds. At one point, the colleague leaves the company, so we have to request a new set of credentials from the client to maintain confidentiality. And the kids used the tablet at home to shop online because we had stored the password in the browser for convenience. So, we delete the stored password but fail to recall it a week later, because we had not typed it for a very long time. These and other situations lead to users feeling a considerable burden generated by password authentication. 

\section{The State of the World}
In a wide sense, password authentication encompasses digit-only \glspl{PIN}, graphical schemes like the Android screen lock pattern (see Section \ref{sec:rw:graphical_pws}), and alphanumeric passwords consisting of letters and digits. Throughout this thesis, however, we focus on the latter because they still are the go-to method on the Internet: In 2007, users had around 25 accounts and roughly six distinct passwords on average \cite{Florencio2007LargeScaleStudyPasswordHabits}, while more recent numbers suggest that users keep twelve unique passwords for different purposes \cite{Wash2016UnderstandingPasswordChoices}. With the growing number of accounts the issues around passwords amplify: users pick weak passwords, write them down in unprotected locations, reuse many of them, forget the exotic ones, and share credentials with other people \cite{Stobert2014PasswordLifeCycle}. These behaviors are referred to as \textit{password coping strategies}.

Combating the risks entailed by these issues, researchers and practitioners have developed numerous approaches to support users in password authentication. Four central themes have emerged in their efforts: education, enforcement, assistance, and persuasion. 

\subsubsection{Education}
Weak password practices were attributed to a lack of understanding of their consequences for the longest time \cite{Sasse2001WeakestLink}. Thus, first approaches to mitigate the situation leaned on educating users through trainings, and textual instructions \cite{Herley2009SoLongThanksExternalities}. Explaining all risks and defense strategies to users in lengthy prose is doomed to fail, because password security is a secondary goal that is dominated by a primary task, e.g., reading emails \cite{Whitten1999WhyJohnnyCantEncrypt}. Nevertheless, carefully crafted advice and explanations may be a viable strategy for those who actively seek information \cite{Sasse2005UsableSecurityPosition, Ur2017DataDrivenPWMeter}. 

\subsubsection{Enforcement}
Since the inception of passwords in the 1960s, users have tried to create memorable passwords that are often based on simple dictionary words. To combat low complexity, enforcing password rules through policies is commonplace. Passwords must meet length and complexity requirements, e.g., a certain number of uppercase letters, digits, or symbols. The choice of a specific policy is not trivial for service providers, and many organizations make poor tradeoffs \cite{Florencio2010WhereDoPoliciesComeFrom, Shay2016DesigningPasswordPolicies}. The plethora of policies even fosters insecure password practices, because users try to find easy ways to comply with the rules, which they do in very predictable ways \cite{Inglesant2010TrueCostOfUnusablePolicies, Komanduri2011OfPasswordsAndPeople, Ur2015PWCreationLab}. 

\subsubsection{Assistance}
Users have to deal with a high number of passwords that also interfere with each other, so people reuse passwords and write them down on paper or digital files. To lower the risks entailed by these coping strategies, assistive tools like \glspl{PWM} and password generators automate certain interactions to boost both security and usability. However, they come at a price, e.g., lock-in effects to a particular vendor that make it difficult to move to a different tool in the future. Thus, many users rationally refrain from adopting them \cite{CSID2012PasswordHabits}. 

\subsubsection{Persuasion}
Finally, the youngest strategy to support users in password authentication is based on principles of \textit{persuasive technology}. Fogg defined this paradigm as ``any interactive computing system designed to change people's attitudes and behaviors'' \cite[p. 1]{Fogg2002Persuasive}. While assistive systems often that goal, persuasive technology tries to create sustainable impact even when the assistive trigger is absent. Therefore, such interventions often come as \textit{behavior-change support system} \cite{Oinas-Kukkonen2013BCSS}. In the realm of authentication, password meters are the most representative form of persuasive interventions. Those \gls{UI} elements often appear on registration pages of web services, and give feedback on the strength of a user-selected password. They often implement various \textit{nudges}, i.e. a small, transparent attempt to change behaviors \cite[p. 4]{Thaler2008Nudge}, to convince the user to pick a more suitable password. The user stays in control and is free to move on without following the advice, which is perhaps the biggest challenge in the design of persuasive password support. So far, the use of nudges has shown mixed results in empirical research \cite{Egelman2013DoesMyPasswordGoUpToEleven, Renaud2017LessonsLearnedNudges, Ur2017DataDrivenPWMeter}, and the spectrum of persuasive interventions in the wild is fairly narrow. 

\section{Problem Statement and Research Objectives}
Password-related challenges and risks for the users are at the heart of this dissertation. The balance between usability and security, especially for passwords, has been under investigation for several decades. This allows us to observe tectonic shifts in the hassles that users have to deal with. 
% problem statement:
% central problem theme: mental models are not well understood
% why this is important: ``form follows function'' if we don't get the function right, the form is negligible. 
% approach: different aspects: mm of pw strength and personality influence, mm of coping strategies
The notion of the inconsiderate user, who is the ``weakest link'' in the authentication chain and notoriously refuses security measures, has started to crumble: There is considerable evidence that many users want more control over their security and that they are willing to sacrifice usability for it \cite{Kessem2018IBMFutureIdentity}. This might not even be necessary, or lie in their best interest, which I illustrate below. 
% -- could be due to raised awareness about data breaches etc.
\subsection{Balancing the Costs}
Changing the status quo of the authentication world is a game-theoretic problem \cite{Bonneau2015ImperfectAuthentication}. It involves risks and opportunities that need to be balanced in terms of their costs for different ``players''. 

\subsubsection{Maintaining Usable Password Practices}
Fortunately, the costs of being attacked remain hypothetical for many users \cite{Herley2015Counterfactuals}. However, there is a growing number of people who have experienced an attack with all its consequences \cite{BKA2016Bundeslagebild}. 
% money loss
First, usable but risky password practices, like excessive reuse and obvious password choices, can cause financial damage for end-users. For instance, an attacker who manages to impersonate a user might be able to withdraw money from bank accounts. At the moment, attacks on digital wallets containing cryptocurrency are highly lucrative, so protecting these assets with strong passwords is vital\footurl{https://blog.dashlane.com/cryptocurrency-exchange-password-power-rankings-2018/}{24.03.2018}. 
% identity theft
Herley \etal noted that \textit{``money is the most obvious loss, but time, frustration and reputation are also at stake''} \cite{Herley2012PersistenceOfPasswords}, beside the emotional distress \cite{Shay2014ReligiousAunt}. Accounts that have a weak password are more likely to be hijacked \cite{Wang2016TargetedGuessingUnderestimated}, and it often takes victims painful effort to recover from identity theft\footurl{https://www.businesswire.com/news/home/20151006006149/en/Latest-Data-Breach-Spotlights-Identity-Restoration}{11.01.2018}. Although social engineering (see Section \ref{sec:rw:attack_vectors}) is a central threat for companies, weak password practices among employees also generate considerable financial losses\footurl{https://www.helpnetsecurity.com/2017/09/19/infosec-weakest-links/}{09.04.2018}. 

\subsubsection{Striving for Stronger Password Practices}
At the other end of the security-usability spectrum, moving to strong password practices often inflates usability challenges for users that are largely neglected by security experts. For instance, typing an overly complex password takes long and is error prone \cite{Shay2014CanLongPasswordsBeSecureAndUsable}. Such passwords are especially tedious to enter on mobile phones or devices that were not originally designed for text input, e.g., smart TV sets \cite{Melicher2016UsabilityMobileTextPasswords}. Moreover, most people are incapable of creating and memorizing a strong, unique password for every single account on their own. So, it is unrealistic to expect that they will do so without the use of external aids like handwritten notes or password managers. The cost of using such methods is a dependency on the tools that users did not ask for. In any case, strong passwords are no panacea in boosting online security. They do not stand a chance in fending off social engineering attacks where the victim unwillingly surrenders the password in plain text. In an effort to alleviate the risks of unknowingly forfeited credentials, service providers often require users to reset passwords after a given period. Such expiration policies are rather ineffective \cite{Chiasson2015QuantifyingExpiration} and responsible for many support desk calls whose resolutions are costly \cite{Adams1999UsersEnemy, Sasse2005UsableSecurityPosition}. 

\subsection{Improving an Innately Annoying Interaction}
Passwords are annoying for users, and as we can see above, there is no easy solution to alleviate this situation. It is impossible to remove all usability pain points \cite{Bonneau2012ReplacePasswords}. Yet, no alternative can fully replace passwords, either (see Section \ref{sec:rw:authentication_without_pws}). Thus, Herley and Van Oorschot point out that \textit{``supporting passwords better is a vast opportunity for improvement''} \cite{Herley2012PersistenceOfPasswords}, because making even a small change can have an impact on so many users. 

So far, persuasive strategies have produced mixed results regarding their efficacy in supporting users with passwords. This could be due to several issues: Mental models and coping strategies evolve over time \cite{Stobert2014PasswordLifeCycle, Volkamer2013MentalModels}, which has seen little attention in the design of persuasive interventions. Much of the past research dealt with one-shot triggers in isolation, but many context factors and costs for different stakeholders were left out. In an analogy to the design principle ``form follows function'', a better understanding of the functions of user support can help design assistive and persuasive solutions (the form). Moreover, many highly effective nudges from other domains have not been adapted for password support, so we simply may not have discovered the best intervention, yet. However, since nudging strategies are nuanced, we need a structured exploration of how we can bring them to password authentication to remove its most important pain points.
% it is hard to study passwords

\subsection{Research Questions}
In this thesis, I take a holistic approach to address the problem of providing users with the right support in their password practices. To accomplish this, the research presented here tries to answer the following questions:
\begin{itemize}
	\item[\textbf{RQ1}] What is the role of psychological factors and mental models for password selection and coping strategies?
	\item[\textbf{RQ2}] How can password authentication be simplified for users? 
	\item[\textbf{RQ3}] How can we design persuasive strategies to support users in any password-related tasks?
\end{itemize}

\section{Main Contributions}
As highlighted above, several aspects of password support have been left out in the literature, although there are many reasons to consider their importance. First and foremost, a broader understanding of contextual factors that contribute to the formation of specific coping strategies is necessary to improve password support. The primary goal of this thesis is to provide this understanding on a fundamental level by exploring existing factors and addressing them with persuasive designs. The solutions include new paradigms for the design of persuasive interventions.
% see summary, but describe the abstract results / implications rather than rolling them up all over again.
% around one page.
\subsubsection{Insights into Context Factors of Password Practices}
Researchers seem to have reached consensus that the context in which a password goes through its life-cycle \cite{Stobert2014PasswordLifeCycle} is an exploratory variable for users' practices. However, contextual factors have merely been addressed in the discussions of empirical findings. Only a few studies specifically correlated users' backgrounds to their password practices (e.g. \cite{Kessem2018IBMFutureIdentity, Mazurek2013Measuring}) and they mostly focused on demographic factors. We contribute several insights that enrich the understanding of a wider range of context factors. Specifically, we address users' mental models are associated with their password practices. We do this through a novel method to study mental models in-the wild with the aid of a game. Moreover, we are the first to thoroughly investigate the interconnection between personality traits and password authentication. The insights gathered through three online studies revealed interesting associations between personality, attitudes, and behaviors regarding passwords. Lastly, we performed an extensive audit of the real-world constraints that set the context for password reuse. All these insights shape our understanding of the problem space as the foundation for persuasive interventions. 

\subsubsection{Investigation of Persuasive Strategies}
% story: 
With the context factors in mind, we extended the range of persuasive design strategies. As a starting point, we investigated users' explicit, tacit, and latent needs of password feedback. From this exploration, we contribute the ``\textit{show-explain-help-empower}'' paradigm that serves as a heuristic for persuasive password assistance. We followed this up with two studies that were carried out both in the lab and in the field: The first study was the first of its kind to evaluate the Decoy effect for choice architectures in password authentication. Here we learned important lessons about the interplay between feedback and feedforward, and about the role of simplification in persuasion. The second study was focused on empowerment. We evaluated different dimensions of usability of emojis inside text-based passwords. The study delivers timely insights, because an increasing number of web-services enable users to pick such emoji-passwords and there are some issues that need attention from the very start. 

\subsubsection{A Structured Process for the Design of Persuasive Password Support}
Finally, the exploration of the context factors, and the design studies on persuasive assistance in password authentication are melted into a framework for structuring future design processes in this domain. I contribute the \gls{P4P} framework. It respects the dynamism of the status quo, aids in finding the right interventions, and implementing them successfully. To that end, we present a design exercise that demonstrates how the \gls{P4P} framework can be used. 


\section{Thesis Structure}
This dissertation encompasses four major parts that unravel the different aspects of persuasive password support. I chose to structure the content with fourteen self-contained chapters. Although they do follow a narrative, it is possible to read them in any order by following the provided cross-references for the necessary background information. Part \ref{part:related_work} is an exhaustive overview over the related work that serves as the basis for all the discussions in later parts. In Part \ref{part:problem_space}, I report on empirical research that explores the various contextual factors of password selection and coping strategies. Part \ref{part:design_space} then shows how these factors helped to craft novel persuasive design strategies. Lastly, Part \ref{part:synthesis_conclusion} establishes a research and design framework, and concludes with a reflection on the gained insights and future work. In the following, I highlight the contents of the individual chapters with the questions they try to answer.

% can be up to two pages.
%\textbf{Chapter \ref{chap:intro}:} %Foundations and general background on passwords

\subsubsection{Part \ref{part:related_work}: Foundations of Usable Authentication}

\paragraph{Chapter \ref{chap:rw:passwords}: Foundations} %Foundations and general background on passwords
This chapter provides an overview of password-based authentication from a system-perspective. 
Questions answered: \vspace*{-5pt} \begin{itemize}[leftmargin=*,itemsep=-5pt]
	\item How has password authentication evolved over time?
	\item What benefits, drawbacks, and threats do passwords entail?
	\item What is a strong, what is a weak password?
	\item Why do we still need passwords when there are more advanced schemes?
\end{itemize}
%How have we come to this place?
%What are the risks of using passwords?
%What is a strong, what is a weak password?
%Why do we still need passwords when we have all these other authentication schemes?

\paragraph{Chapter \ref{chap:rw:user_perspective}: Human Factors} % User side
I describe the method space to study passwords, before discussing findings about users' password practices. The chapter also highlights the central approaches that have been implemented to mitigate security risks on the user side. 
Questions answered: \vspace*{-5pt} \begin{itemize}[leftmargin=*,itemsep=-5pt]
	\item How do we conduct valid research on passwords with humans and ethics in mind?
	\item How do users cope with passwords? What makes their practices particularly risky?
	\item What can we do to steer people away from risky behavior? 
\end{itemize}

\paragraph{Chapter \ref{chap:rw:summary}: Related Work Summary} This chapter describes the status quo of password authentication and highlights ill-defined aspects that warrant further research.
%Summary of related work and derivation of open questions.
%\begin{itemize}[leftmargin=*,itemsep=-5pt]
%	\item What are the primary issues that have been previously identified?
%	\item What are the ill-defined aspects that warrant further research?
%\end{itemize}


\subsubsection{Part II: Exploring the Context Factors}
\paragraph{Chapter \ref{chap:pasdjo}: Mental Models of Password Strength} %Mental models of password strength (PASDJO)
We present a novel approach to study the perception of password strength: PASDJO, the password game. A longitudinal field study aimed to quantify common misconceptions about the benefits of password complexity, which are an underlying context factor for password practices. 
Questions answered: \vspace*{-5pt} \begin{itemize}[leftmargin=*,itemsep=-5pt]
	\item How well can users gauge password strength? 
	\item Do we have to update our views on users' capabilities?
	\item Is a game suitable to collect the necessary data?
	\item How effective is the game to educate users?
\end{itemize}

\paragraph{Chapter \ref{chap:policies_reuse}: Policies and Reuse} %An audit of password policies in terms of reusability
This chapter reports on a thorough audit of the password policies of the most-visited websites in Germany. It explains  external context factors that shape password reuse in the real world.
Questions answered: \vspace*{-5pt} \begin{itemize}[leftmargin=*,itemsep=-5pt]
	\item How consistent are password policies in the wild?
	\item Is it possible to find a password that meets all requirements at once?
\end{itemize}


\paragraph{Chapter \ref{chap:pws_and_personality}: Personality in Password Practices} % Password personality
This chapter presents three empirical studies about the role of personality traits in password practices. In particular, we shed light on the psychometric context factors for the usability of policies, mental models of password strength, and password selection behavior.
Questions answered: \vspace*{-5pt} \begin{itemize}[leftmargin=*,itemsep=-5pt]
	\item Is personality associated with password practices, attitudes, and behaviors?
	\item How well can we model such associations?
	\item What are the specific implications on the design of personalized password support?
\end{itemize}
%Does personality have any association with password related behavior, attitudes, and mental models?

\paragraph{Chapter \ref{chap:mental_models_pwm}: Mental Models of Password Managers} %Mental Models of password managers (small scale)
We present a qualitative user study eliciting the users' motivations to either adopt or dismiss password managers. A fine-grained mental model is established to depict biases as context factors.
Questions answered: \vspace*{-5pt} \begin{itemize}[leftmargin=*,itemsep=-5pt]
	\item Why are people (not) using password managers?
	\item How do they make sense of their functionality?
\end{itemize}

%What kinds of risks are involved?
\subsubsection{Part III: Persuasive Design Strategies}
\paragraph{Chapter \ref{chap:feedback_modalities}: Feedback Requirements} %Exploring needs in persuasive feedback (enriching understanding of mental models)
This chapter presents two studies on users' explicit and implicit expectations around password feedback. We derive a paradigm for persuasive password support. 
Questions answered: \vspace*{-5pt} \begin{itemize}[leftmargin=*,itemsep=-5pt]
	\item What are users' needs in persuasive feedback?
	\item How would they design a feedback system?
\end{itemize}

\paragraph{Chapter \ref{chap:decoy}: The Decoy Effect} %Decoy effect
We carefully craft a choice-architecture for password support and explore a marketing phenomenon as nudging strategy. The chapter reports on an online study and highlights the interconnection between feedback and feedforward. 
Questions answered: \vspace*{-5pt} \begin{itemize}[leftmargin=*,itemsep=-5pt]
	\item Does the decoy effect work to make stronger passwords more attractive?
	\item How effective is feed-forward in combination with feedback?
\end{itemize}

\paragraph{Chapter \ref{chap:emojipasswords}: Emoji-passwords} %Empowerment through emojis. 
This chapter presents emoji-passwords as an approach to simplify memorization in persuasive ways. We investigate different facets of usability and report on a mixed-methods study. 
Questions answered: \vspace*{-5pt} \begin{itemize}[leftmargin=*,itemsep=-5pt]
	\item How usable are emoji-passwords?
	\item What are the risks and potentials of emoji-passwords?
	\item How do platform-dependent differences affect memorability?
\end{itemize}


\subsubsection{Part IV: Synthesis}
\paragraph{Chapter \ref{chap:perdespassup}: P4P Framework} %P4P
This chapter synthesizes the insights from the first three parts to establish a new framework for the design of persuasive password support. Through a design exercise, I show how it can be applied to develop a novel password manager. 
Questions answered: \vspace*{-5pt} \begin{itemize}[leftmargin=*,itemsep=-5pt]
	\item How can we design persuasive password support in a structured way?
	\item How do we practically apply the framework?
\end{itemize}

\paragraph{Chapter \ref{chap:summary}: Summary} %Summary
In this chapter, I reflect on the presented research and draw conclusions. The contributions are summarized, and contrasted by the limitations of the methodology. I provide eight meta recommendations for future design work. 
Questions answered: \vspace*{-5pt} \begin{itemize}[leftmargin=*,itemsep=-5pt]
	\item What have we learned?
	\item What are the implications and limitations?
	\item What do we need to consider in the future?
\end{itemize}

\paragraph{Chapter \ref{chap:the_end}: The End} %The End
In the final chapter of this thesis I present open research topics and show their potentials. The dissertation concludes with a reflection on the role of password-authentication in the present and the future. 
Questions answered: \vspace*{-5pt} \begin{itemize}[leftmargin=*,itemsep=-5pt]
	\item What research topics have been opened up by this thesis?
	\item What still needs to change to make users' authentication practices easier?
\end{itemize}
%What will the future bring?
%What needs to change to make users' lives easier, to reach them more effectively, and to ease the transition into a password-less world?

\section{Style Choices}

\paragraph{Singular They:} Throughout this dissertation pronouns are used in the plural although speaking about an individual, e.g. ``the user'' is mostly referred to as ``they'' instead of ``he'' or ``she'', to avoid discrimination of certain demographic groups. 
\paragraph{Plurals:} As is common in HCI literature, the author utilizes ``We'' instead of ``I'' to acknowledge the work of collaborators. In later more opinionated parts, the explicit usage of ``I'' intends to communicate the subjective nature of thoughts and interpretations.
\paragraph{Footnotes:} Throughout the thesis, footnotes are excessively used to link to web content. Publications in scientific archives appear in the list of references at the end of the thesis.



% story
% so far: gut feelings and assumptions that there might be more to pw selection than what we can measure
% solution: look for external biases / influences / factors
% contribution 1: mental models of different password practices: strength, pwm usage.
% contribution 2: influence of personality on password practices (we're the first here!)
% contribution 3: context factor: real world that sets arbitrary constraints on password reuse. 
% contribution 4: a new methodology to study password strength perception. 
