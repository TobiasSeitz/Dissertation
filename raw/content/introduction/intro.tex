%!TEX root = ../../diss.tex
\chapter[Introduction]{Introduction}\label{chap:intro}
% INTRO -- 8-10 pages incl. overview. 

% a day in the life of a password user, i.e. everyone. 
% hassle free.
We may not realize it anymore, but most of us need multiple passwords every day. We enter a four-digit PIN to unlock our phone to check incoming messages even before breakfast. Paywalls shield the news of the day, so they force us to log into our favorite news website before we can read them on the way to work. Once arrived, our fingers automatically type the password to unlock the computer. A colleague requests access to a client's platform, so we write down the credentials or share them via a password manager that itself is password protected. When we come home, the kids have used the shared tablet to log into their social media accounts, so we need to re-authenticate if we want to check our own. Interacting with password authentication has become ubiquitous, and we merely carry out the tasks. They are the de facto standard when it comes to access control on the Internet. While it may seem easy to deal with passwords because we are so acquainted with them, there is a wide range of problems they entail.

% all the hassle.
The phone requests the PIN whenever we take it out of our pocket. The news website informs us that our password has expired, so we need to pick a new one that is not part of the last three ones that we used before. The work PC requires not only our credentials but also a one-time password that our phone generates for us every sixty seconds. The colleague leaves the company, so we need to request a new set of credentials from the client to maintain confidentiality. And the kids used the tablet at home to shop online because we had stored the password in the browser for convenience. So, we delete the stored password but fail to recall it a week later, because we had not typed it for a very long time. These and other situations lead to users feeling a considerable burden generated by password authentication. 

\section{The State of the World}
Password authentication encompasses digit-only \glspl{PIN}, graphical schemes (see Section \ref{sec:rw:graphical_pws}), and alphanumeric passwords consisting of letters and digits. We focus on the latter throughout this thesis because they still are the go-to method on the Internet: In 2007, users had around 25 accounts and roughly six distinct passwords on average \cite{Florencio2007LargeScaleStudyPasswordHabits}, while more recent numbers suggest that users keep twelve unique passwords for different purposes \cite{Wash2016UnderstandingPasswordChoices}. With the growing number of accounts the issues around passwords amplify: users pick weak passwords, reuse many of them, forget them, and share them with other people \cite{Stobert2014PasswordLifeCycle}.  

Researchers and practitioners have developed numerous approaches to support users in password authentication. Four central themes have emerged in their efforts: education, enforcement, assistance, and persuasion. 
\subsubsection{Education}
Explaining all risks and defense strategies to users in lengthy prose is doomed to fail, because password security is a secondary goal that is dominated by a primary task, e.g., reading emails \cite{Whitten1999WhyJohnnyCantEncrypt}. Nevertheless, education is a viable strategy for those who actively seek information \cite{Sasse2005UsableSecurityPosition}. 
\subsubsection{Enforcement}
Moreover, enforcing password rules through policies is commonplace. The choice of a specific policy is not trivial, and many organizations, especially in professional domains, make poor tradeoffs \cite{Florencio2010WhereDoPoliciesComeFrom, Shay2016DesigningPasswordPolicies}. The plethora of policies even fosters insecure password practices, when users try to comply in very predictable ways \cite{Inglesant2010TrueCostOfUnusablePolicies, Ur2015PWCreationLab}. 
\subsubsection{Assistance}
To mitigate this, assistive tools like \glspl{PWM} and password generators automate certain interactions to boost both security and usability. They come at the price, e.g., lock-in effects to a particular vendor. Thus, many users still refrain from them \cite{CSID2012PasswordHabits}. 
\subsubsection{Persuasion}
Finally, the youngest strategy to support users in password authentication is based on principles of \textit{persuasive technology}. Fogg defined this paradigm as ``any interactive computing system designed to change people's attitudes and behaviors'' \cite[p. 1]{Fogg2002Persuasive}. While assistive systems certainly often achieve to change behaviors, persuasive technology tries to create sustainable interventions in that users stick to their behavior even when the trigger is absent. Therefore, it often comes as \textit{behavior-change support system} \cite{Oinas-Kukkonen2013BCSS}. In the realm of authentication, password meters are the most representative form of persuasive interventions. Those elements often appear on registration pages of web services, and give feedback on the strength of a user selected password. They try to use \textit{nudges} \cite{Thaler2008Nudge} to convince the user to pick a suitable password. The user stays in control and is free to move on without following the advice, which is perhaps the biggest drawback of persuasive interventions. So far, nudges have often failed at early evaluation stages \cite{Renaud2017LessonsLearnedNudges}, and thus the spectrum of persuasive interventions in the wild is fairly narrow. 

% explain pieces of the title (like ema)
% supporting users - how is this done
% password authentication - what is it, some high level benefits and drawbacks with references to the most important related work
% persuasive design - definition and examplary applications in password authentication. 

\section{Problem Statement and Research Questions}
The challenges and risks of passwords for the users are at the heart of this dissertation. 
% problem statement:
% central problem theme: mental models are not well understood
% why this is important: ``form follows function'' if we don't get the function right, the form is negligible. 
% approach: different aspects: mm of pw strength and personality influence, mm of coping strategies
The notion of the inconsiderate user, who is the ``weakest link'' in the authentication chain and notoriously refuses security measures, has since started to crumble: There is considerable evidence that users want more control over their security and that they are willing to sacrifice usability for it \cite{Kessem2018IBMFutureIdentity}. 
% -- could be due to raised awareness about data breaches etc.

\subsubsection{The Costs of Weak Password Practices}
For many users, the costs of being attacked fortunately remain hypothetical \cite{Herley2015Counterfactuals}. However, there is a growing number of users who have experienced an attack with all its consequences. 
% money loss
First, weak passwords can cause financial damage for end-users, if an attacker manages to impersonate them and withdraws money from bank accounts. At the moment, attacks on digital wallets containing cryptocurrency are highly lucrative, so protecting these assets with strong passwords is vital\footurl{https://blog.dashlane.com/cryptocurrency-exchange-password-power-rankings-2018/}{24.03.2018}. Although social engineering (see Section \ref{sec:rw:attack_vectors}) is a central threat for companies, weak password practices among employees also generate considerable financial losses \footurl{https://www.helpnetsecurity.com/2017/09/19/infosec-weakest-links/}{09.04.2018}. 
% identity theft
Herley \etal noted that \textit{``money is the most obvious loss, but time, frustration and reputation are also at stake''} \cite{Herley2012PersistenceOfPasswords}, beside the emotional distress \cite{Shay2014ReligiousAunt}. Accounts that have a weak password are more likely to be hijacked \cite{Wang2016TargetedGuessingUnderestimated}, and it often takes victims a lot of effort to recover from identity theft\footurl{https://www.businesswire.com/news/home/20151006006149/en/Latest-Data-Breach-Spotlights-Identity-Restoration}{11.01.2018}. 

% threats
% impact on users' lives. 
% opportunities:

\subsubsection{The Costs of Striving for Strong Passwords}
At the other end of the security-usability spectrum, strong passwords can be a large burden for users, too. Typing a long complex password occupies much time and is error prone \cite{Shay2014CanLongPasswordsBeSecureAndUsable}. They are especially tedious to enter on mobiles or devices that were not designed for a lot of text input, e.g., smart TV sets \cite{Melicher2016UsabilityMobileTextPasswords}. Striving for more security, service providers often force users to reset passwords after a given period. Password expiration is responsible for many support desk calls whose resolutions cost the company money \cite{Adams1999UsersEnemy, Sasse2005UsableSecurityPosition}. 

\subsubsection{Improving an Innately Annoying Interaction}
% it is hard to study passwords
% it is impossible to remove all pain points.
% opportunities: even small improvements are big successes because they affect such a large user base
%``supporting passwords better is a vast opportunity for improvement.'' \cite{Herley2012PersistenceOfPasswords}

% PWs are here to stay.

\subsubsection{Research Questions}

 

% pws: an annoying interaction

\begin{itemize}
	\item[\textbf{RQ1}] What is the role of psychological factors and mental models for password selection and coping strategies?
	\item[\textbf{RQ2}] How can password authentication be simplified for users? 
	\item[\textbf{RQ3}] How can we design persuasive strategies to support users in any password-related tasks?
\end{itemize}

\section{Main Contributions}
% see summary, but describe the abstract results / implications rather than rolling them up all over again.
% around one page.
\subsubsection{A Better Understanding of Password Psychology}

\subsubsection{Persuasion Through Simplification}

\subsubsection{A Structured Process for the Design of Persuasive Password Support}



\section{Thesis Structure}
% can be up to two pages.
\paragraph{Part I: Foundations of Usable Authentication}

\textit{Chapter 2:} %Foundations and general background on passwords
How have we come to this place?
What are the risks of using passwords?
What is a strong, what is a weak password?
Why do we still need passwords when we have all these other authentication schemes?

\textit{Chapter 3:} % User side
How do we conduct experiments with humans and ethics in mind?
How do users cope with passwords? What makes their behavior risky?
What can we do to steer people away from risky behavior? 

\textit{Chapter 4:} 
%Summary of related work and derivation of open questions.
What are the main issues that have been previously identified?
What are the things that are not well understood and need more work?


\paragraph{Part II: Evaluating the Problem}
\textit{Chapter 5:} %Mental models of password strength (PASDJO)
How well can users gauge password strength?
Can a game tell us this?
How much data is necessary?
How well do people learn from playing the game?

\textit{Chapter 6:} %An audit of password policies in terms of reusability
What kind of passwords can be reused across the board, and which can't?

\textit{Chapter 7:} % Password personality
Does personality have any association with password related behavior, attitudes, and mental models?

\textit{Chapter 8:} %Mental Models of password managers (small scale)
Why are people (not) using password managers?
How do they make sense of their functionality?
%What kinds of risks are involved?


\paragraph{Part III: Persuasive Design Strategies}
\textit{Chapter 9:} %Exploring needs in persuasive feedback (enriching understanding of mental models)
What are users' mental models of current persuasive feedback?
What are users' expectations and needs for future persuasive feedback?

\textit{Chapter 10:} %Decoy effect
Does the decoy effect work to make stronger passwords more attractive?
How effective is feed-forward in combination with feedback?

\textit{Chapter 11:} %Empowerment through emojis. 
How usable are emoji-passwords?
What are the risks and potentials of emoji passwords?

\paragraph{Part IV: Synthesis}
\textit{Chapter 12:} %P4P
How should we design future persuasive password support?
Can we get some structure into the process? 

\textit{Chapter 13:} %Summary
What have we learned? 
What are the caveats?

\textit{Chapter 14:} %The End
What will the future bring?
What needs to change to make users' lives easier, to reach them more effectively, and to ease the transition into a password-less world?

\section{Style Choices}
\begin{itemize}
\item \textbf{Singular They}\footurl{https://en.wikipedia.org/wiki/Singular_they}{06.01.2018} Throughout this dissertation pronouns are used in the plural even when speaking about an individual, e.g. ``the user'' is mostly referred to as ``they'' instead of ``he'' or ``she'', to avoid discrimination of certain demographic groups. 
\item As is common in HCI literature, the author also utilizes ``We'' instead of ``I'' to acknowledge the work of any collaborators. In later more opinionated parts, the explicit usage of ``I'' intends to communicate the subjective nature of thoughts and interpretations.
\item Throughout the thesis, footnotes are excessively used to link to web content. Publications in scientific archives appear in the list of references at the end of the thesis.
\end{itemize}
