\section{Participatory Design of Password Feedback}
% continue rapid approach from first study 
GOAL: design session to actually teach us more about the requirements, not necessarily about the solutions. The prototypical and conjoint solutions tell us about the expectations and needs of users regarding password feedback. 

Short intro about participatory design \cite{Sanders2002ParticipatoryDesign}
%- Involve users in the design of a novel feedback solution
%- participatory design session
%- different groups

\subsection{Method}

Brainstorming Session 

Ball-bearning

 - group work
 - voting for favorites to inform what we should explore further. 
10+10

Design sessions then establish feedback loop to create shared ownershipm,               5

\subsubsection{Participants}
screened for heterogeneous backgrounds

\subsection{Concepts}
- Designs:
- rewards: beautify page / positive reinforcement from friends (weird suggestions but okay)
- analogies: time to crack --> goal: better risk assessment for non-experts.
- playfulness: bubbles / vault / slotmachine to make random character replacements more exciting / represent strength contribution of different elements in some way (fruit salad)


\subsection{Learnings}

- interesting aspects:
	- a lot of the concepts are visualizable with little text.
	- missing: background information.
	

\subsection{Process Reflection}
What did people say about being involved? What are the opportunities and drawbacks that they saw?