\section{Participatory Design of Password Feedback}
% continue rapid approach from first study 
GOAL: design session to actually teach us more about the requirements, not necessarily about the solutions. The prototypical and conjoint solutions tell us about the expectations and needs of users regarding password feedback. 

%todo maybe pick up the insights from the previous survey

Short intro about participatory design \cite{Sanders2002ParticipatoryDesign}
%- Involve users in the design of a novel feedback solution
%- participatory design session
%- different groups

\subsection{Method}
We used participatory design approach to learn new requirements for persuasive feedback. The methodology included a co-design session with various exercises, an exploration and refinement of the initial ideas, a feedback loop on the progress of the concepts, and the creation of a final interactive digital prototype.

\subsubsection{Procedure of the Co-Design Session}
%Brainstorming Session, 60 minutes, different exercises, keep tempo high.
% 1
The 60-minute co-design session started with a briefing about the goals and informed consent to record the session on video. Afterwards, we sensitized participants for the topic with two \textit{ball-bearing exercises}. For this, participants were divided into two groups and those faced each other in a circle. Each person interviewed another on a specific question of password behavior for one minute, until the outer circle moved to the next interviewer until they were talked to all people from the inner circle. In the first round, interviewers were given questions on personal password selection and coping strategies. The second round focused on recommendations to different user groups. The interviewers then summarized their insights and shared them with the group. The exercise was helpful to learn other people's password strategies and broaden the participants' horizon. In summary, participants showed typical selection strategies based on word-digit-symbol patterns and would recommend personal, memorable events as starting points for password selection.

What followed was a straightforward brainstorming exercises in two groups. The question they tried to answer was ``How would a registration form of an email provider need to be designed to help you create a secure password?'' After five minutes, half of each group moved to the other group. In total, participants produced 17 distinct ideas and presented them on a whiteboard. Everyone received two votes for their favorite ideas. Participants then formed groups of two or three to create paper prototypes for the three ideas that had the most votes. The facilitator made sure to explain the process and the expected fidelity of the outcome. Once the prototypes were ready, the groups presented them to the entire crowd.

\paragraph{Participants}
We recruited seven participants through postings in public groups on social networks. Six of them were female. Three were studying philology, while one each were studying pedagogy, media informatics and business studies. The only male participant was an employed product designer. The average age was 23, ranging from 19 to 29. As an incentive, we offered a 10€ shopping voucher that was handed out after the brainstorming session. We had tried to recruit a more diverse sample, but we did not receive other such responses. 

\subsubsection{10 plus 10 Method and iterative improvement}
Based on the participants' paper prototypes, we designed ten variations of each idea, i.e. 30 concept candidates. This procedure was inspired by the \textit{10 plus 10 method} to explore the design space. %todo maybe add reference.
Afterwards, we discussed the feasibility of each solution and identified six candidates that were presented to the study participants. Their feedback informed the decision on the final prototype. After it was implemented with standard web-technologies, participants provided a final round of qualitative feedback and a reflection on the process. Participation thus occurred in each step of the design process. 

\subsection{Concepts and Prototype}
From the 30 designs, six made it to the final stage of the selection process. 
- Designs:
- rewards: beautify page / positive reinforcement from friends (weird suggestions but okay)
- analogies and requirements feedback: time to crack --> goal: better risk assessment for non-experts. / represent strength contribution of different elements in some way (fruit salad) / vault 
- playfulness: bubbles / vault / slotmachine to make random character replacements more exciting / 

\subsubsection{``Bubbles'' Prototype}

\subsection{Learnings}

- interesting aspects:
	- a lot of the concepts are visualizable with little text. (emphasize: ``show'' and ``help'')
	- missing: background information. (thus ``explain'' was not visible anymore)
	- empower: not really visible
	
% fruit salad 

\subsection{Process Reflection}
What did people say about being involved? What are the opportunities and drawbacks that they saw?



%kicked out content / structure
% 2
%Ball-bearning exercise
%- add picture of concept: two circles, people on the inner circle interview those on the outer circle who move along after 2 minutes. inner circle gets 1-2 minutes to summarize their findings. 
%- purpose: 
%- first round: own behavior and practices
%- what do (other) people know about password strength and feedback, 
%- what do they do to create passwords
%- share experiences
%- take away: mangling words
%- second round: if they were asked by other people (grandmother, relatives, friends/peers)
%- take away: use context or a good memory as a starting point for your password, e.g. the location and year of a treasured vacation
%- explore potential problems of users, share stories
%- facilitator comments on findings.

%\% 3 
%Idea generation
%- task: ``how would a registration form on a webpage have to look, to help you create a secure password?'' context: email provider
%- quantity!
%- 2x5 minutes in different groups
%- put central topics on whiteboard (17 ideas from both rounds)
%
%\% 4 converge
%- everyone votes for their favorite (2 votes) --> here 3 ideas with 4 votes each were the winners
%- groups of 2 or 3 to build a paper prototype (after a brief explanation what that is)
%
%\% 5 present
%- brief presentation of prototype
%- voting for favorites to inform what we should explore further. 