\chapter[Password Selection and the Decoy Effect]{Password Selection and the Decoy Effect}\label{chap:decoy}


%TODO Result summary. 
\section{Background and Context}

Goal: influence / correct mental models of password strength, persuade users to consider alternative strategies. 

Criticism: this is unrealistic because memory interference effects prevent that this works for more than one account. -- yes, but for important accounts, this might still be worthwhile and people can more easily write down three words instead of an 8 character random sequence that might include ambiguous characters (0 and O); Kuo et al also say that writing down is a perfectly valid strategy \cite{Kuo2006HumanSelectionMnemonic}. (\todo{look up more researchers who argue in favor of writign down, e.g \cite{Kothari2017PasswordLogbooks} })

\subsection{Research Questions}

\section{Related Work}

\section{Designing Decoy Options}

Talk about how we did the iterations.

\section{Evaluating the Design}
\subsection{Methods}

no CS students: just like in \cite{Wash2016UnderstandingPasswordChoices}

\subsection{Survey Results}
Mostly focus on the contents of the paper here

\subsection{Live-Deployment Data}
Talk about a couple of findings of the Roskilde dataset. 

\subsection{Limitations}


\section{Discussion}

\subsection{The Ineffectiveness of the Decoy Effect}


\subsection{Creativity Support}
As we have seen in chapter \ref{chap:pasdjo}, users know fairly well what makes a strong password. Thus, we can explain the stronger influence of the passphrase in several ways. There might be a gap between password strength perception and selection that is due to \textit{creativity}. It might be easier for people to decide upon a stronger password if they actually see it. This, however, will not work, if the password is absolutely unrelated to the person because this increases memorization efforts. 

Creativity is usually associated with the ``Openness'' personality trait (see chapter \ref{chap:pws_and_personality}). So we can suspect that people who show less of this trait might be more susceptible to a password suggestion. 

mental model influenced? -- look at qualitative data. 

\section{Take-Aways}



