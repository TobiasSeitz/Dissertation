\chapter[Extending the Password Space with Emoji]{Extending the Password Space with Emoji}\label{chap:emojipasswords}

%Lingo: Visual memory vs lexical memory

Emojis as an ``enabling technology''

in chapter \ref{chap:pws_and_personality} we found that certain users are more open to a passwords that contain emojis. now it is time to explore their feasibility. 

%TODO Result summary. 
\section{Background and Context}

Regarding character/digit positioning: Weir \etal have some data \cite{Weir2010MetricsPolicies} and of course Ur \etal \cite{Ur2015PWCreationLab}



\cite{Kuo2006HumanSelectionMnemonic} has an example of a participant using an emoticon in their password (``<3'') to replace a word. 

Challenges:
Emoji Fragmentation -- Emoji Convergence \footurl{https://blog.emojipedia.org/2018-the-year-of-emoji-convergence/}{16.02.2018}

\subsection{Research Questions}

\begin{itemize}
	\item[RQ1] Do emojis in passwords help users create more \textbf{memorable} passwords?
	\item[RQ2] What is the most feasible \textbf{user interface} for emoji passwords?
	\item[RQ3] How do platform-dependent renderings (\textbf{fragmentation}) affect memorability?
	\item[RQ4] What are users' general \textbf{attitudes} towards using emojis inside passwords?
\end{itemize}

\section{Methodology}


\section{Results}


\subsection{Limitations}


\section{Discussion}


\section{Take-Aways}
