%\subsubsection{Empirical Studies using Persuasive Interventions}
	
	% \subsubsection{Cognition / Decision oriented}
%	\textbf{Set completion}
%	\textbf{Goal gradient effect}
%	\textbf{Defaults} \cite{Chiasson2008PCCP}
	
	% \subsubsection{Playfulness}
	%\cite{Azevedo2012AuthenticationGame,Kroeze2012GamifyingAuthentication}
	%Phishing \cite{Arachchilage2013GameDesingPhishing}
	
	%\subsubsection{Social Influence}
	% papers that use social nudges
	%Egelman \etal evaluated this strategy for password feedback \cite{Egelman2013DoesMyPasswordGoUpToEleven}. A visualization informed participants in their study who well their password fared compared to other users, e.g. ``your password is stronger than that of 85\% of our users''. They did not find evidence that this persuasive strategy influenced people, but maybe the approach could have been more focused on the \textit{proof} aspect, rather than \textit{competition}. The normative message is ``other people's passwords are bad, but many people act this way''. This could actually evoke backfire effects through social proof: although users see their password is stronger, they believe that the social norm is to pick weaker passwords, which makes them conform to the social norm. In fact, Weirich and Sasse have provided empirical evidence for such behavior \cite{Weirich2001PrettyGoodPersuasion}.
	
	%\cite{DiGioia2005SocialNavigationUsableSecurity}
	%\cite{Ashenden2013SecurityLikeSoap}
	
	%\cite{Goldstein2008RoomWithAViewpoint}
	
	%PW authentication = socio-technical mechanics \cite{Sasse2005UsableSecurityPosition}, maybe also the behavior of other users affects on
	
	%``politeness as central element \cite{Bahr2013RationalInterfaces}
	%misusage of social influence. \cite{Muscanell2014WeaponsMisused}
	
	%\subsubsection{Heuristics}
	%\cite{Schneier2008PsychologySecurity}:
	%Probability heuristics \textbf{Availability heuristic}
	%risk heuristics 
	%Cost heuristics (mental accounting, time discounting)
	%Decision-centric heuristics
	
	%\subsubsection{Privacy Nudges}
	%most prominently: Acquisiti \etal \cite{Acquisti2017NudgesPrivacySecurity} \cite{Acquisti2005PrivacyRationality}
	%privacy is an economic problem \cite{Woodruff2014PrivacyFundamentalist}
	%\cite{Yevseyeva2014RiskyNudging}
	%\cite{Balebako2011PrivacyNudgesMobile}				
	%nudging privacy decisions is more prominent in the literature than nudging password decisions. 
	%\cite{Adjerid2016BeyondPrivacyParadox}
	%\cite{Wang2014PrivacyNudgesFacebook}
	%\cite{Almuhimedi2015PrivacyNudges}
	%\cite{Egelman2013ChoiceArchitecture}
	%\cite{Choe2013NudgingAwayVisualFraming}