%\subsubsection{Empirical Studies using Persuasive Interventions}
	
	% \subsubsection{Cognition / Decision oriented}
%	\textbf{Set completion}
%	\textbf{Goal gradient effect}
%	\textbf{Defaults} \cite{Chiasson2008PCCP}
	
	% \subsubsection{Playfulness}
	%\cite{Azevedo2012AuthenticationGame,Kroeze2012GamifyingAuthentication}
	%Phishing \cite{Arachchilage2013GameDesingPhishing}
	
	%\subsubsection{Social Influence}
	% papers that use social nudges
	%Egelman \etal evaluated this strategy for password feedback \cite{Egelman2013DoesMyPasswordGoUpToEleven}. A visualization informed participants in their study who well their password fared compared to other users, e.g. ``your password is stronger than that of 85\% of our users''. They did not find evidence that this persuasive strategy influenced people, but maybe the approach could have been more focused on the \textit{proof} aspect, rather than \textit{competition}. The normative message is ``other people's passwords are bad, but many people act this way''. This could actually evoke backfire effects through social proof: although users see their password is stronger, they believe that the social norm is to pick weaker passwords, which makes them conform to the social norm. In fact, Weirich and Sasse have provided empirical evidence for such behavior \cite{Weirich2001PrettyGoodPersuasion}.
	
	%\cite{DiGioia2005SocialNavigationUsableSecurity}
	%\cite{Ashenden2013SecurityLikeSoap}
	
	%\cite{Goldstein2008RoomWithAViewpoint}
	
	%PW authentication = socio-technical mechanics \cite{Sasse2005UsableSecurityPosition}, maybe also the behavior of other users affects on
	
	%``politeness as central element \cite{Bahr2013RationalInterfaces}
	%misusage of social influence. \cite{Muscanell2014WeaponsMisused}
	
	%\subsubsection{Heuristics}
	%\cite{Schneier2008PsychologySecurity}:
	%Probability heuristics \textbf{Availability heuristic}
	%risk heuristics 
	%Cost heuristics (mental accounting, time discounting)
	%Decision-centric heuristics
	
	%\subsubsection{Privacy Nudges}
	%most prominently: Acquisiti \etal \cite{Acquisti2017NudgesPrivacySecurity} \cite{Acquisti2005PrivacyRationality}
	%privacy is an economic problem \cite{Woodruff2014PrivacyFundamentalist}
	%\cite{Yevseyeva2014RiskyNudging}
	%\cite{Balebako2011PrivacyNudgesMobile}				
	%nudging privacy decisions is more prominent in the literature than nudging password decisions. 
	%\cite{Adjerid2016BeyondPrivacyParadox}
	%\cite{Wang2014PrivacyNudgesFacebook}
	%\cite{Almuhimedi2015PrivacyNudges}
	%\cite{Egelman2013ChoiceArchitecture}
	%\cite{Choe2013NudgingAwayVisualFraming}
	
	As a unique feature compared to other persuasive frameworks, the \gls{PAF} introduced the personality dimension for effective persuasion. . In two studies they looked at the predicting power of different psychometric scales, among which we find the five-factor model (``Big Five''), the General Decision Making Style (GDMS) or the Domain-Specific Risk-Taking scale (DoSpeRT) \todo{attach sources}. They conducted two online experiments using Mechanical Turk. In the first round they used the Ten-Item-Personality Index and correlated the scores with several privacy metrics (e.g. the Privacy Concerns Scale or the Internet Users Information Privacy Concerns scale). They realized that the Big Five model was a rather weak predictor for the different scales, which lead them up to their second experiment, where they focused on decision-making metrics. Here, they observed stronger correlations, between e.g. the rational trait of the GDMS and both PCS and IUIPC. Following these observations, Egelman and Peer interpret them as evidence that privacy attitudes originate from rational decision-making, and also from people's intuitions, which sounds paradoxical at first, but it were different respondents who produced this result. They conclude that the decision-making metrics were about three times as powerful regarding their predictive power than the big five model regarding privacy attitudes and behavior. Finally, they propose the Security Behavior Intentions Scale (SeBIS). This scale intends to isolate confounding factors when correlating personalty traits with security behavior. Also, the authors offer a number of hypotheses and unanswered research questions that highlight how little exploiting personality traits in security nudges has been investigated. 
	
	
		%\subsection{Suggestion}
	%\cite{Forget2008ImprovingPasswordsThroughPersuasion}
	%\cite{Forget2008MemorabilityPersuasivePasswords}
	%\cite{Forget2008PersuasionStrongerPasswords}
	
	% \subsection{Personalization and Adaptation}
	%LastPass report \cite{LastPass2016PersonalitiesGetUsHacked}
	
	%\subsubsection{Personality Traits}
	%\paragraph{Inventories, Constructs, and Proxies}
	%Big Five à la Costa and McCrea \cite{Costa1992NEO}
	%SeBIS (Egelman)
	%Haque construct \cite{Haque2014PsychometricsStrongPassword}
	%\paragraph{Personality Studies in Usable Security and Privacy}
	
	
	%\subsubsection{Personalized Secure User Interfaces}
	%\cite{Kaptein2015PersonalizingPersuasiveTechnologies}