\chapter[Passwords -- A User Perspective]{Passwords -- \\
	A User Perspective}\label{chap:rw:user_perspective}
%lingo: arduous

``but his thoughts were so full of the great riches he should possess, that he could not think of the word to make it open, but instead of `Sesame,' said, `Open, Barley!' and was much amazed to find that the door remained fast shut. He named several sorts of grain, but still the door would not open, and the more he endeavoured to remember the word `Simsim,' the more his memory was confounded, and he had as much forgotten it as if he had never heard it mentioned.'' (Kasim's predicament in \textit{Ali Baba and the Forty Thieves}) \todo{this could be the opening of the chapter / fancychapter}

Morris and Thompson were already concerned with user behavior regarding passwords in 1979 \cite{Morris1979PasswordSecurity}. They identified that users choose predictable passwords and that this can be leveraged for attacks. So, they suggested enforcing a certain minimum password length (six characters). At the time, the users were mostly professionals that received training to operate computers and could thus also have been trained to pick less predictable passwords \cite{Maguire2012YouOnlyLiveTwice}. But as computers were introduced to a larger audience, more people were exposed to password authentication. Naturally, this also induced a growing number of attacks, and it is increasingly difficult for users to defend themselves against them (see. Section \ref{sec:rw:attack_vectors}). Nowadays, password policies are in place that require not only a minimum of eight characters, but also mandate mixed-case letters, digits and special symbols to start with. The HCI community noticed the users' struggle in the 1990s and that we can -- and should -- design authentication systems with usability in mind. Perhaps, one of the breaking points where a new school of thought turned up in the literature was a paper by Adams and Sasse in 1999 \cite{Adams1999UsersEnemy}. The central and novel theme in there was a shift from \textit{fixing} the user to \textit{acknowledging} user behavior and designing for it. The paper managed to see over 1500 citations as of writing this.

This chapter looks at the literature that mostly came after this seminal work. It discusses the users' problems, solutions, feelings, and opinions about using passwords. An essential goal is to give the reader an empathetic perspective and provide background information to understand why it seems hard to come up with viable solutions to make users' lives less frustrating. To get there, we first take a brief look at conducting user research with passwords. Hereafter we disseminate typical coping strategies and solutions. The chapter concludes with a comment on the discourse that has been going on between the very different schools of thought about passwords. 

%Each person who gets in contact with the Internet will at some point create a password.

% Everyone develops their own strategy how to do this and how to cope with passwords, probably already in early teenage years \cite{VonZezschwitz2013SurvivalShortest}. These strategies however are not unique and show macroscopic commonalities, which became evident after the first large-scale password leaks. 

%Important: Under coping strategy, we also understand the selection process, because choosing a weak password over a strong one is also one way to \textit{cope} with the large number of passwords and the memorability burden. (make sure to mention this in the general introduction already.)



%@@TODO cite Wash paper @SOUPS 2016.


\section{Methodology: Running Password Studies}
% not relevant: how to measure password strength. solely focus on the ``HOW'' of data collection
Before we report on insights about user behavior regarding passwords, we take a brief look into running studies that focus on passwords. There are two central aspects that make collecting data particularly challenging: acting ethically and maintaining high ecological validity of the data. In fact, these two goals create an area of tension that demands a critical selection of methods. Komanduri \etal note that ``ideally, password studies would be conducted by collecting data on real passwords created by real users of a deployed system.'' \cite{Komanduri2011OfPasswordsAndPeople} But this would mean that researchers then have access to the user accounts that were under investigation. This is ethically questionable. Maybe the researchers themselves are benevolent, but the data is precious and thus could bring attackers to the scene. Since absolute security can barely be guaranteed, it is best to avoid that users disclose their actual credentials in a user study to the researchers. 

% example for solution of ``proxy'' password
If we cannot collect the participants' passwords, what is the best way to measure, e.g., cause and effect of interventions? Von Zezschwitz \etal used a ``proxy password'' that described the participants' actual passwords \cite{VonZezschwitz2013SurvivalShortest}, but which was insufficient to reconstruct the original. Participants were provided with an offline password analysis tool that they had to operate manually. The analysis served as a proxy and was then shared with the researchers by pasting it to the questionnaire form. 

% ethical dilemma
Another ethical conundrum is sharing research about password attacks. For instance, one can put forward new cracking approaches (e.g. \cite{Marechal2008AdvancesPWCracking, Narayanan2005FastDictionaryAttacks, Weir2009PCFG}) that potentially affect common strength metrics (see Section \ref{sec:rw:pw_strength_metrics}) -- but attackers also benefit from this kind of knowledge. From an HCI perspective, one can also unfold how users select passwords, which allows optimizing cracking efficiency \cite{Weir2010MetricsPolicies, Wheeler2016zxcvbn}. 

%TODO maybe extend this intro and talk about harbach's and Krol's findings
%Krol2016ExperimentDesign


 many people behave normally, and if you ask them in the survey if they did, this can improve the quality of the data drastically \cite{Fahl2013EcologicalValidityPasswordStudy}

issues:
\begin{itemize}
	\item finding efficient ways to attack passwords - this might also help attackers. 
	\item sometimes researchers ``phish'' participants to obtain their passwords (\cite{Egelman2013DoesMyPasswordGoUpToEleven, Haque2014Hierarchy, Mazurek2013Measuring}) -- to conceal the study purpose and get ecologically valid passwords. 
\end{itemize}

do people know if their passwords are studied?


Not seldom, a triangulated approach is the most feasible to reach the actual ``truth'' about user behavior. In the following, the most prominent study methods are presented. 


%\begin{itemize}
%	\item[Ethics] Leaked Data, collecting real passwords from study participants
%	\item[Ecological Validity] role playing, passwords are never used in real life, 
%\end{itemize}
%
%
%
%general overview a la. how to run a good password study 

\cite{Harbach2016HardLockLife,Henze2013EmpiricalResearchUbiquitous,Kuhn1993ParticipatoryDesign,VonZezschwitz2013SurvivalShortest,Hassenzahl2003AttrakDiffGerman,Mazurek2013Measuring,Egelman2015SeBIS,Rosoff2014BehavioralExperimentsFraud,Savage2012GainingWisdomCrowds}


% briefly show the benefits and drawbacks of each method. 
%TODO maybe some Usable Security book already has this? look into garfinkel/cranor book...
\subsection{Analyzing Password Leaks and (Semi-)Public Data}
closest to actual password behavior, with one large disadvantage: 
\cite{Bonneau2012LinguisticProperties}
\cite{Bonneau2012ScienceOfGuessing}
\cite{Veras2012VisualizingSemanticsPasswords}
\cite{Wash2016UnderstandingPasswordChoices}


\subsection{Laboratory studies}
% not focused on password: \cite{Sotirakopoulos2011ReplicationSSLWarnings}
mostly used to collect qualitative data.
semi/structured interviews: \cite{Stobert2014PasswordLifeCycle}


makes memorability studies a bit more cumbersome but are not impossible. 

Adams Sasse 1997 \cite{Adams1997MakingPWsSecureAndUsable}
Weirich on Persuasion \cite{Weirich2001PrettyGoodPersuasion, Weirich2005PersuasivePasswordSecurity}

café study also here because the experimenter is present and you have almost as much level of control as in the lab (except for the distracting surroundings but for passwords that might help sometimes)
\cite{VonZezschwitz2013SurvivalShortest}

\subsection{Field-, diary- and in-the-wild studies}
%TODO maybe add subsubsections or paragraphs.. because this approach encompasses a lot of study types.

online surveys 
- with an interactive prototype 
- or just with a survey tool

diary studies here.  \cite{Hayashi2011DiaryStudyPWs}

Experience Sampling \cite{Consolvo2003ESM}

In the wild: \cite{Chamberlain2012ResearchInTheWild}; \cite{Renaud2017LessonsLearnedNudges}

people are not always aware that their data is logged and behave normally (high eco val)

Example for in-the-wild data: 
Flor\^{e}ncio and Herley probably conducted the largest study to date on password habits. Their intention was to find out among other things A) how often people type passwords, B) how many sites share a password C) how many distinct passwords a user has, and D) how the strong the passwords are. They utilized the Windows Live Toolbar for Internet Explorer to collect in-the-wild data from up to 500000 users during three months of running the collection

\cite{Florencio2007LargeScaleStudyPasswordHabits}. The established protected password lists (PPL) to avoid intruding into people's privacy. They found that users had about 7 distinct passwords in 2007, and that passwords are re-used at about 6 sites in average. Interestingly, they found that stronger passwords are not re-used as often as weak passwords (only around 4 sites). It was not possible to trace the incoming data back to a specific user, which might have resulted in over counting of entries. Also, it was not measured how long the actual password entries takes. If users only used regular dictionary words without any modification, the key logging module of the toolbar would record a password reuse event (PRE) every time the user entered the word -- also in regular text searches, for example. Another limitation could be that they used entropy as a proxy for password strength. However, as discussed in the previous chapter, we have seen that this metric is more robust for system-generated passwords and that strength estimation has evolved over the past ten years. 

\subsection{Crowd-sourcing and Mechanical Turk Studies}
Special kind of ``field'' study which is why they are reported separately.

problem: in europe it's not immediately possible, but there are alternatives. 

also, adding more complex, interactive prototypes is challenging

\cite{Peer2017BeyondTheTurk}
\cite{Ross2010WhoAreTurkers}

MTurk
propagated and most commonly used at CMU e.g. \cite{Mazurek2013Measuring} \cite{Shay2014CanLongPasswordsBeSecureAndUsable} \cite{Shay2016DesigningPasswordPolicies}
\cite{Shay2015UsablePoliciesMTurk}
\cite{Ur2016PerceptionsPassword} \cite{Melicher2016UsabilityMobileTextPasswords} \cite{Ur2017DataDrivenPWMeter}

most commonly, the study topic is disclosed so users know that their passwords are analyzed. 

\cite{Huha2015UserReplaceablePasswords}

problem:
Password guessability depends on context factors and MTurk passwords - albeit slightly weaker - are a good approximate of real passwords \cite{Mazurek2013Measuring}


\subsection{Triangulation}
(survey, log analysis) \cite{Wash2016UnderstandingPasswordChoices}

\subsection{Theory and Reasoning}
not only empirical methods but also logic reasoning, argumentation is strong among usec papers


\subsection{The Bottom Line: Emerging good practices and tools}

UX: AttrakDiff is often 

for surveys, IMCs are recommendable \cite{Oppenheimer2009InstructionalManipulationChecks}
memory clearing tasks are often used in memorability studies.
PGS 


Qualitative studies:

Quantitative: 
Yan 2004 Memorability \cite{Yan2004PasswordMemorabilitySecurity}

Position Papers:
Ives Domino Effect \cite{Ives2004DominoEffectReuse}

In the workplace: \cite{Adams1997MakingPWsSecureAndUsable, Inglesant2010TrueCostOfUnusablePolicies}


\todo{Add a table with advantages and disadvantages of different study methods.}

			
	
	There's a Scale that we can use to cheaply measure security intentions so we can better weight study data from individuals \cite{Egelman2015SeBIS}
	SeBIS is a useful tool and looks robust, but it might need more time to really be considered highly valid \cite{Egelman2016BehaviorEverFollows}

%\subsection{Best Practices on Ethical Standards}
	look into ``Critical'' folder on mendeley
	


	
\section{User Behavior Regarding Passwords}\label{sec:rw:how-users-cope}
Passwords are the cornerstone of \textit{knowledge-based} authentication. And although ``knowledge'' can be stored in and retrieved from computers, it is still a human capability to learn things to later ``know'' them. So, humans are a large factor in the equation and their actions and behavior to gain knowledge with and about passwords deserve to be studied in detail. 

Some cybersecurity researchers started blaming system failures and vulnerabilities on users. For instance, Feldmeier \etal stated: ``The main weakness in any password system is that users often choose easily guessable passwords: English words, names, trivial extensions to English words, etc., because they are easy to remember'' \cite{Feldmeier1990UnixPasswordSecurity}. It quickly became a dictum that users were the ``weakest link'' in the figurative authentication chain and investing into the technical infrastructure might not be enough \cite{Sasse2001WeakestLink}. 


Industry reports: \cite{CSID2012PasswordHabits}


Although passwords were designed to protect ``systems'', the HCI community

\todo{add a table that has all problems on one side and the possible user coping strategies on the other side.}

Take human factors into account when you design secure systems \cite{Sasse2005UsableSecurityPosition}


\cite{Adams1999UsersEnemy} is considered the mother of all HCI \& USEC papers. -- but there were many others before that.



``password overload'' and ``memory interference'' as technical terms must appear here (for discussion see \cite{Yang2016MnemonicSentenceBased}).

in the same vein: \cite{Chiasson2009InterferencesGraphical}


password mechanisms and their users are a socio-technical system and the social aspect weighs heavy \cite{Weirich2001PrettyGoodPersuasion}

Security is too hard for end-users and we need to make it more usable. (fair enough it was the early days of USEC) \cite{Dourish2004UserStrategiesEveryday}

insights into the password burden and context information where people log in, categorization of accounts \cite{Hayashi2011DiaryStudyPWs}


a formal model of the Password Life Cycle, codebook for coping strategies \cite{Stobert2014PasswordLifeCycle}.

	\subsection{Selecting Weak Passwords}
	% things to answer in this section:
	% what's so difficult about selecting a strong password?
	% what do users do if they want to select a strong password?
	% why don't they always do it?
	% when do they select stronger passwords?

	pass\textit{word} implies it has to be a word. Other names for the concept, but basically the same meaning: security code, passcode, secret, credentials, access token
	
	\cite{Jakobsson2013BenefitsUnderstandingPWs}
	
	People show predictable modification behavior \cite{Gaw2005ReuseRecycle}
	
	
	a lot of RockYou's passwords are based on dates and this can be visualized and used for attacks \cite{Veras2012VisualizingSemanticsPasswords}
	
	Greek users do not behave differently than the rest, but the top 100 passwords are a bit different \cite{Violettas2014PasswordsAvoidGreece}
	
	
	\cite{Li2017PersonalInformation}

	\subsubsection{Why do Users Select Weak Passwords?}
	
	Users act insecurely because they choose weak passwords that they reuse \cite{Riley2006WhatUsersKnowWhatTheyDo}
	
	people don't take password policies at organizations seriously.  \cite{Weirich2005PersuasivePasswordSecurity}
	
	- policies allow it (\cite{Seitz2017PoliciesReuse})
	- wrong mental model or misinterpretation of security advice (\cite{Ur2015PWCreationLab, Ur2016PerceptionsPassword, Seitz2017PASDJO})
	- because they don't care
	- They looked at depletion levels and password strength, but the method was kind of flawed so it's difficult to take this seriously \cite{Gross2016EffectCognitiveEffort}
	- ... they underestimate the threat
	- ... they are right to judge the account as low value (who would hack me?) \cite{LastPass2016PersonalitiesGetUsHacked}
	- Passwords survive for a long time, sometimes in a modified version, and they tend to get slightly more complex over time and for important accounts\cite{VonZezschwitz2013SurvivalShortest}
	
	Passphrases are super predictable because most people use common phrases \cite{Bonneau2012LinguisticProperties}
	
	\cite{Wang2015ChinesePWs}
	
	thesis on the ``why'': \cite{Notoatmodjo2007ExploringWeakestLink}
	
	qualitative studies \cite{Ur2015PWCreationLab, Stobert2014PasswordLifeCycle} 
	quantitative studies \cite{Ur2016PerceptionsPassword, Seitz2017PASDJO}
	
	
	Text entry under lab conditions for multiple password selection tasks in a row doesn't have a large effect on typical password metrics \cite{Yang2014EntryAffectsPasswordSecurity}
	
	input modality can be a strong influence on password selection in that mobile devices will lead to less diverse passwords \cite{VonZezschwitz2014HoneyIShrunkTheKeys}
	
	Users understand quite a bit about password security, but we should help them how attacks work \cite{Ur2016PerceptionsPassword}
	
	
	Users are not too bad at creating strong passwords, but often they misinterpret security advice which leads to weak password practices \cite{Ur2015PWCreationLab}
	
	Even after a credential leak people are not very invested with their weak passwords: LinkedIn's password reset email can be considered ineffective, but the more severe problem is that the email focused on LinkedIn only, because it's super important to change the password on all other sites \cite{Huh2017TooBusy}
	
	\subsubsection{What's the Problem with Weak Passwords?}
	usually, people tend to use stronger PWs for important accounts
	
	Stronger passwords are not always necessary, do not force users to waste effort. \cite{Florencio2007DoStrongWebPasswords}	
	
	\cite{Shay2014ReligiousAunt}

	\subsection{Password Reuse}
	too many accounts problem.
	
	People reuse their passwords and show predictable modification behavior \cite{Gaw2005ReuseRecycle}
	
	The burden of passwords was relatively high in 2007, people reuse passwords \cite{Florencio2007LargeScaleStudyPasswordHabits}
	
	finite effort, and the payoff is invisible (comparison to smoking: I won't be affected, and in many cases that's true. But if you are affected you regret your behavior. )
	
	reuse is the most convenient way but probably the most severe threat to one's online identity and finances. This is a hard problem. 
	
	it's not just the passwords, it's also the user name 
	
	frequently entered passwords are reused more often \cite{Wash2016UnderstandingPasswordChoices} -- but there are contradictory results on this matter. -- Stobert and Biddle argue in the other direction \cite{Stobert2014PasswordLifeCycle}. 
	
	reuse statistic overview of different papers: 
	\cite{Wash2016UnderstandingPasswordChoices} in the discussion section
	Users try to prioritize stronger passwords as reuse candidates, and this also means that users try to follow security advice (they belief strong passwords are more important than unique passwords) \cite{Wash2016UnderstandingPasswordChoices}
	
	term that you read ever so often: users have a ``go-to password'' that they try first
	and then often the policy can make them change another one. But! If the go-to password is strong and has certain characteristics (as is demonstrated in chapter \ref{chap:policies-reuse}), reusing this is possible and its threats perhaps underestimated. 
	
	\textbf{What's the problem?}
	consequences: phishing attacks are problematic because of the domino effect (Ives \etal) Password reuse is difficult to defend against and we should look into understanding user behavior better \cite{Ives2004DominoEffectReuse}.
	
	but reuse isn't bad per se, it's necessary \cite{Florencio2014PasswordPortfoliosFiniteUser, ZhangKennedy2016RevisitingPasswordRules}. 
	

	
	
	It's enough to know one low-value password that you mangle to crack a large part of high-value passwords \cite{Haque2014Hierarchy}
	
	
	%    \subsubsection{Reuse Strategies}
	address the role of policies (see \cite{Seitz2017PoliciesReuse}).
	
	Participants rely on their memory, but reuse passwords, and they have a suboptimal mental model of how attacks work \cite{Gaw2006PasswordManagement}
	
	Users don't care if an account is financial or not, as long as it's perceived as high value, they reuse the password for that \cite{Bailey2014StatisticsReuse}
	
	password reuse is common and modifications are predictable, so it's easy for attackers to optimize their attacks with knowledge from leaked credentials of a specific account \cite{Das2014TangledWeb}

	Password Categories -- arch over to mental accounting from behavioral economics -- \cite{Thaler2004}
	
	Experts have certain articulate strategies to select and manage their passwords, and their situation awareness lets them judge important and non-important accounts more consistently  	\cite{Stobert2015ExpertPassword} 
	but: not all of them behave the same way \cite{Loutfi2015PasswordsOtherSideOfTheFence}

	Categorization: 
	compare password categorization to mental accounting, then we can cite \cite{Stockinger2015TowardsBE}
	
	category: depending on policies. \cite{Stobert2014PasswordLifeCycle}
	
	
	``Users see `good' passwords (that are memorable and conform to the policy) as a `resource', which they continue to use for new applications even if the original use is no longer allowed.'' (\cite{Inglesant2010TrueCostOfUnusablePolicies})
	
	Users try to prioritize stronger passwords as reuse candidates, and this also means that users try to follow security advice (they belief strong passwords are more important than unique passwords) \cite{Wash2016UnderstandingPasswordChoices}.
	
	Coping with passwords by choosing weak passwords and reusing them is absolutely necessary, the problem is how to do it right \cite{Florencio2014PasswordPortfoliosFiniteUser}.
	
	A new scale and insights into password support. Password hierarchy \cite{Haque2015PhdProposal}
	
	\subsection{Tools}
	
	\subsection{Writing Down}
	\cite{Herley2012PersistenceOfPasswords} is in favor of writing down IF the location is secure enough.
	
	Users struggle to manage passwords, but are unfamiliar with supporting tools, so they have developed elaborate strategies to cope \cite{Stobert2014Agony}
	
	Many people like using a password logbook and have interesting motivations to do so \cite{Kothari2017PasswordLogbooks}
	
	\cite{Conklin2004PWAuthenticationSystemPerspective}
	
	Problem: Accessibility for Local Attackers
Word documents post-its (use a screen shot of french newspaper that was featured on tv and you could see one of their passwords in the back. spouses can access them .	


	\subsection{Fallback Methods}
	Click on ``forgot'' password basically every time - to avoid this, some services mainly rely on one time passwords, because users are going to forget theirs anyway. 

	\cite{Bonneau2015SecretsLies}
	


	\subsection{Account Sharing}
	Account sharing is very common and often necessary, while current designs do not take this behavior into account \cite{Singh2007PasswordSharing}


\section{The Role of Mobile Devices}
touch some small aspects, especially Melicher's Paper \cite{Melicher2016UsabilityMobileTextPasswords} and \cite{VonZezschwitz2014HoneyIShrunkTheKeys}
\cite{Haque2014PsychometricsStrongPassword} 

talk about the rise of graphical and biometric authentication. 


set the stage for the emoji passwords later. 



%%%%%%%%%%%%%%%%%%%%%%%%%
%%%%%
%%%%% COUNTERMEASURES
%%%%%
%%%%%%%%%%%%%%%%%%%%%%%%%
\section{Countermeasures}

if we must keep the human in the loop, e.g. because constraints dictate so, we need to support them well and carefully \cite{Cranor2008FrameworkReasoning}

Usable security shouldn't only be done for end users but also for the people who implement security systems \cite{Acar2016NotYourDeveloper}


	\subsection{Password Composition Policies}
	
	the idea of policies dates back to the 70s: Morris and Thompson suggested to make users
	either choose longer passwords or assign passwords to them - There are flaws in password authentication and we can do something to alleviate the problem, but cannot get rid of it entirely \cite{Morris1979PasswordSecurity}
	
	The algorithms at the time weren't suitable to defend against cracking attacks so it was thought to introduce a few mechanisms to shift responsibility to the users \cite{Feldmeier1990UnixPasswordSecurity}
	
	%TODO here we need a reference to Bill Burr's idea of nailing policies down to a NIST standard.

	Proctor \etal found in 2002 that certain ``proactive password restrictions'' lead to stronger passwords \cite{Proctor2002ImprovingAuthenticationProactivePasswordRestrictions}. In two laboratory experiments, they had participants create a new password for a university account. The policies differed in the required minimum length (five and eight) and additional requirements like upper-/lowercase letters and digits. In the first experiment, where passwords only had to be five characters long, introducing the additional requirements had a greater effect on strength than in the eight-character-minimum condition. Increasing the minimum length by three characters already had a stronger effect. Interestingly, they concluded ``Perhaps the most important message of this study is that restrictions on user-generated passwords may not accomplish their intended goals.'' The early hypothesis that policies are more or less ineffective is particularly surprising because in the fifteen years that followed, many research papers were written about such restrictions and many of them ultimately came to similar conclusions. In the following some of the most influential works are summarized.
	%now we need to report on a couple of papers that come to the conclusion that policies are somewhat ineffective.
	Inglesant and Sasse report on a diary study of password policies in corporate contexts \cite{Inglesant2010TrueCostOfUnusablePolicies}. a postulation that password policies should be based on HCI principles rather than security considerations alone. The idea of a holistic password policy is put forward 
	
	 
	
	%Lingo: Adhere to a policy
	

	
	
	Weir \etal categorize policies into ``explicit'' and ``implicit'' policies \cite{Weir2010MetricsPolicies}, where explicit policies have predefined rules about the password structure (e.g. LUDS policy). Implicit policies are based on the estimated strength and are somewhat more volatile and intransparent to the users. Example: blacklist only becomes visible once the user tries to pick a password that's contained in the list. There are also ``external'' policies where the user's password is automatically changed by the system to add some randomness
	
	
	Policies influence security and usability, meters are often ineffective because only stringent meters seem to work \cite{Ur2012HelpingUsersCreateBetterPasswords}
	
	LUDS as a key term \cite{Wheeler2016zxcvbn}
	
		
	Range of Policies as shown by Shay \cite{Shay2014CanLongPasswordsBeSecureAndUsable}
	definitely mention: 28.0\% of passwords in comp8 fulfilled the symbol requirement only by placing ``!'' at the end and using no other symbols. 
	
	Zhang-Kennedy 2016 (here and also below) \cite{ZhangKennedy2016RevisitingPasswordRules}
	
	
	Workplace-focused: 
	
	
	The authors present how they would design an experiment about persuasive messaging, but didn't really carry it out (and the method doesn't seem suitable) \cite{Zakaria2013DesigningEffectiveSecurityMessages}
	
	Don't try to fix the user, fix the system first, especially do not impose strict requirements and nonsensical policies \cite{Florencio2014AdministratorsGuide} 
	
	Policy changes are a nuisance, but users feel more secure afterwards, and users write down passwords, share them and base them on dictionary words 	\cite{Shay2010EncounteringPasswordRequirements}
	
	Most detailed insights into effects of password policy design we have to date, password substring blacklist suggestion \cite{Shay2016DesigningPasswordPolicies}
	
	\cite{Weir2010MetricsPolicies}
	
	\cite{Wang2015EmperorsPolicies}
	
	
	\cite{Florencio2010WhereDoPoliciesComeFrom}
	
	\cite{Horsch2016PasswordPolicyMarkup}
	
	\cite{Chiasson2015QuantifyingExpiration}
	\cite{Blocki2013OptimizingPasswordPolicies}
	
	Early comparison of the effects of policies on \textit{human} password selection by Komanduri \etal - A carefully chosen policy can lead to more secure and more usable passwords, in this case the basic16 policy was well-received \cite{Komanduri2011OfPasswordsAndPeople}
	
	
	Shay tried to come up with an algorithm that lets administrators decide which policy to use - We could pick a password policy algorithmically based on a number of variables in this context \cite{Shay2009PolicySimulation}.
	
	basic16, 3class12, and 2word16 seem like the winners in terms of security and usability, but persuasion can help to diversify the themes for word-based passwords \cite{Shay2014CanLongPasswordsBeSecureAndUsable}
	
	
	Penalizing users by making them wait can be a strong nudge to comply and create a stronger password \cite{Malkin2013Waiting}
	
	It would be nice to have an easy way to create a repository with all password policies, but it isn't easy - at least it helps to agree on a standard language to define a password policy \cite{Steves2015PasswordPolicyLanguage}
	
	It would be good to have a formal description of password policies in a standardized schema \cite{Horsch2016PasswordPolicyMarkup}

	\subsection{Advice and Guidelines}
	
	there is a plethora of advice and guidelines out there, but it's not all the same \footurl{https://www.ncsc.gov.uk/guidance/helping-end-users-manage-their-passwords}{22.12.2017}
	
	education in password matters only has so much effect. 
	\cite{Forget2007HelpingUsers} says that even instructions don't work
	
	There's some work that argues that users want to create stronger passwords at least for some accounts, but they had
	misinterpreted security advice. 
	this is also reflected in \cite{Ur2016PerceptionsPassword}
	
	Great overview and critical discussion: \cite{ZhangKennedy2016RevisitingPasswordRules}
	
	
	make it persuasive \cite{Zakaria2013DesigningEffectiveSecurityMessages}
	
	In privacy: people's mental models about privacy are vague and we can see it in their drawings/explanations that they're somewhat overly pessimistic and that education plays the major role \cite{Kang2015MentalModelsDrawing}
	
	
	Stay realistic about password requirements and user effort \cite{Florencio2016CommACM}
	
	We can give better recommendations to users that are appropriately secure, actionable and usable \cite{ZhangKennedy2016RevisitingPasswordRules} (maybe also mention in policy section)
		
	
	\subsection{Offering Memorization Techniques}
	
	%TODO some papers actually could be categorized as password generator papers (semi-involved with human password selection.)
	
	early approaches: random but pronounceable passwords. (good overview in \cite{Kuo2006HumanSelectionMnemonic})
	
	\cite{Bonneau2014ReliableStorage56Bits}
	\cite{Forget2007HelpingUsers}
	Sort of displaced here, but it might be discussed in Section \ref{sec:rw:authentication_without_pws} \cite{Forget2015CYOA}
	\cite{Brown2004GeneratingPWs}
	
	In a convenience sample, persuasive text passwords were sufficiently memorable \cite{Forget2008MemorabilityPersuasivePasswords}
	
	It might be worthwhile to add a gamification layer onto password managers if you want to memorize some of your most important passwords \cite{Kroeze2012GamifyingAuthentication}
	
	They proposed a model to securely generate site-specific passwords, but this is all just an idea \cite{Maqbali2016PasswordGenerators}
	
	
	Security-wise this maybe a good idea, but the actual implementations is not as usable as it could be, however letting users modify system-assigned passwords seems beneficial in general \cite{Huha2015UserReplaceablePasswords}
	
	\paragraph{Mnemonics}
	Using contextual cues related to a website to base a mnemonic PW on might make them more memorable \cite{Mcevoy2016ContextualizingMnemonicPhrase}
	
	Yet another memorization aid - work in progress \cite{Lyastani2016PWMangling} 
	
	Even mnemonic phrase based passwords can be attacked easily, but if you tell people to choose something personal that no one else would choose and give an example might be the best piece of advice \cite{Yang2016MnemonicSentenceBased}
	
	People's mnemonic passwords are predictably based on common phrases from movies, literature, etc., but still the resulting passwords are better than traditional passwords \cite{Kuo2006HumanSelectionMnemonic}
	
	\paragraph{Passphrases}
	(\todo{maybe even merge this with subsection on advice})
	Advantages: PW scheme doesn't have to be changed, people are generally familiar with the concept of passwords, better to enter on virtual keyboards, e.g. TVs (although we don't have any data for that, but that's okay because passwords play a minor role (but still exist there)).
	
	Mnemonic phrase based passwords are strong and memorable \cite{Yan2004PasswordMemorabilitySecurity}.
	
	\cite{Keith2009PassphraseDesign}
	
	Disadvantages: more typos (can be relieved by displaying the password in plain text  \cite{Melicher2016UsabilityMobileTextPasswords}), user choice often predictable
	
	users might not even understand what you want from them if you say they should create a password based on a phrase \cite{Forget2007HelpingUsers}
	
	\cite{Bonneau2012LinguisticProperties}: passphrases already deployed in PGP, Caine and Abel password leak \cite{Carnavalet2014AnalyzingPWStrengthMeters} 
	
	\cite{Shay2012CorrectHorseBatteryStaple}
	
	
	\paragraph{Pronounceable Passwords}
	pronounceable passwords are useful, but we don't know the best way to generate them \cite{Goldberg2015UnspeakablePasswords}
	
	
	\subsection{Password Meters and Real-Time Feedback}
	
	\todo{Define what we mean by password meter} Because some papers don't differentiate between the meters, verbal feedback, suggestions, and real-time feedback on policy fulfillment. (this will be good to show with examples from the real world.)
	
%	Shay \etal say that real-time feedback is not part of the password meter \cite{Shay2015SpoonfulOfSugar}
	
	
	History: zxcvbn paper has some intel on early work on proactive password checks (in the 80s) \cite{Wheeler2016zxcvbn}. specifically (from 1995) \cite{Bishop1995ProactivePasswordChecking}
	
	
	pro-active checks -- dictionary checks Shay argues in favor of dictionary checks \cite{Shay2014CanLongPasswordsBeSecureAndUsable} 
	black lists \cite{Habib2017Blacklists} 
	
	spoonful (guidance) \cite{Shay2015SpoonfulOfSugar}
	\cite{Forget2008ImprovingPasswordsThroughPersuasion}
	
	be careful not to talk too much about persuasion in this chapter. 
	
	We should consider social nudges in usable security, too \cite{DiGioia2005SocialNavigationUsableSecurity}
	
	
	A sudden surge of n-gram scores can be used to detect modifications of common/weak passwords \cite{Tupsamudre2016MarkovStrength}
	
	
	passwords created using telepathwords were stronger and still memorable, but some Ps didn't like the idea \cite{Komanduri2014Telepathwords}
	
	Password meters from the industry are bad, machine-learning meters are good \cite{Wang2016fuzzyPWM}
	
	Users can well estimate the strength of their passwords, and can be nudged to act more securely for important accounts, but not for unimportant stuff \cite{Egelman2013DoesMyPasswordGoUpToEleven}
	
	Meters have a small effect on password strength, but slightly annoy people and it doesn't really matter how they look \cite{Ur2012HowDoesYourPasswordMeasureUp}
	
	Preliminary work to Ur et al.'s data-driven password meter \cite{Eargle2015YouCanDoBetter}
	Use text feedback to craft more effective password meters, but this depends on the policy used \cite{Ur2017DataDrivenPWMeter}
	
	If you tell people how long it will take to crack their password, they are more likely to read your recommendations and modify their password \cite{Vance2013FearAppeals}
	
	Password meters score the same password very inconsistently and don't explain their scoring, and it seems the Dropbox meter is one of the better solutions \cite{Carnavalet2014AnalyzingPWStrengthMeters}
	
	the presentation of password requirements can affect usability and security, because it can prevent errors and speed up selection for some stricter policies \cite{Shay2015SpoonfulOfSugar}
	
	\subsection{Password Managers}
	
	list all the pro's and con's of PWMs here in different dimensions, e.g. security and usability, kind of similar to \cite{Bonneau2012ReplacePasswords}
	
	Dashlane has the best usability and security tradeoff - but the usability evaluation is kind of worthless without more background information \cite{AriasCabarcos2016ComparingPWM}
	
	Tapas is a password manager based on two-factor authentication and was well received in a 30/10 participant user study \cite{McCarney2012Tapas, McCarney2013PWMThesis}
	
	%% from SOUPS MM Poster
	We situate our work in understanding user behavior and attitudes regarding passwords. Here, large parts of the literature focus on \textit{coping strategies} that emerge with a growing number of accounts \cite{Florencio2007LargeScaleStudyPasswordHabits, Florencio2014PasswordPortfoliosFiniteUser}. For example, Stobert and Biddle conducted qualitative analyses to formalize the way users live with their passwords (the ``Password Life Cycle") \cite{Stobert2014PasswordLifeCycle}. This model depicts how users choose, commit, reuse, and reset their passwords. Their work also delivers valuable insights into memorization and organization strategies: Users have mental lists of passwords, e.g. a list for important accounts or a list per website topic. Without explicitly mentioning, the findings contribute to a mental model of password reuse. This is important, because reuse is one of the most common coping strategies \cite{Das2014TangledWeb, Gaw2006PasswordManagement, Hayashi2011DiaryStudyPWs} and many researchers discourage it, because a breach at one site can compromise many others \cite{Bonneau2012ScienceOfGuessing, Komanduri2011OfPasswordsAndPeople}. 
	
	To facilitate coping with passwords and possibly minimize reuse, dedicated tools have been investigated and proposed. Besides industry-driven password managers, HCI research has proposed a number of alternatives. For instance, Stobert and Biddle also propose a password manager that is designed to boost trust as it does not directly store passwords, but rather offers a image cues to recall passwords \cite{Stobert2014PWMThatDoesntRemember}. 
	
	Finally, other researchers followed a mental model approach to understand how users make sense of security mitigations. For instance, Kang et al. utilized drawing tasks to establish users' mental model of data disclosure on the Internet \cite{Kang2015MentalModelsDrawing} to find guidelines for more privacy-sensitive solutions. Bravo-Lillo et al. focused on creating a mental model of security warnings \cite{BravoLillo2011WarningsMentalModel} to improve their framing and timing.
	
	
	\cite{Bojinov2010KamouflagePWM}
	
	\cite{Fagan2017UsersConsiderationsPWMs}
	
	
	
\section{The Passionate Discourse About Passwords}\label{sec:rw:passionate_discourse}
talk about how critical and passionate this topic has been debated in the literature

--> surprisingly, a lot of position papers (sometimes pure argumentation, sometimes with mathematical/logical reasoning) 

Address Herley's, Florencio's, Sasse's work here. 

e.g. Counterfactuals, Bullying, Finite Effort 
