\chapter[The Human Side: Selecting and Coping]{The Human Side: Selection and Coping}\label{chap:selection_coping_strategies}


Flor\^{e}ncio and Herley probably conducted the largest study to date on password habits. Their intention was to find out among other things A) how often people type passwords, B) how many sites share a password C) how many distinct passwords a user has, and D) how the strong the passwords are. They utilized the Windows Live Toolbar for Internet Explorer to collect in-the-wild data from up to 500000 users during three months of running the collection

\cite{Florencio2007LargeScaleStudyPasswordHabits}. The established protected password lists (PPL) to avoid intruding into people's privacy. They found that users had about 7 distinct passwords in 2007, and that passwords are re-used at about 6 sites in average. Interestingly, they found that stronger passwords are not re-used as often as weak passwords (only around 4 sites). It was not possible to trace the incoming data back to a specific user, which might have resulted in over counting of entries. Also, it was not measured how long the actual password entries takes. If users only used regular dictionary words without any modification, the key logging module of the toolbar would record a password reuse event (PRE) every time the user entered the word -- also in regular text searches, for example. Another limitation could be that they used entropy as a proxy for password strength. However, as discussed in the previous chapter, we have seen that this metric is more robust for system-generated passwords and that strength estimation has evolved over the past ten years. 


@@TODO cite Wash paper @SOUPS 2016.


\section{Methodology: Running Password Studies}

general overview over 

	\subsection{Ecological Validity}
	\subsection{Mechanical Turk Studies}
	
\section{How Users Traditionally Cope with Passwords}
	\subsection{Password Reuse}
	\subsection{Selecting Weak Passwords}
	
	\subsection{Tools (Writing Down)}
		\subsubsection{Problem: Accessibility for Local Attackers}
Word documents post-its (use a screen shot of french newspaper that was featured on tv and you could see one of their passwords in the back. spouses can access them .

	\subsection{Problem: Password Re-Use}
		\subsubsection{Policy Fulfillment}
		\subsubsection{Re-Use Approaches}

\section{Helping Users}
	\subsection{Advice and Guidelines}
	
	education in password matters only has so much effect. 
	
	
	\subsection{Memorization Techniques}
	\subsection{Real-Time Feedback}
	\subsection{Password Managers}
	
	%% from SOUPS MM Poster
	We situate our work in understanding user behavior and attitudes regarding passwords. Here, large parts of the literature focus on \textit{coping strategies} that emerge with a growing number of accounts \cite{Florencio2007LargeScaleStudyPasswordHabits, Florencio2014PasswordPortfoliosFiniteUser}. For example, Stobert and Biddle conducted qualitative analyses to formalize the way users live with their passwords (the ``Password Life Cycle") \cite{Stobert2014PasswordLifeCycle}. This model depicts how users choose, commit, reuse, and reset their passwords. Their work also delivers valuable insights into memorization and organization strategies: Users have mental lists of passwords, e.g. a list for important accounts or a list per website topic. Without explicitly mentioning, the findings contribute to a mental model of password reuse. This is important, because reuse is one of the most common coping strategies \cite{Das2014TangledWeb, Gaw2006PasswordManagement, Hayashi2011DiaryStudyPWs} and many researchers discourage it, because a breach at one site can compromise many others \cite{Bonneau2012ScienceOfGuessing, Komanduri2011OfPasswordsAndPeople}. 
	
	To facilitate coping with passwords and possibly minimize reuse, dedicated tools have been investigated and proposed. Besides industry-driven password managers, HCI research has proposed a number of alternatives. For instance, Stobert and Biddle also propose a password manager that is designed to boost trust as it does not directly store passwords, but rather offers a image cues to recall passwords \cite{Stobert2014Versipass}. 
	
	Finally, other researchers followed a mental model approach to understand how users make sense of security mitigations. For instance, Kang et al. utilized drawing tasks to establish users' mental model of data disclosure on the Internet \cite{Kang2015MentalModelsDrawing} to find guidelines for more privacy-sensitive solutions. Bravo-Lillo et al. focused on creating a mental model of security warnings \cite{BravoLillo2011WarningsMentalModel} to improve their framing and timing. 
