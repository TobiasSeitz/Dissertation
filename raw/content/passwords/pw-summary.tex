%!TEX root = ../../diss.tex

\chapter[Related Work Summary]{Related Work Summary}\label{chap:rw:summary}
%lingo: school of thought

\subsection*{Main Take-Aways}
% what is the current state of the world?
The current landscape of password research deals with both technical (Chapter \ref{chap:rw:passwords}) and human factors (Chapter \ref{chap:rw:user_perspective}). From a systems perspective, we can take away the following state of the world:
\begin{enumerate}
	\item There have been numerous attempts to replace passwords, but none of them are a panacea, because they entail a number of disadvantages that thwart their adoption. Any authentication scheme needs to withstand a plethora of attacks, but passwords seem to offer the best trade-off regarding usability, deployability and security compared to other schemes. Single-Sign On, multi-factor, as well as biometric authentication have gained importance and are likely to be further adopted in the future. Especially implicit biometric approaches can complement password-based authentication.
	\item Entropy as a password strength proxy only works for system generated passwords. Thus, the results of a number of studies would need to be reassessed or replicated through new studies. As of now, the most reliable metric is performing actual guessing attacks on passwords. Configuring attacks is not trivial but crucial for the results. Currently, the Password Guessing Services is seen as the most robust tool. If cracking is impossible (e.g. for proactive password checks), the best strength proxy is to estimate a guess number for skillful, informed, and resourceful attackers. Neural networks and zxcvbn are among the most useful proxies at this point, but it is worthwhile to triangulate to obtain the full picture.  
	\item Password strength is difficult to agree upon and highly context dependent. This makes it hard to give homogeneous, consistent advice to users.
\end{enumerate}
The technical aspects are contrasted and complemented by these human factors:
\begin{enumerate}
	\item Users need to manage many password-protected accounts, which is one of the drivers to come up with individual coping strategies. Although these strategies are often intuitively developed, they show commonalities among larger user groups. Passwords go through a life-cycle no matter what strategy is used. Much of user behavior can be classified as risky, but in many ways it is rational.
	\item Users tend to select predictable passwords, reuse them often, and write them down in physical or digital notes. While there are many hypotheses to explain user behavior, the problem is probably too nuanced to reach a final consensus regarding its origins. It appears to be an assessment on a per-user level.
	\item Many solutions have been proposed to guide and support users. Policies enforce requirements on users to mitigate strength issues. However, the design space of password policies is large and the best parameters are context dependent. Policies do not guarantee ``better'' passwords, but real time feedback can make them more bearable. Password managers can take some responsibility from users which had been shifted to them in previous years. Explicit user education was once seen as primary tool to combat risky behavior, but users continued to dismiss it, which has led to the understanding that education alone does not solve all problems.
	\item Persuasive design has been a well-studied area in password authentication to aid user education. It leverages cognitive biases and heuristics to achieve more secure user behavior. Password meters encompass many persuasive strategies at once and offer an interesting design space to influence password selection. However, their real-world implementations have brought inconsistencies to light, which reduced their persuasive power. 
	\item Studying passwords has a multitude of facets and the method toolkit is large. Most studies rely on online-surveys (especially through \gls{mTurk}), usability tests in the lab, or qualitative interviews. Several principles for the design of valid studies in usable security and privacy have emerged in the past two decades. 
\end{enumerate}

\subsection*{Open Questions}
Based on the review of the related literature we have identified a number of questions that have not been answered to full extent. 
% what do we not already know about the current state of the world
% optimally, this motivates all the research questions for the following chapters?
\begin{enumerate}
	\item Since results regarding user behavior and knowledge about password strength are mixed, we still have not uncovered all context-factors and confounding variables for password selection. However, we need to have this kind of understanding if we aim to design better support systems to either reinforce or break certain habits. Since user education has been going on since decades, we ask: What have users learned about password security in the meantime? What are their mental models of password strength? Can we trace back certain behavior to failed mitigations, or are certain personality traits more decisive?
	
	\item The design space for persuasive interventions has not been fully exhausted. The effectiveness of persuasion is still somewhat low. What are novel, radical approaches to steer users towards stronger passwords? How do we find them? Which cognitive biases might be most suitable for the design of persuasive password interventions? Can we find new horizons to empower users to create stronger passwords that are still memorable? What would a holistic solution look like?
	
	\item Password reuse is rampant and still a great risk for users, primarily in the form of identity theft following social engineering attacks. However, this aspect has not been addressed by the design of password policies. If policies cannot always effectively combat weak passwords, do they prevent password reuse? Can persuasive design aid the secure reuse of passwords?
\end{enumerate}

I try to answer these questions with empirical data and reasoning in the remainder of this thesis.


