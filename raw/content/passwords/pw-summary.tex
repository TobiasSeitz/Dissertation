%!TEX root = ../../diss.tex

\chapter[Related Work Summary]{Related Work Summary}\label{chap:rw:summary}

%lingo: school of thought

\section{Main Take-Aways}
What are the main take-aways?

\begin{enumerate}
	\item There have been numerous attempts to replace passwords, but all of them kind of failed.
	\item even if you force users to include certain elements in their passwords, the outcome isn't necessarily ``better'' (stronger, more usable)
	\item using entropy as a password strength proxy is unsuitable, which renders a couple of results and claims questionable (e.g. strength reports in \cite{Florencio2007LargeScaleStudyPasswordHabits})
	\item configuring a cracking approach correctly is hard, but very important to reliably estimate a given password's capacity to withstand different attacks \cite{Bonneau2012ScienceOfGuessing, Kelley2012GuessAgain, Ur2015MeasuringRealWorldAccuracies, Weir2009PCFG}
\end{enumerate}

\section{Open Questions}
\begin{enumerate}
\item If policies cannot effectively combat the issue of password strength, do they prevent password reuse which appears to be a more problematic issue? 
\end{enumerate}

\section{Meta}

Summarize: what were the years when most work was done? what type of research was it? what are the most common methods?

It's a bit hard to pinpoint specific contributions, because the papers often shed light on certain nuances of password selection -- just like this thesis.

A lot of position papers, respectively theoretical papers with no empirical data.

CMU focuses on policies and tries tiny modifications of policies and feedback. It's like you almost can predict the next paper they are going to write. Stringent line of research (look at conclusions and wait for the paper).


there are very few papers that try to replicate findings


