\subsection*{The Passionate Discourse About Passwords}\label{sec:rw:passionate_discourse}
As a meta summary, we observe that the often mixed or even contradictory results of empirical password studies have generated a very passionate discourse about them. 

Different schools of thought: one line of researchers want to influence user behavior (policies) and believes in the efficacy of making users pick stronger passwords and behave securely. Another group wants to avoid passwords as much as possible by advocating other authentication schemes. The third group is more fatalistic in the sense that they do not see much hope for security interventions because users are not going to react to them in the desired way (Herley, Florêncio)

.. perhaps the discourse might calm down somewhat if more replication studies were carried out. However, \textit{publication bias} plays a significant role in the decision to go through the effort of replicating a study. 


Summarize: what were the years when most work was done? what type of research was it? what are the most common methods?

It's a bit hard to pinpoint specific contributions, because the papers often shed light on certain nuances of password selection -- just like this thesis.

A lot of position papers, respectively theoretical papers with no empirical data.

CMU focuses on policies and tries tiny modifications of policies and feedback. It's like you almost can predict the next paper they are going to write. Stringent line of research (look at conclusions and wait for the paper).

%there are very few papers that try to replicate findings

\todo{restructure, this doesnt need to go here, merge with above section?}
talk about how critical and passionate this topic has been debated in the literature

--> surprisingly, a lot of position papers (sometimes pure argumentation, sometimes with mathematical/logical reasoning) 

Address Herley's, Florencio's, Sasse's work here. 

e.g. Counterfactuals, Bullying, Finite Effort 


central theme: solution XYZ is going to cure everything and will kill passwords in the long run. \subsection*{The Passionate Discourse About Passwords}\label{sec:rw:passionate_discourse}
As a meta summary, we observe that the often mixed or even contradictory results of empirical password studies have generated a very passionate discourse about them. 

Different schools of thought: one line of researchers want to influence user behavior (policies) and believes in the efficacy of making users pick stronger passwords and behave securely. Another group wants to avoid passwords as much as possible by advocating other authentication schemes. The third group is more fatalistic in the sense that they do not see much hope for security interventions because users are not going to react to them in the desired way (Herley, Florêncio)

.. perhaps the discourse might calm down somewhat if more replication studies were carried out. However, \textit{publication bias} plays a significant role in the decision to go through the effort of replicating a study. 


Summarize: what were the years when most work was done? what type of research was it? what are the most common methods?

It's a bit hard to pinpoint specific contributions, because the papers often shed light on certain nuances of password selection -- just like this thesis.

A lot of position papers, respectively theoretical papers with no empirical data.

CMU focuses on policies and tries tiny modifications of policies and feedback. It's like you almost can predict the next paper they are going to write. Stringent line of research (look at conclusions and wait for the paper).

%there are very few papers that try to replicate findings

\todo{restructure, this doesnt need to go here, merge with above section?}
talk about how critical and passionate this topic has been debated in the literature

--> surprisingly, a lot of position papers (sometimes pure argumentation, sometimes with mathematical/logical reasoning) 

Address Herley's, Florencio's, Sasse's work here. 

e.g. Counterfactuals, Bullying, Finite Effort 


central theme: solution XYZ is going to cure everything and will kill passwords in the long run. 