%!TEX root = ../../diss.tex

\chapter[The Downfall of Passwords]{The Downfall of Passwords}\label{chap:downfall_of_passwords}


\section{Shared Authentication}
Single Sign-On (SSO). 
\subsection{Technical Challenges}
Identity Providers, 
Reference Models:
\begin{itemize}
\item \textbf{OAuth} Twitter, Google
\item \textbf{OpenID} Google, PayPal
\item \textbf{Facebook Connect} 
\end{itemize}
Problems: 
Successful identity providers such as Facebook take a central role as the ``sole identity provider, which does little for privacy'' \cite{Bonneau2015ImperfectAuthentication}.
\subsection{Studies}
\subsection{Limitations}

\textbf{Privacy} The biggest players for shared authentication are Facebook, Google, and Twitter. However, it is exactly these three companies for which users raise the most privacy concerns. On the one hand, users trust the companies because they know how much data they store and manage, and only few breaches are known. On the other hand, users are doubtful that their credentials will be in safe hands and not shared with third parties (users lack the understanding of how shared authentication works internally and cannot separate privacy and security). 

\textbf{Single Point of Failure} Embracing the opportunities of shared authentication, users integrate it into their password management / coping strategies \ref{chap}. 

\section{Authentication on Mobile Devices}



\section{Do we still need passwords?}

% das passt nicht unbedingt hier rein, ist aber sehr wichtig. 
Bonneau et al. argue \cite{Bonneau2015ImperfectAuthentication} that passwords are an imperfect technology that are difficult to replace. One, industry has found ways to work around the many drawbacks and can compensate breaches to a large part. Second, alternatives to passwords are often privacy invasive. For example, identity providers like Google or Facebook often collect large amounts of personal data on the users. Third, Bonneau et al. point out that empirical evidence from practice often contradicts results produced by academic research. They advocate that researchers rethink their model of users, who often behave too predictably and whose behaviors one should not try to change. Instead, academia could tackle new approaches that would be dangerous to the success of businesses and thus are seldom tried out. 

% noch mehr Zusammenfassung für später:
The authors point out how user models assumed by researchers often do not apply in reality, for instance, their behavior is anything but random: Users pick from a limited set of passwords that is far smaller than random passwords. Just as \cite{Florencio2014PasswordPortfoliosFiniteUser}, the paper strongly discourages focusing on offline attack scenarios and suggests that users should at most try and protect themselves against online attack scenarios. Consequently, the lessons from the past about attempts to improve password strength through changing user behavior are questionable in the authors' eyes. Even more so, because other attacks (phishing, malware, eavesdropping, stealing from servers or identity tokens) are not fended off at all by stronger passwords. Online attacks can drastically be mitigate by rate-limiting and contextual information (e.g. geolocation). Yet, the authors see that not all sites employ these methods, ``probably to avoid denial of service''. Users also often receive too much advice from security experts that is often contradictory and in extreme cases boiled down to ``Pick something you cannot remember and do not write it down''. 

Bonneau et al. answer the question whether we still need passwords with a differentiated ``yes'': \textit{Passwords appear to be a Pareto equillibrium\footnote{TODO add definition of this here. It's about game theory.}}. Also the learning curves for password authentication at a new service is virtually non-existent. However, as many researchers in the community, the authors see the feasibility of multi-factor authentication in progressive or continual ways as the most promising future. The challenges for privacy and usability we find in these approaches are mostly ill-defined or not validated. The user experience of such systems may improve while the drawbacks come at a high price that the users may not understand at all. 