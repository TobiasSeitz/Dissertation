%!TEX root = ../../diss.tex
\chapter[Foundations]{Foundations}\label{chap:rw:passwords}
%lingo: adversaries statt attackers

% general broad description of where passwords play a role and what authentication is.
Passwords are but one puzzle piece in the realm of cyber security. They are part of the access control paradigm, which is commonly divided into three steps: identification, authentication, and authorization \cite{Riley2006IdentityAuthenticationDistinct}. Identifying a user is usually done with a prompt for a user name, so the system tries to answer the question ``who are you?''. Authentication is about proving that a user -- or more broadly an entity -- is who they claim to be, i.e. authentication is about verification of an identified entity. In other words, the system asks ``How can you prove that you are who you claim to be?''. This verification process can be based on three central elements, namely something that you know (any kind of secret), something that you are (any kind of property), or something that you have (any kind of secret token). One could argue that the second category could be extended by ``something that you do?'', e.g. provide your individual signature. However, authentication does not necessarily \textit{require} identification as a first step. It is well possible to authenticate entities even if they are anonymous. One everyday example are passwords for WiFi networks: Most of the time, the access point does not require a user name; the password, or pre-shared key, in a WPA protocol is enough to join the network. 
Last, authorization is the decision over which resources an authenticated entity is allowed to use, or an answer to ``can the user access this?''?

For this thesis, \textbf{authentication} remains the center of attention. In this chapter, we take a look at various forms of authentication and establish an argument as to why ``something that you know'' is still the most prevalent and significant authentication paradigm to date. 

\section{A Brief History of Passwords}
The idea of protecting resources with ``something that you know'' is hundreds of years old. Think about the magic words ``open, Sesame!'' that Ali Baba spoke to enter a den used as treasury by forty thieves\footnote{Ali Baba is a fictional character in a story from ``Thousand and One Nights'' as recorded by Antoine Galland in the early 18th century. \url{http://www.pitt.edu/~dash/alibaba.html} \la{16.12.2017}}. The story already illustrates password misuse, because Ali Baba was not the rightful owner of the den and he impersonated the thieves. Nonetheless, this authentication paradigm was brought to the digital world in the early 1960s by the Massachusetts Institute of Technology (MIT) \footnote{\url{https://www.wired.com/2012/01/computer-password/}, \access{16.12.2017}}. Researchers had built a mainframe computer that was programmable by multiple users. At the time, one of the most valuable resources beside the physical device was the \textit{time} granted to use the machine. Thus, the Compatible Time Sharing System (CTSS) was conceived to give every user a certain quota of hours to operate the computer. The quota was enforced by creating user accounts that were password protected. The interaction very much resembles what we still use today: after typing the user name, the password is requested and hidden from the screen during entry. 

Shortly afterwards the flaws of the system became evident when 
- first exploit in 1962 that exposed passwords \footnote{\url{https://www.slideshare.net/CAinc/history-of-the-password/7-In_1962_a_software_bug}}. 
Alann Scherr wanted to use the computer for a longer time that's why he needed to impersonate other users of the same computer.
- 1978 NIST - Data encryption standard (DES), was alive until 1998 when attacks became feasible, then AES (1997) became the standard


The man who invented the password - Fernando Corbató
``I have to confess, I used to use a crib sheet,'' he told the Journal. ``I don't think I'm guarding any great secrets.''\footnote{\label{foot:burr_regrets}\url{http://www.businessinsider.com/inventor-of-the-password-2014-5?IR=T}, \access{23.10.2017}}

more history in the introduction of \cite{Bishop1995ProactivePasswordChecking}, and of course as early as 1979 \cite{Morris1979PasswordSecurity}



The benefits and drawbacks of using passwords have been studied extensively. Some drawbacks already become apparent when we recall Ali Baba's story: Ali Baba overhears the thieves saying the magic words and can immediately authenticate with them (a big security issue). He also told the ``password'' to his brother, but he fails to recall it (the biggest usability issue).
\begin{table}
	\begin{tabular}{|l|l|l|}
		Benefits | Drawbacks 
	\end{tabular}
\end{table}
benefits: easy to implement, users know how they work, concept is easy to learn, can be shared, can be replaced when they are compromised, can be revoked by the sys admin, quick to enter, work without a user name (WiFi) so it's easy to scale. 
drawbacks: users have to cope with many passwords, so they use risky strategies \ref{sec:rw:how-users-cope}, 
it's difficult to manually create a password that is both usable and secure (good enough to protect against offline attacks)


\section{What is Password Strength? - Metrics and Statistics}\label{sec:rw:pw_strength_metrics}


How to attack passwords
- here the most important paper is \cite{Ur2015MeasuringRealWorldAccuracies} -- mention tools like HashCat/oclHashcat, John The Ripper 

give some real world examples how passwords are attacked online and how they have been attacked offline. 

The initial approach towards estimating the strength of passwords was to look at their \textit{Shannon entropy}(@@Quelle). 

Lately the community reached widespread consensus that the realistic strength of password can be defined as \textit{the number of attempts that an attacker would need in order to guess it} \cite{Dellamico2015MonteCarlo}

Other definitions: ``The strength of a password should represent the amount of effort an adversary must employ to break the password'' \cite{Carnavalet2014AnalyzingPWStrengthMeters}. 

Don't forget: Password strength is dynamic! New leaks can render strong passwords very weak. Yang \etal state: ``If a strategy is widely used, then attackers may develop strategy-specific methods which can efficiently guess the passwords.''



\cite{Peisert2013PriciplesAuthentication,Bonneau2012ScienceOfGuessing,Scott1995GDMS,Dellamico2015MonteCarlo,Ur2015MeasuringRealWorldAccuracies,Egelman2015SeBIS,Conklin2004PWAuthenticationSystemPerspective,Schmidt2013Pitfalls,Kelley2012GuessAgain,Mazurek2013Measuring,Weir2010MetricsPolicies}

	\subsection{Entropy vs. Guesswork}
	
	
	William Burr regrets NIST policy (see footnote \ref{foot:burr_regrets})	
	
	Entropy (Shannon) -- degree of randomness. often too focused on characters and doesn't address human password selection, which is far more predictable 
	(effective vs. theoretical password space)
	
	@@TODO tell the story of all the attacks -- Kelley, Weir, Bonneau, Melicher, Johnson, Ur etc. Password Guessability Service etc. 
	
	Leaked password dictionaries: Most commonly: RockYou (gamer site), MySpace, LinkedIn, Gawker, Yahoo!
	Mark Burnett  \cite{Burnett2005PerfectPasswords} 
	
	Tell the history of how guesswork was modeled. Starts out with \cite{Weir2010MetricsPolicies}, then \cite{Kelley20012GuessAgain} and \cite{Bonneau2012ScienceOfGuessing}. 
	
	NIST entropy was THE measure for password strength until the Weir paper and shortly afterwards. \cite{Komanduri2011OfPasswordsAndPeople} is probably the last CMU paper that uses entropy as strength metric (and is proud of it).
	
	Entropy is relative to / dependent on a given set of passwords. 
	
	\subsection{Beta-guess-rate}
	$\lambda_{\beta}$
	quasi: how likely is it to find the top $\beta$ passwords in a given dataset?
	see bonneau, and also used in \cite{Yang2016MnemonicSentenceBased}
	brostoff and sasse  proposed 10 as a threshold for rate-limiting (before account is locked and password invalidated until reset) \cite{Brostoff2003TenStrikes}
	
	
	
	
	\subsection{The zxcvbn Approach}
Daniel Wheeler presented an approach towards password strength estimation by looking at a conservative expected guess attempt number \cite{Wheeler2016zxcvbn}. The idea is to utilize pattern matching against dictionaries and leaked password corpora and then calculate the minimum rank over a series of frequency ranked lists. In other words, the approach is heuristic instead of probabilistic, because the ranking is based on searching through the patterns and ranking them not only based on their likelihoods, but other factors like keyboard sequences. The implementation of the algorithm is called zxcvbn\footnote{The name zxcvbn originates from the bottom row on a QWERTY keyboard. Many users mistakenly consider this approach secure because the resulting password looks fairly random.}. Wheeler showed that in an online attack scenario \cite{Florencio2014AdministratorsGuide} the algorithm estimates the number of guesses accurately within an order of magnitude of 2 -- consistently better than NIST guidelines to date and KeePass strength estimators. That is, utilizing 100,000 tokens stored within a 1.5 Megabyte file zxcvbn conservatively estimates the number of guesses required to crack a password. Beyond the online-attack threshold, the results are mixed, but we can observe that adding more tokens to the dictionary improves accuracy even more. The great benefit of using this method is its speed and size that make it a lightweight tool that is prepared for widespread adoption, to relieve users from ``LUDS'' policies (lowercase, uppercase, digits, symbols). We can conclude that zxcvbn is a reasonable tool when we collect meta statistics about passwords, e.g. in studies where it is ethically questionable to collect plain text passwords \cite{Seitz2016SuggestionsDecoy}. Wheeler also points out that at this point there is no study comparing the effects the different estimators have on user behavior. 

papers that used or mentioned zxcvbn:
\cite{Komanduri2014Telepathwords}
\cite{Wang2016fuzzyPWM}
\cite{Ur2017DataDrivenPWMeter}
\cite{Yang2016MnemonicSentenceBased}

Weaknesses \cite{DeCarnedeCarnavalet2015PasswordMeters}: only english dictionary words, fixed dictionary with the constraints of being transfered to the client, so some passphrases are considered strong while they are not really super great, e.g. ``dolce\&gabana''
not true anymore but in the paper: reversed passwords are not recognized (Wheeler updated this). 


\section{What is a ``Bad'' Password?}
This question is flawed, and thus cannot be answered without context. The word ``bad'' is judgmental and depends on whom is asked. A security expert might call a password ``bad'' if it only contains lowercase letters, but a regular mainstream user might call it ``bad'' because they cannot type it quickly or can't remember it well. 

Bishop and Klein said in 1995: ``A good password is one that is easily remembered, yet difficult to guess.'' \cite[p. 231]{Bishop1995ProactivePasswordChecking}

Let's look at 

types of attacks: phishing, online (rainbow), offline, targeted online. \cite{ZhangKennedy2016RevisitingPasswordRules}. 

When does password strength really help? 

This paragraph from Herley and Van Oorschot says it all \cite{Herley2012PersistenceOfPasswords}:
``Fifth, when are off-line attacks a threat? While dependent on implementation, access to salted hashed pass- words requires attacker effort; long gone are the days when password hash files were by default world read- able. A disgruntled ex-sysadmin who steals hashed pass- words is the often-conjectured foe in this attack; yet, if un-trusted individuals have had unfettered unaudited access to the authentication server, a site’s problems go well beyond password strength. Sixth, are there ways to protect against off-line attacks besides password strength? Mandating password changes once hashes leak might be better than strong policies at all times. Only if a leak goes unnoticed (and a password change isn’t forced) does strength potentially help. Of course, reliably detecting leaks or break-ins it- self remains difficult. Finally, how much strength is required to protect against off-line attacks? The bar is clearly much higher than for online attacks (assuming lockout or rate-limiting policies in place), but at what strength are attacks effectively addressed? More strength is always better for security,'' 

discuss online and offline attacks
\cite{Wang2016fuzzyPWM} has an overview/comparison
\cite{Florencio2014AdministratorsGuide} shows models
\cite{Florencio2014PasswordPortfoliosFiniteUser} has math

A problem about only taking ``online'' attacks into account is that they do allow denial of service attacks, for example if there's lockout policy in place that invalidates passwords after 10 failed login attempts, an attacker would only need to take a list of email addresses (readily available on the internet) and run 10 or more guesses per user and lock all of them out at once. 



We shouldn't call it good or bad. A less judgmental terminology is ``strong'' and ``weak'' because this doesn't include opinions. 

A ``strong'' password is one that cannot be guessed easily by a human or a machine, regardless of the level of information they have about the owner of the password. 


The idea behind a PCFG based policy is to reject passwords that are too ``probable''. 

Characteristics of weak passwords \cite{Burnett2005PerfectPasswords}
\begin{enumerate}
	\item Are used by many users
	\item Can be found through an internet search
	\item Are short (6 characters or less)
	\item Have leaked
	\item Are written down where anyone can access them
\end{enumerate}

But there are other, non-technical qualities of passwords that are usually not covered by the weak-strong spectrum. 

When is a password \textbf{``Appropriate''}? This largely depends on the usage context. \cite{Gaw2005ReuseRecycle, Haque2014Hierarchy}
Florencio \etal argue that it is absolutely fine to choose a weak password for ``don't-care'' accounts \cite{Florencio2014}.

Requirements from a user perspective: memorable and easy/fast to type on all devices.

this creates tensions that are discussed in detail in Chapter \ref{chap:rw:user_perspective}.


\section{Authentication without Passwords}
% das passt nicht unbedingt hier rein, ist aber sehr wichtig. 
Bonneau et al. argue \cite{Bonneau2015ImperfectAuthentication} that passwords are an imperfect technology that is difficult to replace. One, industry has found ways to work around the many drawbacks and can compensate breaches to a large part. Second, alternatives to passwords are often privacy invasive. For example, identity providers like Google or Facebook often collect large amounts of personal data on the users. Third, Bonneau et al. point out that empirical evidence from practice often contradicts results produced by academic research. They advocate that researchers rethink their model of users, who often behave too predictably and whose behaviors one should not try to change. Instead, academia could tackle new approaches that would be dangerous to the success of businesses and thus are seldom tried out. 

% noch mehr Zusammenfassung für später:
The authors point out how user models assumed by researchers often do not apply in reality, for instance, their behavior is anything but random: Users pick from a limited set of passwords that is far smaller than random passwords. Just as \cite{Florencio2014PasswordPortfoliosFiniteUser}, the paper strongly discourages focusing on offline attack scenarios and suggests that users should at most try and protect themselves against online attack scenarios. Consequently, the lessons from the past about attempts to improve password strength through changing user behavior are questionable in the authors' eyes. Even more so, because other attacks (phishing, malware, eavesdropping, stealing from servers or identity tokens) are not fended off at all by stronger passwords. Online attacks can drastically be mitigate by rate-limiting and contextual information (e.g. geolocation). Yet, the authors see that not all sites employ these methods, ``probably to avoid denial of service''. Users also often receive too much advice from security experts that is often contradictory and in extreme cases boiled down to ``Pick something you cannot remember and do not write it down''. 

Bonneau et al. answer the question whether we still need passwords with a differentiated ``yes'': \textit{Passwords appear to be a Pareto equilibrium\footnote{\todo{add definition of this here. It's about game theory.}}}. Also the learning curves for password authentication at a new service is virtually non-existent. However, as many researchers in the community, the authors see the feasibility of multi-factor authentication in progressive or continual ways as the most promising future. The challenges for privacy and usability we find in these approaches are mostly ill-defined or not validated. The user experience of such systems may improve while the drawbacks come at a high price that the users may not understand at all. 

	\subsection{Graphical Passwords \& Visual Authentication Methods}
ColorPIN, Ema's work, \cite{Renaud2009VisualSnakeOil} 

	\subsection{Biometrics}
	
	\cite{Jakobsson2014HowToWearYourPW,DeLuca2012TouchMeOnce,Peisert2013PriciplesAuthentication,Rybnicek2014RoadmapContinuousAuth}
	

	fingerprint omnipresent 
	retinal scan
	face recognition (iPhone X)
	
	
	Anecdote to show that it is a bad idea\footnote{\url{https://www.theguardian.com/world/2017/nov/08/qatar-airways-plane-forced-to-land-after-wife-discovers-husbands-affair-midflight}, \access{08.11.2017}}
	
	\subsection{Multimodal and Implicit Authentication}
	
	actually the technical report gives some insights about this \cite{Stockinger2011ImplicitAuthentication}, so that will inspire a bit.
	also look into the USEC folder in the office, the papers and annotations are in there.
	
		it's interesting that the most cited CHI paper of the last 5 years is about implicit authentication \cite{DeLuca2012TouchMeOnce}
	
	Machine learning. Accuracy / Recall / Error rate / True Positve / False Positives etc. -- based on \textbf{probability} and not on \textbf{certainty}. To account for the lack of certainty, in all current state-of-the-art mechanisms passwords serve as fallback authentication. So, while it reduces the number of password entries and so speeds up interactions in many cases, these schemes cannot fully replace passwords. What is worse is that the rare interaction often tempts users to choose a very memorable secret, which in most cases is weaker than a password that is entered on a regular basis. So an attacker that is successfully denied access to a system protected by multimodal authentication will always have the opportunity to authenticate with a password -- the system needs to assume a false negative and offer the fallback authentication scheme. In this scenario, while the primary authentication is more usable and very secure, the entire system's security is lowered by its fallback scheme. This fact is often played down and neglected in the marketing of new multimodal authentication. 
	
	Take away: implicit authentication cannot replace explicit authentication, but it can make an existing scheme more secure. Fallback schemes tend to lower system security. 
	

	
	\subsection{Shared Authentication and Hardware Tokens}
Single Sign-On (SSO). 
Identity Providers, 
SecureID or ``Gnubby'' (see Google)
Two-Factor Authentication / 2-step --> hints to 2-step in ``Some UNIX systems have instituted what is called an "external security code" that must be typed when dialing into the system, but before logging in. If this code is changed periodically, then someone with an old password will likely be prevented from using it.'' \cite{Morris1979PasswordSecurity}

Reference Models:

\cite{Egelman2013ProfilePassword,Sun2010BillionKeys}

\begin{itemize}
\item \textbf{OAuth} Twitter, Google
\item \textbf{OpenID} Google, PayPal
\item \textbf{Facebook Connect} 
\end{itemize}
Problems: 
Successful identity providers such as Facebook take a central role as the ``sole identity provider, which does little for privacy'' \cite{Bonneau2015ImperfectAuthentication}.


\textbf{Privacy} The biggest players for shared authentication are Facebook, Google, and Twitter. However, it is exactly these three companies for which users raise the most privacy concerns. On the one hand, users trust the companies because they know how much data they store and manage, and only few breaches are known. On the other hand, users are doubtful that their credentials will be in safe hands and not shared with third parties (users lack the understanding of how shared authentication works internally and cannot separate privacy and security). 

\textbf{Single Point of Failure} Embracing the opportunities of shared authentication, users integrate it into their password management / coping strategies \ref{sec:rw:how-users-cope}. 

	
\subsubsection{One Time Passwords}

OTP also have the possibility to replace a password that you have to memorize, but there are disadvantages as well. 
illustrate the workflow from Slack.

Authentication Melee paper has some intel on one-time passwords: \cite{Ruoti2015AuthenticationMelee}


\section{Passwords are Here to Stay}
no solution fully replaces them

most approaches that are seen as more usable are often based on probability rather than certainty (TPR / FPR / EER). fallback to passwords probable even for other authentication schemes. everything else is often less usable than passwords. 

maybe discuss this at the end of the thesis: passwords are cool, and we can continue to use them because they are so easy to replace in case of a breach. 

An approach that is currently not undertaken in internal security audits is the following: there are password leaks once a month (see data breach reports). the security team could take the passwords (if they are hashed - crack them first) and invalidate all matching passwords in their own password database. Of course, they need to inform their users. I could sketch a flowchart what this should look like. This would allow reusing passwords more securely without shifting much responsibility to the users. 

``We argue that it is time to admit that passwords will be with us for some time, and moreover, that in many instances they are the best-fit among currently known solutions.'' \cite{Herley2012PersistenceOfPasswords} 

The Herley and Van Oorschot paper really has lots of good quotes and thoughts for this \cite{Herley2012PersistenceOfPasswords}


\cite{Kirlappos2012SecurityEducation,Loutfi2015PasswordsOtherSideOfTheFence,DeAngeli2005PictureThousandWords,Florencio2013WhereDoAllTheAttacksGo,Herley2008ProfitlessEndeavor,Sasse2015,Dittrich2009,Herley2009SoLongThanksExternalities,Vantaggiato2015WeStillNeedPasswords,Florencio2010WhereDoPoliciesComeFrom,Schrittwieser2013,Bonneau2015ImperfectAuthentication,Cyber2014,Florencio2007DoStrongWebPasswords,Sasse2005UsableSecurityPosition,Aebischer2017PicoInTheWild,Forget2007HelpingUsers,Herley2009IfWereSoSmart,Acar2016NotYourDeveloper,Sasse2016,Renaud2009VisualSnakeOil}

