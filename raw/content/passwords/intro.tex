\label{sec:rw:intro}


The following chapter is a survey of password literature of the last three decades. We discuss HCI publications from the publishers ACM, USENIX, IEEE, and the Internet Society. More than \todo{add number} publications bear the term ``password'' already in the title. This large number speaks for the importance of the topic and the many attempts of making progress in helping users. The meta analysis shows when the number of publications peaked and what kind of research method was used. Interestingly, much of the literature does not report on empirical studies but discusses certain phenomena on a higher level. 

\todo{talk about the different focus areas}: mental models, policies, feedback, attacking / cracking / technical / algorithms, coping strategies / managers, critical / positions, alternative systems.

%Internet revolution in the 90s, Web 2.0, Smartphone Revolution in 2007, Industry 4.0
%but also PINs for Credit cards and mobile phones etc. 

To illustrate these findings, this part covers three main aspects of research on password-based authentication. 

Chapter \ref{chap:rw:passwords} describes the foundations of password authentication, its advantages and drawbacks, potential alternatives to passwords, and why this topic is so passionately debated inside and outside the usable security community. 

Chapter \ref{chap:rw:user_perspective} takes a deep-dive into the HCI challenges arising from passwords and shows potential solutions. 

In Chapter \ref{chap:rw:persuasion}, we look at ways to influence decisions with persuasive design. We present interdisciplinary work to highlight the potential of this design direction to make coping with passwords more convenient for users.  

Finally, Chapter \ref{chap:rw:summary} takes a bird's eye view and briefly summarizes the central insights from past research. It also discusses the related work on a meta level. 

% doesnt have to be more than a page. Ema has got a nice introduction there. 