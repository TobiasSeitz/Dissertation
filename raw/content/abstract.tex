\selectlanguage{english}
\markboth{Abstract}{Abstract}
\section*{\LARGE\rmfamily\bfseries\scshape{Abstract}}
%Through advances in web-information systems, an increasing number of services are offered online instead of local applications. 
Activities like text-editing, watching movies, or managing personal finances are all accomplished with web-based solutions nowadays. The providers need to ensure security and privacy of user data. To that end, passwords are still the most common authentication method on the web. They are inexpensive and easy to implement. Users are largely accustomed to this kind of authentication but passwords represent a considerable nuisance, because they are tedious to create, remember, and maintain. In many cases, usability issues turn into security problems, because users try to work around the challenges and create easily predictable credentials. Often, they reuse their passwords for many purposes, which aggravates the risk of identity theft. There have been numerous attempts to remove the root of the problem and replace passwords, e.g., through biometrics. However, no other authentication strategy can fully replace them, so passwords will probably stay a go-to authentication method for the foreseeable future.

Researchers and practitioners have thus aimed to improve users' situation in various ways. There are two main lines of research on helping users create both usable and secure passwords. On the one hand, password policies have a notable impact on password practices, because they enforce certain characteristics. However, enforcement reduces users' autonomy and often causes frustration if the requirements are poorly communicated or overly complex. On the other hand, user-centered designs have been proposed: Assistance and persuasion are typically more user-friendly but their influence is often limited. In this thesis, we explore potential reasons for the inefficacy of certain persuasion strategies. From the gained knowledge, we derive novel persuasive design elements to support users in password authentication.

The exploration of contextual factors in password practices is based on four projects that reveal both psychological aspects and real-world constraints. Here, we investigate how mental models of password strength and password managers can provide important pointers towards the design of persuasive interventions. Moreover, the associations between personality traits and password practices are evaluated in three user studies. A meticulous audit of real-world password policies shows the constraints for selection and reuse practices.

Based on the review of context factors, we then extend the design space of persuasive password support with three projects. We first depict the explicit and implicit user needs in password support. Second, we craft and evaluate a choice architecture that illustrates how a marketing phenomenon can provide new insights into the design of nudging strategies. Third, we tried to empower users to create memorable passwords with emojis. The results show the challenges and potentials of emoji-passwords on different platforms. 

Finally, the thesis presents a framework for the persuasive design of password support. It aims to structure the required activities during the entire process. This enables researchers and practitioners to craft novel systems that go beyond traditional paradigms, which is illustrated by a design exercise.% We show how it can be applied in practice and create a concept for a novel password management tool. % We conclude with specific directions for the future.

%
%The dissertation reports on empirical results about the context factors that shape password practices. Here, we highlight the role of understanding the mental models that users have established around password security. Moreover, we empirically analyze explanatory variables that had not yet been considered in password support, e.g. personality traits. Finally, we present strategies to empower users to adopting stronger password practices. 
%
%The thesis breaks down the problem space of password authentication, offers new insights into password practices, and reports on several user studies to address ill-defined issues. In summary, the results indicate that a triangulated, holistic approach is the most promising direction. A framework and usage scenarios that incorporate multiple strategies is presented and serve as a toolbox for the design of password support systems. 
%%=======================================================================================================================
%\clearemptydoublepage
\clearpage
%=======================================================================================================================

\markboth{Zusammenfassung}{Zusammenfassung}
\section*{\LARGE\rmfamily\bfseries\scshape{Zusammenfassung}}
\selectlanguage{ngerman}
%=======================================================================================================================
Heutzutage ist es möglich, mit web-basierten Lösungen Texte zu editieren, Filme anzusehen, oder seine persönlichen Finanzen zu verwalten. Die Anbieter müssen hierbei Sicherheit und Vertraulichkeit von Nutzerdaten sicherstellen. Dazu sind Passwörter weiterhin die geläufigste Authentifizierungsmethode im Internet. Sie sind kostengünstig und einfach zu implementieren. NutzerInnen sind bereits im Umgang mit diesem Verfahren vertraut jedoch stellen Passwörter ein beträchtliches Ärgernis dar, weil sie mühsam zu erstellen, einzuprägen, und verwalten sind. Oft werden Usabilityfragen zu Sicherheitsproblemen, weil NutzerInnen Herausforderungen umschiffen und sich einfach zu erratende Zugangsdaten ausdenken. Daneben verwenden sie Passwörter für viele Zwecke wieder, was das Risiko eines Identitätsdiebstals weiter erhöht. Es gibt zahlreiche Versuche die Wurzel des Problems zu beseitigen und Passwörter zu ersetzen, z.B. mit Biometrie. Jedoch kann bisher kein anderes Verfahren sie vollkommen ersetzen, so dass Passwörter wohl für absehbare Zeit die Hauptauthentifizierungsmethode bleiben werden. 

ExpertInnen aus Forschung und Industrie haben sich deshalb zum Ziel gefasst, die Situation der NutzerInnen auf verschiedene Wege zu verbessern. Es existieren zwei Forschungsstränge darüber wie man NutzerInnen bei der Erstellung von sicheren und benutzbaren Passwörtern helfen kann. Auf der einen Seite haben Regeln bei der Passworterstellung deutliche Auswirkungen auf Passwortpraktiken, weil sie bestimmte Charakteristiken durchsetzen. Jedoch reduziert diese Durchsetzung die Autonomie der NutzerInnen und verursacht Frustration, wenn die Anforderungen schlecht kommuniziert oder übermäßig komplex sind. Auf der anderen Seite stehen nutzerzentrierte Designs: Hilfestellung und Überzeugungsarbeit sind typischerweise nutzerfreundlicher wobei ihr Einfluss begrenzt ist. In dieser Arbeit erkunden wir die potenziellen Gründe für die Ineffektivität bestimmter Überzeugungsstrategien. Von dem hierbei gewonnenen Wissen leiten wir neue persuasive Designelemente für Hilfestellung bei der Passwortauthentifizierung ab. 

Die Exploration von Kontextfaktoren im Umgang mit Passwörtern basiert auf vier Projekten, die sowohl psychologische Aspekte als auch Einschränkungen in der Praxis aufdecken. Hierbei untersuchen wir inwiefern Mental Modelle von Passwortstärke und -managern wichtige Hinweise auf das Design von persuasiven Interventionen liefern. Darüber hinaus werden die Zusammenhänge zwischen Persönlichkeitsmerkmalen und Passwortpraktiken in drei Nutzerstudien untersucht. Eine gründliche Überprüfung von Passwortregeln in der Praxis zeigt die Einschränkungen für Passwortselektion und -wiederverwendung. 

Basierend auf der Durchleuchtung der Kontextfaktoren erweitern wir hierauf den Design-Raum von persuasiver Passworthilfestellung mit drei Projekten. Zuerst schildern wir die expliziten und impliziten Bedürfnisse in punkto Hilfestellung. Daraufhin erstellen und evaluieren wir eine Entscheidungsarchitektur, welche veranschaulicht wie ein Marketingphänomen neue Einsichten in das Design von Nudging-Strategien liefern kann. Im Schlussgang versuchen wir NutzerInnen dabei zu stärken, gut merkbare Passwörter mit Hilfe von Emojis zu erstellen. Die Ergebnisse zeigen die Herausforderungen und Potenziale von Emoji-Passwörtern auf verschiedenen Plattformen. 

Zuletzt präsentiert diese Arbeit ein Rahmenkonzept für das persuasive Design von Passworthilfestellungen. Es soll die benötigten Aktivitäten während des gesamten Prozesses strukturieren. Dies erlaubt ExpertInnen neuartige Systeme zu entwickeln, die über traditionelle Ansätze hinausgehen, was durch eine Designstudie veranschaulicht wird. 
%=======================================================================================================================
\selectlanguage{english}