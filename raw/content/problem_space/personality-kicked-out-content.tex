% SET THE SCENE

% 1. Passwords are a burden for all of us, and it is getting worse. Users have a portfolio, but what is it made of and why?
As passwords remain the main authentication method on the web, users face an ever increasing challenge to select and maintain passwords. Users' strategies to mitigate this burden are often predictable, such that they use easily guessable passwords and re-use them on more than one service. If required to change a specific password, e.g. after a database breach occurred, users often simply pick another from their own password portfolio \cite{Bonneau2015ImperfectAuthentication, Florencio2014PasswordPortfoliosFiniteUser, Stobert2014PasswordLifeCycle}. How users decide among alternative passwords is a highly individual process, and it is an open question how individual predisposition of the psyche influences the decision making in these situations.
%Although the motivation for acting this way is well understood, we do not know how much of this kind of predictable behavior can be explained by looking at the user's psyche.  
%Although we have indication that passwords are maintained for a long time \cite{VonZezschwitz2013SurvivalShortest}, we know little about the influencing factors on how the actual portfolio comes to be. 


% 2. demographics and education do not explain everything about privacy attitudes, personality and GDMS are valuable.
Apart from demographic and educational factors that evidently affect password creation and coping strategies \cite{Mazurek2013Measuring}, researchers in usable security and privacy have started considering psychological variables to improve explanations for predictable user behavior. Two prominent examples for such variables are the Big-Five personality traits \cite{Goldberg1990BigFive} and the general decision making style (GDMS) \cite{Scott1995GDMS}. The Big Five model tries to characterize a person's attitudes and behavior with five distinct traits, namely openness, conscientiousness, extraversion, agreeableness and neuroticism. The GDMS model describes how people decide by classifying decision-making into rational, intuitive, dependent, avoidant and spontaneous styles. How users form privacy attitudes has been explored extensively \cite{Acquisti2005PrivacyRationality,Korff2014TooMuchChoice,Spiekermann2001EPrivacyPreferences,Woodruff2014PrivacyFundamentalist}. Egelman and Peer, however, were one of the few who suggested taking the Big-Five traits and decision making style into account to explain the preconditions for privacy attitudes \cite{Egelman2015AverageUser}. They found out that someone's decision making style is more predictive than personality traits regarding general privacy attitudes. %In this paper, we investigate if this is also true for another crucial type of privacy behavior, which is to judge the strength of passwords. 

% 3. creating a password is also a decision making process. 
% 4. password strength perceptions have been explored for average users, but where are the individual differences? 
%    --> why are the individual differences necessary?
%    --> why do we think that the differences could result from personality or decision making style?

Decision-making tasks can also be found in password selection: When a user is required to create a password, he or she can either come up with a new one by combining letters, digits and symbols or pick an old password from his or her personal portfolio that usually consists of around five distinct passwords \cite{Florencio2007LargeScaleStudyPasswordHabits}. In any case, the outcome of this decision-making process depends on evaluating ease of typing, memorability, or the perceived strength. When users judge the strength and memorability of passwords, they do this in predictable ways \cite{Ur2016PerceptionsPassword}. They often base their judgment on the variety of characters forming the password. Although there are indeed certain themes in the way users perceive strength, the perceptions are not shared unanimously among all users. Examining what causes the discrepancies is important to help users better understand password strength and to design usable support systems. We hypothesize that the perception of password strength may be influenced by variables like personality traits or decision making style. 

In this paper, we investigate the associations between the Big-Five personality traits, general decision making style and users' perceptions of password strength. In an online study with 100 participants, we administered psychometric tests and collected subjective strength ratings of diverse passwords. We examined the associations between the psychometrics and password strength ratings. Among our key findings, we observed that respondents with high openness scores were more skeptical than the rest about the strength of passwords shown in the study. Moreover, those who score high on conscientiousness were more likely to compare two given passwords by the variety of characters. Rational decision makers performed significantly worse in identifying the stronger of two given passwords.



\section{Background and Related Work}
\subsection{Password Strength Metrics}
%First of all, strong passwords are desirable to protect personal information, 
%TODO motivation: Why do we need strong passwords? Do they achieve anything?
%TODO wir brauchen noch den Hinweis, dass z.B. Substitutionen oder erratbare Modifikationen nicht viel bringen. 
%\subsubsection{Metrics}
Finding an objective and reliable measure for the strength of a given password is difficult. The NIST-entropy of a password is a standard and commonly used measure. It reports the degree of randomness of the characters inside a password \cite[ Appendix A therein]{Burr2011NIST}. However, as more advanced threat models emerged, more realistic measures were proposed. In offline-attack scenarios, the entire database containing the passwords as hashes is obtained by the attackers, which happens sporadically even with highly frequented services like LinkedIn \cite{Florencio2007DoStrongWebPasswords,Scott2016ProtectingLinkedIn,Shay2016DesigningPasswordPolicies}. This means attackers can try millions of times to guess passwords and their efforts are only limited by time and computing power. Multiple researchers proposed that the number of guesses required to crack a password is often a more accurate metric for strength \cite{Kelley20012GuessAgain, Shay2016DesigningPasswordPolicies, Weir2010MetricsPolicies}. Carnegie Mellon University has established a Password Guessing Service (PGS) that allows uploading a list of passwords and receive success rates from various cracking approaches \cite{Ur2015MeasuringRealWorldAccuracies}. To use the service, however, the passwords need to be collected and uploaded in clear-text. This is not always possible in password studies, mostly for ethical reasons. There are other means to estimate the required number of guessing attempts. For example, the zxcvbn algorithm shows high accuracy up to one million guesses, which is a realistic cut-off threshold for online attacks \cite{Wheeler2016zxcvbn}. It can be implemented as a lightweight script and is easy to include in pro-active password checks.


\subsection{Human Factors in Password Strength}
Leaked data from real-world accounts has repeatedly shown that user selected passwords are often predictable \cite{Bonneau2012ScienceOfGuessing}. Given the freedom to select any password, many people opt for simple, short, memorable words or numbers that are easy to type. Because such passwords are vulnerable to informed guessing attacks, service providers try to prevent them by introducing a set of requirements to reduce the risk of account hijacking. However, such composition policies are not implemented identically on all web services \cite{Wang2015EmperorsPolicies}. Often, when users create an account, they re-use a password from elsewhere \cite{Das2014TangledWeb}, which is sometimes prevented if policies differ in requirements. Users then often modify the password until the requirements are met \cite{Inglesant2010TrueCostOfUnusablePolicies,Komanduri2011OfPasswordsAndPeople}. The resulting passwords do not necessarily gain strength, if they are only appended by digits or symbols at predictable positions \cite{Weir2010MetricsPolicies}. Thus, balancing the demands in terms of usability and security of a password policy is challenging and has been under constant research in the past few years \cite{Melicher2016UsabilityMobileTextPasswords,
	Shay2016DesigningPasswordPolicies, 
	Shay2014CanLongPasswordsBeSecureAndUsable,
	Wang2015EmperorsPolicies}. One important aspect of password policies that possibly affects how users evaluate password strength is their educational effect. Users are often exposed to requirements that do not necessarily lead to stronger passwords. The length of the password is often more crucial for the strength than character diversity. For instance, when policies require three different character classes with minimum length twelve (3class12), user-selected passwords are often more guessable than those created with a simple length requirement of 16 characters (basic16) \cite{Shay2014CanLongPasswordsBeSecureAndUsable}. At the same time, users are being told that character variety is necessary to form strong passwords \cite{Ur2012HowDoesYourPasswordMeasureUp}. The long-term consequences are that users sometimes have a suboptimal perception of the factors that add to the objectively measurable strength of a password \cite{Ur2016PerceptionsPassword}. %To this point, psychological factors affecting the susceptibility to educational effects of password policies are underexplored, which we try to   
%HH: Das könnte man mit einem Beispiel unterstützen. Ich denke, das klassische Beispiel ist, dass die Länge wichtiger ist als ob eine Zahl vorkommt.



Beside the constraints dictated by composition policies, many users have developed coping strategies for handling authentication tasks \cite{Stobert2014PasswordLifeCycle}. For example, the value of an account is decisive whether users pick a strong or weak password from their portfolio. Stobert and Biddle argue that this process is deliberate and even IT experts are prone to choose weak passwords for accounts that they do not deem worthy to protect  \cite{Stobert2015ExpertPassword}. Flor\^{e}ncio and Herley argue that this behavior is rational from an economics and efficiency perspective \cite{Florencio2014PasswordPortfoliosFiniteUser}. Still, if users receive security advice by trusted peers, they might reconsider behaviors like password re-use \cite{Das2014EffectSocialInfluenceSecuritySensitivity}.


It was also shown that nudging is effective to make users try and increase password strength. Presentation effects of the nudges can play a role. For example, the design of a password meter can impact how much effort users put into the creation of the password \cite{Ur2012HowDoesYourPasswordMeasureUp}. If shown suggestions, participants in a controlled online study created stronger passwords depending on the kind and number of suggestions \cite{Seitz2016SuggestionsDecoy}. Another intrinsic factor is the degree and type of exposure to social influence in the form of stories or advice. Das et al. showed that such advice affects the intrinsic motivation of acting more secure on the web \cite{Das2014EffectSocialInfluenceSecuritySensitivity}. 

The participants' meta password characteristics are important to find out if past actions influence strength ratings. The participants reported to use passwords that were nine characters long ($Md=9,~SD=3$), which is consistent with the finding that most real-world policies require a minimum of eight or nine characters \cite{Wang2015EmperorsPolicies}. The medians for other password characteristics were 1 uppercase letter ($SD=1$), 2 digits ($SD=3$), 0 special characters ($SD=1$) and 1 dictionary word ($SD=1$).

%\subsubsection{Behavior and Strength Rating}
Before we analyzed the influence of personality traits on the ratings, checked associations between meta passwords and password ratings. We performed linear regression analyses as shown in Table \ref{tbl:personalpw_regression}. The regression models reveal that ratings were mostly associated with the length and usage of uppercase letters inside one's own password. Model fit was $R^2_{adj}=0.12$ for overall rating. Particularly in the second row of Table \ref{tbl:personalpw_regression}, we can see that participants, who report to include more uppercase letters in their meta passwords, were more likely to give lower overall strength ratings (cf. Figure \ref{fig:MetaPW-ViolinPlot}). Regarding the comparison task, we found that the meta passwords were only very weakly associated with the outcome (mean $R^2_{adj}=0.03$). Hence, we do not present detailed statistics. 


% Table generated by Excel2LaTeX from sheet 'Correlation Matrix SLUDS'

\begin{table}%[htbp]
  \centering
  \small
  \caption{Regression analysis with participants' proxy password characteristics as independent variables and ratings for password groups. Especially the use of uppercase letters is negatively correlated with the ratings. Numbers in bold indicate statistical significance ($p<0.05$)}
  \resizebox{\linewidth}{!}{
\begin{tabular}{l|r|rr|rrr|rrr}
    \multicolumn{1}{c|}{\multirow{2}[1]{*}{\textbf{Proxy PW Characteristic}}} & \multicolumn{9}{c}{\textbf{Password Ratings}} \\
      & \multicolumn{1}{c|}{Overall} & \multicolumn{1}{l}{Long} & \multicolumn{1}{l|}{Short} & \multicolumn{1}{l}{Weak} & \multicolumn{1}{l}{Medium} & \multicolumn{1}{l|}{Strong} & \multicolumn{1}{l}{Special} & \multicolumn{1}{l}{Digits} & \multicolumn{1}{l}{Uppercase } \\
    \midrule
    \rowcolor[rgb]{ .922,  .945,  .871} Length & -0.13 &   & -0.12 & \textbf{-0.20} &   &   & \textbf{-0.21} &   & -0.13 \\
    \rowcolor[rgb]{ .922,  .945,  .871} Uppercase Letters & \textbf{-0.33} & \textbf{-0.31} & \textbf{-0.26} &   & \textbf{-0.37} & \textbf{-0.25} & -0.18 & \textbf{-0.41} & \textbf{-0.30} \\
    \rowcolor[rgb]{ .922,  .945,  .871} Digits &   &   &   &   &   &   &   &   &  \\
    \rowcolor[rgb]{ .922,  .945,  .871} Symbols &   &   &   &   &   &   &   & 0.12 &  \\
    IT Background &   &   &   & -0.10 &   &   &   &   &  \\
    Education &   &   & 0.11 &   &   & -0.12 &   &   &  \\
    Gender &   &   &   &   &   &   & 0.20 &   & 0.10 \\
    \midrule
    \textbf{F} & \textbf{7.64} & \textbf{10.29} & \textbf{3.74} & \textbf{3.27} & \textbf{15.33} & \textbf{3.89} & \textbf{3.40} & \textbf{9.45} & \textbf{4.13} \\
    \textbf{p} & 0.00 & 0.00 & 0.01 & 0.04 & 0.00 & 0.02 & 0.02 & 0.00 & 0.01 \\
    \textbf{Adj. $R^2$} & \cellcolor[rgb]{ .675,  .831,  .502} 0.12 & \cellcolor[rgb]{ 1,  .922,  .518} 0.09 & \cellcolor[rgb]{ .992,  .812,  .494} 0.08 & \cellcolor[rgb]{ .973,  .412,  .42} 0.04 & \cellcolor[rgb]{ .592,  .804,  .494} 0.13 & \cellcolor[rgb]{ .976,  .545,  .443} 0.06 & \cellcolor[rgb]{ .988,  .702,  .475} 0.07 & \cellcolor[rgb]{ .388,  .745,  .482} 0.15 & \cellcolor[rgb]{ .992,  .922,  .518} 0.09 \\
    \bottomrule
    \bottomrule
    \end{tabular}%
        }
%  \resizebox{\linewidth}{!}
%{
%
%
%}
  
  \label{tbl:personalpw_regression}%
\end{table}%
