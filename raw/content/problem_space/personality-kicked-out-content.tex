% SET THE SCENE

% 1. Passwords are a burden for all of us, and it is getting worse. Users have a portfolio, but what is it made of and why?
As passwords remain the main authentication method on the web, users face an ever increasing challenge to select and maintain passwords. Users' strategies to mitigate this burden are often predictable, such that they use easily guessable passwords and re-use them on more than one service. If required to change a specific password, e.g. after a database breach occurred, users often simply pick another from their own password portfolio \cite{Bonneau2015ImperfectAuthentication, Florencio2014PasswordPortfoliosFiniteUser, Stobert2014PasswordLifeCycle}. How users decide among alternative passwords is a highly individual process, and it is an open question how individual predisposition of the psyche influences the decision making in these situations.
%Although the motivation for acting this way is well understood, we do not know how much of this kind of predictable behavior can be explained by looking at the user's psyche.  
%Although we have indication that passwords are maintained for a long time \cite{VonZezschwitz2013SurvivalShortest}, we know little about the influencing factors on how the actual portfolio comes to be. 


% 2. demographics and education do not explain everything about privacy attitudes, personality and GDMS are valuable.
Apart from demographic and educational factors that evidently affect password creation and coping strategies \cite{Mazurek2013Measuring}, researchers in usable security and privacy have started considering psychological variables to improve explanations for predictable user behavior. Two prominent examples for such variables are the Big-Five personality traits \cite{Goldberg1990BigFive} and the general decision making style (GDMS) \cite{Scott1995GDMS}. The Big Five model tries to characterize a person's attitudes and behavior with five distinct traits, namely openness, conscientiousness, extraversion, agreeableness and neuroticism. The GDMS model describes how people decide by classifying decision-making into rational, intuitive, dependent, avoidant and spontaneous styles. How users form privacy attitudes has been explored extensively \cite{Acquisti2005PrivacyRationality,Korff2014TooMuchChoice,Spiekermann2001EPrivacyPreferences,Woodruff2014PrivacyFundamentalist}. Egelman and Peer, however, were one of the few who suggested taking the Big-Five traits and decision making style into account to explain the preconditions for privacy attitudes \cite{Egelman2015AverageUser}. They found out that someone's decision making style is more predictive than personality traits regarding general privacy attitudes. %In this paper, we investigate if this is also true for another crucial type of privacy behavior, which is to judge the strength of passwords. 

% 3. creating a password is also a decision making process. 
% 4. password strength perceptions have been explored for average users, but where are the individual differences? 
%    --> why are the individual differences necessary?
%    --> why do we think that the differences could result from personality or decision making style?

Decision-making tasks can also be found in password selection: When a user is required to create a password, he or she can either come up with a new one by combining letters, digits and symbols or pick an old password from his or her personal portfolio that usually consists of around five distinct passwords \cite{Florencio2007LargeScaleStudyPasswordHabits}. In any case, the outcome of this decision-making process depends on evaluating ease of typing, memorability, or the perceived strength. When users judge the strength and memorability of passwords, they do this in predictable ways \cite{Ur2016PerceptionsPassword}. They often base their judgment on the variety of characters forming the password. Although there are indeed certain themes in the way users perceive strength, the perceptions are not shared unanimously among all users. Examining what causes the discrepancies is important to help users better understand password strength and to design usable support systems. We hypothesize that the perception of password strength may be influenced by variables like personality traits or decision making style. 

In this paper, we investigate the associations between the Big-Five personality traits, general decision making style and users' perceptions of password strength. In an online study with 100 participants, we administered psychometric tests and collected subjective strength ratings of diverse passwords. We examined the associations between the psychometrics and password strength ratings. Among our key findings, we observed that respondents with high openness scores were more skeptical than the rest about the strength of passwords shown in the study. Moreover, those who score high on conscientiousness were more likely to compare two given passwords by the variety of characters. Rational decision makers performed significantly worse in identifying the stronger of two given passwords.

