\chapter[Mental Models of Password Strength]{Mental Models of Password Strength}\label{chap:pasdjo}

\section{Background and Context}

\begin{itemize}
\item we expect users to create strong passwords
\item what if they don't know what adds to password strength?
\item Ur et al. \cite{Ur2016PerceptionsPassword}
\item  Shay \etal also used perceived strength as metric \cite{Shay2015SpoonfulOfSugar}
\end{itemize}


\subsection{Research Objectives}
\begin{itemize}
	\item find more evidence of password misconceptions
	\item educate users at the same time? --> problematic because we didn't really focus on it and left it to the users to figure out how it works.
\end{itemize}

\subsubsection{Questions}
\begin{itemize}
\item[1] 
\end{itemize}

\section{Related Work}

\section{Approach: PASDJO - The Password Game}


\section{Log Analysis}

\subsection{Sample}
\subsection{Results}


\section{Discussion}

\section{Limitations}
\begin{itemize}
	\item were already addressed in new versions of the game. 
	\item used as an activity at Google	
\end{itemize}


\section{Summary}
\begin{itemize}
	\item users performed better than anticipated
	\item misconceptions mainly involve passphrases and mangled passwords
	\item the game was updated after the first results were in and is still deployed. Google even uses it internally. 
	\item the source code is available on GitHub.
\end{itemize}