\chapter[Password Policies and Reuse]{Password Policies and Reuse}\label{chap:policies_reuse}
% okay, so people know what makes strong passwords okayish
In the previous chapter, we reached the conclusion that users' mental models of password strength are fairly accurate with small exceptions. We can thus expect that users at least try to create strong passwords if they deem the account worth protecting. 
% but they underestimate the threat of password reuse. 
However, this is only one side of the medal: password reuse is rampant. Coping with the high number of passwords by reusing them is common, but hard to defend against - in part because some degree of reuse is necessary \cite{Florencio2014PasswordPortfoliosFiniteUser, ZhangKennedy2016RevisitingPasswordRules}.  At the same time, password reuse might expose users to an even greater risk than weak passwords. Password reuse renders the security advantages of picking a very strong password void. In case an attacker obtains a user's plain text password, they gain access to all accounts that share this strong password. Studies have shown that users tend to underestimate the risks generated by password reuse \ar.

% policies fail to induce strong passwords, do they prevent reuse?
As explained in Section \ref{sec:rw:policies}, password composition policies are one of the interventions targeted at weak passwords. However, in many cases users fulfill requirements in predictable ways; the primary goal is thus missed. So, if password policies do not always help password strength, they might still prevent password reuse: if they were heterogeneous across different web sites with mutually exclusive requirements, users cannot reuse passwords like they would naturally do. In this chapter, we investigate password policies of one hundred of the most-visited web-services in Germany and try to find how well their differences prevent password reuse. Some results of this investigation have been previously published together with Manuel Hartman, Jakob Pfab, and Samuel Souque \cite{Seitz2017PoliciesReuse}. In this chapter we shed light on the findings and discuss them in the context of supporting password authentication. 

\section{Background and Context}

% reuse issues
\cite{Das2014TangledWeb}
\cite{Jaeger2016AnalysisOfLeaks}
\cite{Stobert2014PasswordLifeCycle}
\cite{Wash2016UnderstandingPasswordChoices}

``diversify'' passwords \cite{Segreti2017AdaptivePolicies}

% Blacklists prevent reuse of leaked passwords. 
\cite{Habib2017Blacklists}

% you can't display the blacklisted words in the UI, it would be good to have a policy description
closely related: \cite{Wang2015EmperorsPolicies}

	It would be nice to have an easy way to create a repository with all password policies, but it isn't easy - at least it helps to agree on a standard language to define a password policy \cite{Steves2015PasswordPolicyLanguage}

It would be good to have a formal description of password policies in a standardized schema \cite{Horsch2016PasswordPolicyMarkup}



% summary: in an ideal world, policies would make users pick good passwords AND prevent direct reuse. 
\section{Method}

\section{Results}
\subsubsection{Policies are Mostly homogenous, with slight differences}
\subsubsection{Policies do not prevent password-reuse}

\section{Discussion and Implications}

\section{Limitations}
