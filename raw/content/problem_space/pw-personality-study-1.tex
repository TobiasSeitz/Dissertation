\section{Study 1: Policies}
% general motivation - why do we investigate this?
We start out with the exploration of psychological factors for the design of password policies. We were motivated by the fact that at this point, policies are a one-fits-all solution that evidently does not work in the same ways for all users: Shay \etal found a range of preferences for diffuzzzzzzhjtzzzzzzzzzzzzzzzzzzzzzzzzzzzzzzzzzzzzzzzzzzzzzzzzzzzzzzzzzzzzzzzzzzzzzzzzzzzzzzzzzzzzzzzzzzzzzzzzzzzzzzzzzzzzzzzzzzzzzzzzzzzzzzzzzzzzzzzzzzzzzzzzzzzzzzzzzzzzzzzzzzzzzzzzzzzzzzzzzzzzzzzzzzzzzzzzzzzzzzzzzzzzzzzzzzzzzzzzzzzzzzzzzzzzzzzzzzzzzz234			Z


GOALS: explore associations between big-five traits and password selection under different policies, both on usability and security metrics. Investigate the effects of using a non-traditional password policy based on emojis. user preference for one policy or another. explorative study so no p-values.

metrics: difficulty to create password, preference in real world
items: ``It was difficult to create a password that meets the requirements'', ``I found the password requirements bothersome'', ``It was easy to create a \textit{new} password''. (5-point scale)

statistics: ranking and regression

independent variables (predictors) personality traits scores on the five dimensions and control variables gender, age, IT proficiency.

\subsection{Method}
explorative study
BFI-K (Rammstedt) 21 items -- openness is coded with 5 items, the rest with 4. Five point scales. Scores summed up after keying.
repeatedly create passwords (within groups), order counterbalanced. emoji12 / 2word12 / 3class12
emojis entered with a picker (see emoji passwords chapter)
integrated into SosciSurvey survey tool (iframe)
deactiveate auto-completion
briefed not to participate through mobile phone. 

demographics first, privacy briefing, BFI-K, PW selection.
continuous scenario 1) Banking password 2) someone gained access to bank account so the password was reset 3) password has expired. 
(realistic threat principle Krol \etal \cite{Krol2016ExperimentDesign})

incentive: raffle of shoppig vouchers, emails stored separately, no correlation between the responses and email addresses.

demography.
N before dropout and sanity checks 222, afterwards 164. age \average 24.7 (SD=5.5). 79 female, 83 male, 2 preferred not to answer. 65 people had had formal training in computer science or information technology.

40\% reuse, 32\% reuse with modifications or algorithm, 17\% create new one, 11\% password manager or generator (very much like the data from other sources!). 53\% memorize, 10\% write down cues, 16\% write them down on paper, 21\% use electronic aides.

limitations:
- homogeneous sample: most between 20 and 28, academics. but personality traits are spread out across this population normally.
- no memorability part
- within group design
- no mobiles, but they have an influence on password selection \ref{sec:rw:user-behavior} from qual feedback we could tell that some participants used mobiles and this might have had an effect on pw selection (maybe rerun analyses without these participants?)
- we started out with emoji16, but made the switch to emoji12 (after 43 participants), because emoji16 received the most negative feedback, but it was mostly due to the length. data removed for ranking (1-61) and difficulty to create (1-43). 

statistical analyses::
additive regression model (AM) = extension of linear regression, but is better fit for non-linear associations.
mgcv R package


\subsection{Results}
medium associations between extraversion, agreeableness, and neuroticism (basically the other three dimensions that were not useful before), but control variables are associated to a larger degree.

emoji policy similar results as traditional policies, i.e. it is wortwhile to experiment with it

\paragraph{Descriptives and Hypothesis Tests}
creation difficulty: Repeated measures linear mix model ANOVA. random intercept to account for the fact that people generally rate policy usability low. baseline (any could be chosen) emoji 12.

emoji \average 7.89 (in [3;15] range), 2word12 + 0.12, 3class12 - 0.70 (easier)
serial positioning effects do make a difference, but it also appears trivial in the overall sample.
summary: no big difference between policies and ranking. (this is what makes the remainder more exciting, because a closer look at the data brings out how the numbers come to be)


\paragraph{Creation Difficulty}
% gender
Female participants reported more difficulties for emoji12 ($\beta=0.91$) and 2word12 ($\beta=1.35$). 
for 3class12 correlation was smaller ($\beta=-0.46$) and pointed in the opposite direction.
% it background
IT background revealed interesting effects too: emoji12 ($\beta=-0.10$) trivial, 3class12 ($\beta=0.21$) weak, 2word12 ($\beta=0.61$) medium. it looks as though people with higher IT knowledge struggle with a word-based policy, potentially because it is much less common in the wild. 
% order of the policy
If you factor in the order of the policy, the results look clearer. if emoji12 and 3class12 were shown in position 2, creation was more reported to be more difficult emoij12-2 ($\beta=0.52$), 3class12-2 ($\beta=1.19$) 
% age
we observed non-linear associations between age and creation difficulty, but the small age range forbids us to make conclusive inferences from our data.

%emoji personality
mostly non-linear associations for the emoij12 policy
neuroticism weak linear association emoji12 ($\beta=-0.22$), i.e. it was slightly easier for people with higher neuroticism scores (this is a great result, albeit weak. neurotic people are by defintion more \textit{emotional}, and emojis seem to cater to this trait.)
agreeableness weak non linear, and interesting curve (up and down) for emoji12
openness weak non linear, but too little data to conclusively decide. 
% 2word12 personality
effects not as clear as for the emoji policy. highly open people report slightly less difficulty to create a 2word12 password.

% 3class12
linear associations are negligible for extraversion, conscientiousness, and neuroticism. 
non-linear associations for agreeableness and openness. at value around 15 (from 20) the difficulty seems to increase (hard to interpret this finding). People with agreeableness scores of 18 and greater find it more difficult to create a 3class12 password by up to 1.5 points. Openness is inversely correlated here. the more open the participant was, the more likely they found it easy to create a 3class12 password. 


\paragraph{Ranking}
logistic additive regression %https://web.stanford.edu/~hastie/Papers/AdditiveLogisticRegression/alr.pdf
did policy X land on top spot ? yes = 1, no = 1. thus, no encompasses the two other options.
Logit model. see timo's work to better understand what's going on. 

\begin{table}
	\centering
	\caption{\label{tab:pws_pers:distribution-binary-ranks} Distribution of binary rankings of the three available policies. Evidently, 3class12 was ranked best in most cases. }
	\begin{tabular}{llll}
		~ 			& emoji12	& 2word12 	& 3class12 \\ \hline\hline
		1st rank	& 18		& 12		& 67 \\
		other rank 	& 80 		& 85 		& 31 \\ \hline
		n			& 98		& 98		& 98		 
	\end{tabular}
\end{table}


% predictors: demography and structure
strong effects. the probability of a non-IT person putting emoji12 at the top is $exp(\beta–{IT-0}=9.87)$, so around ten times higher. only one IT-person ranked emoji12 first. (this fits to the finding that non-IT people found it easier to create an emoji12 password).
% order of policies 
the order in which policies were displayed also produced a notable effect. If emoji12 is shown in 2nd $exp(\beta–{emoji-pos2}=0.27)$ or 3rd position $exp(\beta–{emoij-pos3}=0.76)$, the likelihood to rate it the best policy slightly decreases. Other positionings, as well as age-related preferences, were inconclusive.

% personality
generally weak associations.
extraversion, weak linear effect regarding 2word12. likelihood to put 2word12 first decreases with higher extraversion scores $exp(\beta–{E-1}=0.82)$. for agreeableness both linear and non-linear effects are visible; higher scores increase chances to vote emoji12 first $exp(\beta–{A-1}=1.28)$. openness or conscientiousness have negligible, random associations. for neuroticism we lack data in edge cases (low or high trait scores) to draw conclusions. %there are more results in the BA, but I do not really see the point of reporting all the details of inconclusive results


\paragraph{Password selection}

\todo{create a emoiji selection histogram}. In essence, some emoji were strongly preferred, e.g. the red heart and the ``pouting face''. analyze: what were the usability ratings/rankings of those who picked the heart vs. the pouting face, i.e. are emojis with a positive connotation used by people who are happy that they can select an emoji password? I guess there are significant effects here. 

\subsection{Lessons Learned}
It could have been useful to include a recall test à la ``which passwords do you still remember after filling out the personality questionnaire?'' 

\subsection{Discussion}
%TODO re-read the discussion section, because the findings are a bit higher level and more understandable
Most interesting finding: Non-IT people significantly less likely to put emoji policy first. s

emoji (negatively) associated with emotional stability, i.e. less stable people who are more emotional in either direction were more positive towards emojis.

non linear associations probably explicable by other variables, e.g. intercorrelation -- 
\todo{discuss the personality traits and associations, try to explain them.}

demographics play a role - (careful now, cowboy): policies could be tailored to gender. but i'm kind of reluctant to put this argument forward. at least IT background could be considered a factor for the policy. a personalized policy would make it more difficult for horizontal attacks because the policy is unpredictable.

mutually exclusive effects for ranking are clear because of the binary nature.

wild idea: if positioning influences the favorability of policies, one could leverage this. like anchoring and decoy (chose your own policy of the three). commitment and consistency as well (commit to your policy, create a strong password to be consistent with your commitment.)