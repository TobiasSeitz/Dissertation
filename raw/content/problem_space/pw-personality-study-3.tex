\section{Study 3: Password Selection}
% ALINE
% GOALS
As a final step in our ``password personality'' exploration, we ran an online survey. Having investigated preferences for policies and the perception of passwords, the main goal of the third study was to evaluate potential associations between personality and password \textit{selection}. To overcome some of the limitations of the previous study, we hoped to increase the sample size and reduce the number of items during the study. Moreover, further answers about the participants' explanations and motivations were considered to better understand the weight of personality factors. We determined the following research questions:
\begin{itemize}
	\item Are there correlations between password features (topology) and personality traits?
	\item Does personality influence password management behavior?
\end{itemize}

\subsection{Procedure and Tools}
The study was designed to take no more than ten minutes. The briefing page informed participants about the purpose of the study and data disclosure policies. After acknowledging the conditions of participation, respondents were asked to create a password. To boost ecological validity, we provided a fictitious but realistic scenario \cite{Komanduri2011OfPasswordsAndPeople}. The task was to come up with a new password for a new email account that they were going to use as their main address. Further, the instruction pointed out that the incentive would only be paid of if the participants chose a password they could recall later on. A \textit{basic8} policy was enforced, as it is one of the most representative policies in the wild (see chapter \ref{chap:policies_reuse}). This loose policy would also allow for both very complex and rather simple passwords, which could be associated with personality traits. Having successfully confirmed the password, respondents were taken to a questionnaire about demographics, just like in the first two studies. 

Next, participants completed the BFI-K questionnaire consisting of 21 items that have to be rated on a 5-point scale. We opted not to use the 50-item inventory for the sake of saving time. We added an item that served as an attention check. It asked to respond to this item with ``disagree''. Failure to follow this instruction allowed us to drop the response from the dataset. The resulting 22 items were shuffled to avoid sequence effects. 

Afterwards, we surveyed respondents about their password management behaviors and preferences. We used multiple-choice and open responses to collect qualitative, self-reported data. For instance, we wanted to know how they cope with multiple accounts or how they reuse passwords. The survey concluded with a recall task, where participants provided their initially chosen password. They could try as often as they liked, and the number of attempts were recorded. In case they were unable to recall their password, they could proceed anyhow and take part in the lottery. If they chose to provide an email address in the final step, this data was stored separately from the questionnaire data to avoid privacy issues. 

\subsection{Recruitment and Sample}

\subsection{Results}


\subsection{Learnings}