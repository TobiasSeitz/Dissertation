\chapter[Summary]{Summary}\label{chap:summary}
This thesis analyzed a multitude of aspects that need to be considered when we try to support users in password authentication with persuasive design. Hereby, the following research questions were addressed:

\begin{itemize}
	\item[\textbf{RQ1}] What is the role of psychological factors and mental models for password selection and coping strategies?
	\item[\textbf{RQ2}] How can password authentication be simplified for users? 
	\item[\textbf{RQ3}] How can we design persuasive strategies to support users in any password-related tasks?
\end{itemize}

Since password authentication has been under investigation for several decades, we first delimited the landscape of related work in Part \ref{part:related_work}. This helped us partially answer all three research questions and identify open questions that had still been underexplored (see Chapter \ref{chap:rw:summary}). We then reported on a number of empirical experiments and explorative research studies to investigate both human factors and environmental constraints of password authentication. \textit{RQ1} was mainly investigated with online studies in the wild and with survey methods. We explored the mental models of password strength by innovating a research method to inexpensively collect data in the wild. Moreover, we conducted multiple surveys to explore associations between personality and attitudes and behaviors in authentication. We tried to answer \textit{RQ2} with mixed methods, both on the qualitative and the quantitative side. Here, we explored the needs users have in persuasive feedback with a survey and participatory design approach, and derived a solution based on the Decoy effect and emojis inside text-based passwords. Finally, to answer \textit{RQ3}, we discussed a framework to guide the design of persuasive strategies, which we then also applied to create a novel password-manager. The following sections discuss the central insights and show how they are connected.

\section{Central Contributions and Insights}
\subsubsection{Psychological Factors and Mental Models in Authentication}
% related work mostly _describes_ behaviors and coping strategies, but not the mental models behind them.
% real world password authentication often sees it as a small hurdle, but not worthy of in-depth ux design.
In Chapter \ref{chap:rw:user_perspective}, we found that most related work \textit{describes} coping strategies. Only sometimes the contributing factors like educational background, psychographics, or mental models that foster certain coping strategies are addressed. Few password alternatives, as presented in Chapter \ref{chap:rw:passwords}, are mature enough to allow investigation of human factors outside the lab, and there is even less insight into the mental models involved. 
%At the same time, real-world authentication systems also often neglects such human factors, which aggravates security issues.

% PASDJO
We filled this gap in several ways. First, we investigated the mental models of password strength, because these are believed to be highly influential on actual password choice. For these purposes, we innovated on research methods to understand latent password strength perceptions: \textit{PASDJO}, the password game, helped in showing that passphrases are often underestimated by users, although \gls{NIST} has started to propagate them in favor of complexity (see Chapter \ref{chap:pasdjo}). The long-standing belief that strong passwords must include a wide range of characters was clearly visible in the data collected during one year of public deployment. However, users were by and large capable of estimating password strength. This constitutes evidence that users are most likely aware of their actions when they select strong and weak passwords for different purposes. At the same time, password strength is often a secondary risk factor, if users reuse passwords too carelessly. 
% Policy Audit
To that end, we audited the composition policies of the 83 most visited web-sites in Germany (see Chapter \ref{chap:policies_reuse}). We were able to show that it is easy for users to reuse a single password on most sites, if it is nine or ten characters long and consists of lower- and uppercase letters and digits. Hence, this finding is another indicator that password policies have contributed to the mental model that character diversity is absolutely required to form a strong password -- and that reuse is less critical, because it is not prevented. 

% PERSONALITY
Real-world constraints like policies thus shape mental models and coping strategies. On the other hand, there is a spectrum of password coping strategies that cannot be explained by environmental factors alone. We hypothesized that a user's personality might play a role in password selection and coping behavior. In three studies (see Chapter \ref{chap:pws_and_personality}), we examined how personality might be associated with different password tasks. We found that personality was a weak, but non-negligible factor in predicting how different user groups deal with policies, perceive strength, or choose passwords. The data allowed us to create user segments in the form of personas that can be used in the development of authentication schemes and support strategies. 
% PWMs
As a side effect, we observed that people with a background in an IT-related field were more likely to adopt a password manager. Looking at the generally low adoption rates of such software, we explored how users perceive passwords managers in Chapter \ref{chap:mental_models_pwm}. We contributed the notion that users appreciate this kind of tool once they were first exposed to the technology, e.g. at work. If they had never used a password manager before, they were unable to anticipate how it might help them. We further advanced the description of mental models of password authentication by distinguishing separate mental spaces. Interestingly, these were currently not fully matched by real-world password managers. 

% summary
In summary, we have to respect different dimensions of factors that contribute to mental models. Environmental constraints like password policies probably have the largest influence. Much traditional advice on how to form ``good passwords'' has led to a skewed mental model of password strength. Second, professional and educational factors are associated in how well users deal with password tasks. Finally, personality also contributes to the shaping of mental models, but warrants further research in this direction. %Current user interfaces, e.g. web sites, do not seem to take mental models into account when they try to simplify the task. 

\subsubsection{Simplification Strategies}
% state of the world
Having explored mental models and other psychological factors in password authentication, we noted a discrepancy between user interfaces and how users make sense of them. To simplify authentication, researchers have tried numerous approaches. Most notably, strength feedback in the form of password meters and real-time feedback about policy fulfillment have been seen as the state-of-the-art to simplify password selection. These solutions assume that there is a policy in place and users struggle to meet them, which seems fair. 
% user need elicitation
On the other hand, the users' needs regarding the feedback were ill-defined and therefore, we filled this gap. Through a mixed methods approach (see Chapter \ref{chap:feedback_modalities}) we specified users' needs and found four central dimensions of password selection support: \textit{showing} current problems, \textit{explaining} the implications, \textit{helping} with improvement, and \textit{empowering} to become creative.

% DECOY
To address these needs and simplify password selection, we explored two persuasive strategies. The first was based on ``showing'' current problems and ``helping'' with improvement. We introduced a \textit{choice architecture} for password selection based on the Decoy effect (see Chapter \ref{chap:decoy}). Through an online experiment, we observed that the Decoy choice architecture did not influence participants as expected. However, displaying a passphrase and making its benefits more visible and easily comparable did result in stronger and longer passwords. Thus, we believe that this kind of combination of feedback and feedforward is the key to simplifying selection strategies for stronger passwords. Although it is not necessary to pick a strong password for every single account, for master-passwords of password managers, it is very recommendable to reduce guessability. The second strategy we explored was aimed at simplifying memorization of passwords, and empowering users to become creative in their selection. To that end, we evaluated the usability of using emojis inside text-based passwords in two study sessions (see Chapter \ref{chap:emojipasswords}). We created a prototype to enter emoji-passwords that allowed us to measure selection patterns and issues arising from fragmentation, i.e. different visual representations of the same set of emojis across platforms. The concept brought about the desire to create more memorable passwords than what is usually possible, thus the simplification approach went in the right direction. However, once participants faced trouble recalling their emoji password, fragmentation issues became evident and participants were reserved towards adopting emoji-passwords in the future. So, although the concept generated interest at first, usability troubles outweighed the benefits of the first impression. However, in one of the personality studies, we found that certain user groups are more inclined to adopt emoji-passwords than others, and it is very likely that they will do so in the near future, because some services like Twitter and Slack already support emoji-passwords. Therefore, the usability issues need to be addressed soon to avoid user frustration due to account lock-outs and inefficient input. Only then will emoji-passwords become a true \textit{simplification} strategy. In conclusion, the task of password creation can be simplified for users through careful tuning of the password policy, feedback, feedforward, and empowerment. 

\subsubsection{Guiding Persuasive Designs}
% not very different to the double diamond, but specificall tailored to password support systems.
% apply the process to demonstrate how design can be guided towards a real-solution.
% password manager.

% evlt noch weiter ausholen aus den bisherigen kapiteln was mitnehmen. 




%\section{Contributions} %TODO choose different name.
%give a walkthrough through all take-aways. that should do and create 1-2 pages. 
%
%\subsection{Theoretical Contributions}
%- framework for the design of password support strategies.
%
%\subsection{Methodological Contributions}
%- measuring passwords in the wild in an ethical way. (meta zxcvbn)
%- novel solution to measure password strength perception. (PASDJO)
%
%\subsection{Empirical Contributions}
%- the decoy effect and password suggestions
%- personality and passwords
%- mental models password managers
%
%\section{Implications}
%% how has this thesis advanced science? / the amount of the world's knowledge.
%\begin{itemize}
%\item Emoji authentication on the web - we're not there yet. 
%\end{itemize}

\subsubsection{Conclusion}
% so -- what did we learn? how has the world changed?
you need to know the psychological aspects if you try to derive persuasive strategies. there's no way around it. 

% very foundational, strategic, explorative character

\section{Limitations}

\subsection{Study Designs}

\subsection{Generalizability}

\subsection{Real-World Measurements}

\subsection{Nudging Ethics and Risks}


%\section{Classification and Dimensions}
%here we talk about in which way the projects contributed to the Framework? and in which way


\section{Lessons Learned}
What would I do differently, if I had to do it all over again?



