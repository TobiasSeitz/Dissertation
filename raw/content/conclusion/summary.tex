\chapter[Summary]{Summary}\label{chap:summary}
This thesis analyzed a multitude of aspects that need to be considered when we try to support users in password authentication with persuasive design. Hereby, the following research questions were addressed:

\begin{itemize}
	\item[\textbf{RQ1}] What is the role of psychological factors and mental models for password selection and coping strategies?
	\item[\textbf{RQ2}] How can password authentication be simplified for users? 
	\item[\textbf{RQ3}] How can we design persuasive strategies to support users in any password-related tasks?
\end{itemize}

Since password authentication has been under investigation for several decades, we first delimited the landscape of related work in Part \ref{part:related_work}. This helped us find pointers to all three research questions and identify open topics that had still been underexplored (see Chapter \ref{chap:rw:summary}). In Parts \ref{part:problem_space} and \ref{part:design_space}, we reported on a number of empirical experiments and explorative research studies to investigate both human factors and environmental constraints of password authentication. \textit{RQ1} was mainly investigated with online studies in the wild and with survey methods. We explored the mental models of password strength by innovating a research method to inexpensively collect data in the wild. Moreover, we conducted multiple surveys to explore associations between personality and attitudes and behaviors in authentication. We tried to answer \textit{RQ2} with mixed methods, both on the qualitative and the quantitative side. Here, we explored the needs users have in persuasive feedback with a survey and participatory design approach, and derived a solution based on the Decoy effect and emojis inside text-based passwords. Finally, to answer \textit{RQ3}, we discussed a framework to guide the design of persuasive strategies in Chapter \ref{chap:perdespassup}, which we then also applied to create a novel password-manager. The following sections discuss the central insights and show how they are connected.

\section{Central Contributions and Insights}
\subsubsection{Psychological Factors and Mental Models in Authentication}
% related work mostly _describes_ behaviors and coping strategies, but not the mental models behind them.
% real world password authentication often sees it as a small hurdle, but not worthy of in-depth ux design.
In Chapter \ref{chap:rw:user_perspective}, we found that most related work \textit{describes} coping strategies. Only sometimes the contributing factors like educational background, psychographics, or mental models that foster certain coping strategies are addressed. %Few password alternatives, as presented in Chapter \ref{chap:rw:passwords}, are mature enough to allow investigation of human factors outside the lab, and there is even less insight into the mental models involved. 
%At the same time, real-world authentication systems also often neglects such human factors, which aggravates security issues.

% PASDJO
We filled this gap in several ways. First, we investigated the mental models of password strength, because these are believed to be highly influential on actual password choice. For these purposes, we innovated on research methods to understand latent password strength perceptions: \textit{PASDJO}, the password game, helped in showing that passphrases are often underestimated by users, although \gls{NIST} has started to propagate them in favor of highly complex passwords (see Chapter \ref{chap:pasdjo}). The long-standing belief that strong passwords must include a wide range of characters was clearly visible in the data collected during one year of public deployment. However, users were by and large capable of judging the quality of passwords. We take this as evidence that users are most likely aware of their actions when they select either strong and weak passwords for different purposes. This insight gives rise to a shift in the way we support users in password selection: In many cases it is unnecessary to provide feedback on \textit{strength}, because users can already estimate it well. Therefore, we can tackle other risky behaviors. 
% Policy Audit
In fact, password strength is often a secondary risk factor, if users reuse passwords too carelessly. To investigate the real-world constraints for password reuse, we audited the composition policies of the 83 most visited web-sites in Germany (see Chapter \ref{chap:policies_reuse}). We were able to show that it is easy for users to reuse a single password on most sites: it only has to be nine or ten characters long and consist of lower- and uppercase letters and digits. Hence, this finding is another indication that password policies have shaped the mental model that at least three different character classes are absolutely required to form a strong password -- and that reuse is less critical, because it is not prevented. Moreover, users do not need a password manager if their go-to passwords are accepted by most websites. However, as soon as they add symbols in the belief that doing so fosters password strength even more, the success rate to reuse the resulting password drops significantly. So, in that case, rejecting passwords based on certain symbols might leave users wondering why these do not boost password strength. The result is a cognitive dissonance in the users' mental models. To resolve this, websites could provide some kind of explanation as to their policy choice, but most fail to do this. Besides, it is unlikely that they disallow certain symbols to hamper password reuse. As a take-away, service providers need to start accepting Unicode passwords without arbitrary length restrictions to avoid confusing users. This allows automatically generating unique passwords of all kinds, which can drive the adoption of password managers. At the same time, Unicode passwords have more ramifications on the use of emojis, which we discuss in a moment. 
% and, if necessary, random password generators, which are the only line of defense for users in mitigating offline attacks \cite{Florencio2016CommACM}. At the same time, 
%1) find ways to detect overly re-used passwords and 2) 

% PERSONALITY
Real-world constraints like composition policies shape mental models and coping strategies. On the other hand, there is a spectrum of password coping strategies that cannot be explained by environmental factors alone. We hypothesized that a user's personality might play a role in password selection and coping behavior. In three studies (see Chapter \ref{chap:pws_and_personality}), we examined how personality might be associated with different password tasks. We found that personality was a weak, but non-negligible factor in predicting how different user groups deal with policies, perceive strength, or choose passwords. The data allowed us to create user segments in the form of personas that can be used in the development of authentication schemes and support strategies. 
% PWMs
As a side effect, we observed that people with a background in an IT-related field were more likely to adopt a password manager. Looking at the generally low adoption rates of such software, we explored how users perceive password managers in Chapter \ref{chap:mental_models_pwm}. We contributed the observation that users appreciate this kind of tool once they were first exposed to the technology, e.g. at work. If they had never used a password manager before, they were unable to anticipate how it might help them. We distinguished important themes that shape mental models about password managers. However, we found that these models were currently not fully matched by commercial software. Consequently, there are novel opportunities to re-design password-managers to better support users. 

% summary
In summary, we have to understand additional dimensions of the factors that contribute to mental models, because if we do not, we will fail to help users avoid risky password practices. Environmental constraints like password policies probably have the largest influence. Much traditional advice on how to form ``good passwords'' has led to a skewed mental model of password strength. Second, professional and educational factors are associated in how well users deal with password tasks. Finally, personality also contributes to the shaping of mental models, and warrants further research in this direction. %Current user interfaces, e.g. web sites, do not seem to take mental models into account when they try to simplify the task. 

\subsubsection{Simplification Strategies}
% state of the world
Our exploration of mental models and other psychological factors in password authentication revealed that users meet complex tasks with their individual simplification strategies. Especially if a website implements a complex password policy, users often try to get away with as little effort as they can, which results in predictable secrets. To prevent this and offer alternative simplification strategies, researchers have tried numerous approaches. One of them is based on real time feedback during password selection that shows how well the password meets the policy and how strong it is. Many different variations of feedback have been proposed (e.g., \cite{Egelman2013DoesMyPasswordGoUpToEleven,Habib2017Blacklists,Shay2015SpoonfulOfSugar,Ur2012HowDoesYourPasswordMeasureUp, Vance2013FearAppeals}), but the designs have mostly been based on assumptions about the users' needs in simple password feedback. 
% user need elicitation
We took a different approach and first tried to validate our assumptions and identify aspects that fell short in the related literature. Through a mixed methods approach (see Chapter \ref{chap:feedback_modalities}) we specified user needs and found four central dimensions of password selection support: \textit{showing} current problems, \textit{explaining} the implications, \textit{helping} with improvement, and \textit{empowering} to become creative. These dimensions can be used to facilitate password selection trough feedback and feedforward in future solutions. 

% SHOW AND HELP: DECOY
Moreover, to address the users' needs and to simplify password selection, we explored two persuasive strategies. The first was based on \textit{showing} current problems and \textit{helping} with improvement. We introduced a \textit{choice architecture} for password selection based on the Decoy effect (see Chapter \ref{chap:decoy}). Through an online experiment, we observed that the Decoy choice architecture did not influence participants as expected. However, displaying a passphrase and making its benefits more visible and easily comparable did result in stronger and longer passwords. Thus, we believe that such a combination of feedback and feedforward is the key to simplifying selection strategies for stronger passwords. Although it is not necessary to pick a strong password for every single account, it is very recommendable to reduce guessability of master-passwords for password managers. 
% EMPOWER: EMOJIS
The second strategy we explored aimed to simplify memorization of passwords and \textit{empower} users to become creative in their selection. To that end, we evaluated the usability of using emojis inside text-based passwords in two study sessions (see Chapter \ref{chap:emojipasswords}). We created a prototype to enter emoji-passwords that allowed us to measure selection patterns and issues arising from different visual representations of the same set of emojis across platforms (\textit{fragmentation}). For our participants, the concept brought about the intrinsic desire to create more memorable passwords than what is usually possible, thus the simplification approach went in the right direction. However, once participants faced trouble recognizing the right emojis because their visual style had changed, this fragmentation lowered memorability and participants were reserved towards adopting emoji-passwords in the future. So, although the concept generated interest at first, usability troubles outweighed the anticipated benefits. In one of the personality studies (Chapter \ref{chap:pws_and_personality}), we had found that certain user groups were more inclined to adopt emoji-passwords than others: they were more acceptable for participants who strongly showed the \textit{Neuroticism} trait. Hence, it is very likely that those users will try to create emoji-passwords in the near future, because some services like Twitter and Slack already support them. Therefore, the usability issues that we identified need to be addressed soon to avoid user frustration due to account lock-outs and inefficient input. Only then will emoji-passwords become a true \textit{simplification} strategy, because right now they miss this target. In conclusion, the task of password creation can be simplified for users through careful tuning of the password policy, feedback, feedforward, and empowerment. 

\subsubsection{Guiding Persuasive Designs}
One interesting aspect of the persuasive solutions presented in related work is that they rarely explain the design process in forming them. Reading the literature gives the impression that solutions are solely derived from isolated ideation and/or related work. Important iteration stages of the human-centered design process are often underrepresented and it stays unclear what led to different design choices. To identify new opportunities and exhaust the design space, I argued to structure the process and activities specifically for the design of password support solutions. 

%Since many of the papers were published in proceedings or journals with an HCI focus, it is very likely that the reported solutions followed a human-centered design process. One way to approach this has been visualized as the ``Double Diamonds'' in varying levels of detail (an example is shown in Figure \ref{fig:summary:hcd-design-process}).

To that end, I developed a framework that takes all the insights from related and original work into account. The Persuasive Design for Password Support (P4P) framework addresses specific tasks and questions of password authentication, and guides through different stages of the process. At the same time, it allows taking shortcuts and move directly to a later stage, given that prior work paints a clear picture of the status quo (see Chapter \ref{chap:perdespassup}). To illustrate its usage and applicability, I demonstrated how it informed different stages of the design of a novel password manager. It embraces the fact that many people desire to reuse passwords and ``stay in charge'' of their most treasured accounts. The password manager thus adapts to the user's coping strategies to make the transition smooth. As this is a large software project in an early stage of its implementation, there is a lot of room for improvement and fine-tuning. We contributed the minimum viable product (MVP) under an open-source license to facilitate further development. 

%\begin{figure}
%	\centering
%	\includegraphics[width=\linewidth]{figures/summary/hcd-design-process}
%	\caption{One version of the "Double Diamonds" to structure a traditional Human-Centered Design process. The P4P framework partially adapts this process and tailors it to password authentication. Image by Dan Nessler \url{https://medium.com/@dan.nessler} \la{24.03.2018}}
%	\label{fig:summary:hcd-design-process}
%\end{figure}

\subsubsection{Eight Recommendations for the Future}\label{sec:summary:recommendations}
% sanity check: who's the audience? % how far is this away from the research I've made?
The above summary allows us to give recommendations on service design and research areas. Some bullets confirm prior work and listing them again should be seen as an emphasis. 
\begin{enumerate}
	% dynamic mental models
	\item \textbf{Consider the evolution of mental models}. Coping strategies adjust to the task load generated by passwords, which fluctuates throughout the years. Users might see complexity as the primary strength component, but this might change as service providers adjust to the recommendations from empirical usability research. 
	
	% strength is not all that important
	\item \textbf{Put less emphasis on password \textit{strength}}. Some researchers have demonstrated that beyond the threshold for online attacks, the benefits of increased password strength are limited. The most important scenarios that really require a strong (and usable!) password are master-passwords and accounts holding particularly sensitive data, e.g. a Dropbox that is full with health records or credit card details. 
	
	% more autonomy
	\item \textbf{Remove restrictions, give autonomy}. Service providers should eradicate unjustified complexity requirements, because they have strongly contributed to unreasonable mental models in the past. Instead, foster password diversity through autonomy, i.e. by empowering users to be creative and make informed decisions. We found that users want to be reasonably secure, but often lack the creativity to come up with adequate passwords.
	The \textit{show}-\textit{explain}-\textit{help}-\textit{empower} paradigm can overcome this creativity barrier and act as an overall guideline for authentication even beyond passwords.
	
	\item \textbf{Prepare for more requests of password replacement schemes}. More and more people are willing to use biometrics as primary authentication method\footurl{https://www-03.ibm.com/press/us/en/pressrelease/53646.wss}{26.03.2018}. 
	They will expect this technology from products. However, companies often market biometrics as panacea for usable and secure authentication, and fail to make users aware of the ramifications. Therefore, passwords are going to be met with resistance, and we need to reassure users that passwords have irrefutable benefits in certain situations. 
		
	\item \textbf{Extend the method space}. \label{recommendation:method_space}We can observe a strong tendency towards studies facilitated through \gls{mTurk}. While the methodology is robust for eliciting quantitative data, the results are only one side of the truth. MTurk studies answer \textit{what works best}, but often fail to explain \textit{why} things work best. Therefore, resurrecting mixed-methods approaches that address qualitative aspects is recommendable for future research in \acrfull{USEC}. 
	
	\item \textbf{Stay realistic}. Nudges wear off over time, so we have to constantly create new persuasive strategies. Then again, users quickly resent paternalistic guidance and also prefer things to stay as they are. We have to acknowledge that there is only so much we can do. Persuasive support strategies will not work for all users in the same way, but if they reach even a small target group and make their lives a little easier, I believe that they are impactful enough. %Habituation has never really been a topic in password persuasion.
	
	\item \textbf{Follow risky ideas}. If we look at the current landscape of research on password support, the design space appears narrow: most published research tackles password meters in different facets. I argue that taking inspiration from other research areas, e.g. behavioral economics, can generate ideas outside the usual spectrum. They might be risky in terms of predictable effect size, but they certainly can counteract habituation effects. 
	% See section xyz. current support is somewhat gridlocked. we need more freshness. results with little to no significnant effects might not be easy to publish, but that's not a valid reason not to undertake the endeavor. 
	
	%\item \textbf{Leverage Trends}. % if biometric breaches become more prominent, it is time to ``sell'' passwords again. 
	
	\item \textbf{Give feedback}. Researchers cannot expect that service providers read academic research papers (let alone dissertations). Therefore, we as a community of user advocates have to become active and point out where things go wrong. For instance, it is important to report issues with password policies to service providers. I have engaged in discussions with globally operating companies and was met with an open ear for improvement areas. In the end, this might translate research into graspable impact. 
	
	%\item \textbf{Personalize user interfaces}. 
	%/ feedback etc. Hard challenge b/c user is not known at auth time. Browser APIs might help (Windows Auth API Hello). Password reset -- more info on user. % choice about authentication scheme --> older adults are more 
\end{enumerate}

\subsubsection{Conclusion}
This thesis has presented a new perspective on a well-known and perhaps unsolvable problem: coping with passwords is hard and annoying for most of us. Nonetheless, reducing the frustration component stays a highly desirable goal. We contributed new insights into the factors that shape coping strategies (mental models, personality) and how to design for the users' implicit and explicit needs (a structured process to fine-tune the mixture of feedback, feedforward, and empowering technology). 

\section{Limitations}
The findings should not be interpreted without context. Their main limitations arise from the way data was collected and consequently how the findings apply to other contexts. In the following, I will briefly discuss these aspects and shed light on potential risks of advancing science with persuasion. 

\subsection{Generalizability}
Although we tried to minimize sampling bias, we cannot fully rule it out in any of the reported studies. In total, the fifteen empirical user studies elicited data from 883 people. The surveys were geographically restricted to Germany, the UK, and the USA, because a) the recruitment tools included the largest user panels in these areas and b) we were only capable to create questions in German and English. Similarly, we only recruited participants from the Munich area for interviews and lab studies. So although we were careful to control for other demographic factors, the samples were rather homogeneous. Also, 883 participants cannot be representative for the entire population of Internet users, so we have to be very careful not to draw specific conclusions about different demographic groups. Future studies will have to solidify the findings for different contexts and user groups. On the upside, geographic factors have not been a major influence on password-related problems and solutions \cite{Violettas2014PasswordsAvoidGreece, Wang2015ChinesePWs}. 

% pasdjo: no demographics, motives to play the game are not the same as selecting passwords, probably technical audience, played on special occasions, sampling bias
% personality: sample size too small to narrow down confidence intervals. / other statistical models possible. 

% choice of zxcvbn strength estimation.

% often very small effect sizes

\subsection{Study Designs and Analysis Methods}
% did the best we could, but there might have been other option
% between groups vs. within groups
% clear text passwords
% Without IVs
Since most of the studies had an exploratory character, there was little room for confirmatory methods. Hence, we mostly relied on correlations and associations between different \textit{dependent} variables. This type of research has the disadvantage that larger sample sizes are required to minimize confidence intervals and to  detect latent effects. We did not always achieve optimal levels of statistical power, which is a caveat that needs to be addressed in further research. Nevertheless, the findings help to inform future hypotheses and to run confirmatory studies based on those. 
% with IVs
In case an independent variable was required to study a phenomenon, we carefully weighed the benefits and shortcomings of various study designs. For the decoy and the emoji-passwords studies (Chapters \ref{chap:decoy} and \ref{chap:emojipasswords}), we resorted to between-groups settings that usually require a large sample size to achieve high statistical power. On the plus side, they better show contrasting effects. Moreover, we used within-groups designs in two out of the three personality studies (Chapter \ref{chap:pws_and_personality}). This allowed for a better understanding of individual preferences and was suitable for smaller sample sizes like in our studies. Exploratory questions make it challenging to anticipate the outcome in either setting, but additional resources might have enabled us to choose an alternative study design. Since we were able to answer our research questions satisfactorily, the choice of our methods was plausible, but future studies might need to reconsider them. 

% zxcvbn
Moreover, we refrained from collecting plain-text passwords for ethical reasons. While other researchers save passwords in clear text, they also need to provide a higher standard of protective mechanisms, like locking access to the data and analyzing it off-line. Since we did not have the resources for such procedures, we found it more reasonable to hash passwords if they needed to be stored. This limits the available depth of post-hoc analyses, which is a caveat. At the same time, the consistent usage of the zxcvbn estimator provided sufficient and reliable details about passwords for the analyses we required. We did, however, have to modify it in order to strip it from sensitive information. 

\subsection{Real-World Measurements}
Apart from the log analysis of \textit{PASDJO}, we could not collect data \textit{in the wild}. The concepts we evaluated in Part \ref{part:design_space} were not yet mature enough to warrant production-level deployment of the nudges. We tried to increase ecological validity by following established practices in password research (cf. Section \ref{sec:rw:methodology}). While these attempts let us assume that participants immersed themselves in the tasks, there is always a small gap between study and real-world contexts. Therefore, we have to leave deployments of our concepts, e.g. emoji-passwords or feedforward techniques, to future work. Moreover, coming back to recommendation \ref{recommendation:method_space}, we might be able to assess the ecological validity of the existing data through ethnographic methods, e.g. diary studies or contextual inquiry.

% ecological validity not optimal
% we deployed feedforward in the wild, but too little sign-ups to draw conclusions 

\subsection{Ethics and Risks}
Studying the users' psyche, like cognitive biases and personality, to aid the design of persuasive interventions bears certain ethical risks. Often, we investigate unconscious phenomena, for instance, how the decoy effect influences users' decision-making. Therefore, we need to always consider how findings in this area might be exploited. There was a recent episode of questionable analysis of personality profiles: Cambridge Analytica, a British political consulting firm, accessed millions of Facebook users' data without their consent to target political campaigns based on their personality and other factors\footurl{https://www.nytimes.com/2018/03/17/us/politics/cambridge-analytica-trump-campaign.html}{27.03.2018}. Although studying users' personality to support them in password authentication appears less critical than politically motivated manipulation techniques, we still have to weigh the benefits against the risks. For instance, if future research corroborates our findings about the associations between personality and password selection, this might allow adversaries to target attacks more efficiently. Thus, the P4P framework includes this important aspect in the hope of seeing more discussions of ethical risks in the future. 

%users beyond demographics and explicit 
% nudging is always opinionated

% personality study: grain of salt. Cambridge Analytica accessed FB profiles to create personality profiles such that they can better target certain groups with persuasive messaging. 

%\section{Lessons Learned}
%What would I do differently, if I had to do it all over again?

% 1) work together with other universities to leverage a) different samples b) tools like PGS

% 2) partner up with companies who run a/b tests of concepts in the wild / on production level. 

% KICKED OUT:

%\section{Classification and Dimensions}
%here we talk about in which way the projects contributed to the Framework? and in which way


% 
% not very different to the double diamond, but specificall tailored to password support systems.
% apply the process to demonstrate how design can be guided towards a real-solution.
% password manager.

% evlt noch weiter ausholen aus den bisherigen kapiteln was mitnehmen. 
%\section{Contributions} %TODO choose different name.
%give a walkthrough through all take-aways. that should do and create 1-2 pages. 
%
%\subsection{Theoretical Contributions}
%- framework for the design of password support strategies.
%
%\subsection{Methodological Contributions}
%- measuring passwords in the wild in an ethical way. (meta zxcvbn)
%- novel solution to measure password strength perception. (PASDJO)
%
%\subsection{Empirical Contributions}
%- the decoy effect and password suggestions
%- personality and passwords
%- mental models password managers
%
%\section{Implications}
%% how has this thesis advanced science? / the amount of the world's knowledge.
%\begin{itemize}
%\item Emoji authentication on the web - we're not there yet. 
%\end{itemize}

%Our findings can improve service design and guide future research. 

%The spectrum of problems is nuanced and solutions appear somehow gridlocked. We know much about the technical side and what people do to live with the password burden. What we do not know to its full extent is how exactly coping strategies form and how to support them best. What are the preconditions? Why do not all users behave the same way? How can we leverage that in designs? With the projects reported in this thesis, I aimed to tackle these aspects with ideas off the beaten path. Some of them came with a certain risk as to their feasibility and how to evaluate them. 

% How is PASDJO useful?
% How is the policy data set useful?
% How is the personality data useful?
% How is the Bubbles / Needs exploration useful?
% How is the emoji-passwords project useful? 
% How is the PWRM useful?
% Where do we go from here?

% so -- what did we learn? how has the world changed?
% you need to know the psychological aspects if you try to derive persuasive strategies. there's no way around it. 
% there are many solutions off the beaten path 
% very foundational, strategic, explorative character, risky research questions. 
% what does this mean for XYZ
% how does this change the world?
% thoughts. 
