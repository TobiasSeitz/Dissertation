%!TEX root = ../../perdespassup.tex

\chapter[Persuasive Design for Password Support]{Persuasive Design for Password Support}\label{chap:perdespassup}

% GOAL: Support users in any kind of task that involves passwords. 
% Why is this necessary? / Motivation
% What does it achieve?
% Where can you use it?


The Password Coping Toolkit (PST) is a holistic approach to help users of any experience level manage their password portfolio of any size. Ideally, continuous usage of the PST enables and encourages users to 

there won't be a formula like ``if X then use strategy Y'' (magic wand strategy). Rather, one needs to give different aspects some thought. 


\section{Research Lens}

% Assumptions -- reduce assumptions. 

%%%%%%%%
% Questions
%%%%%%%
\section{Guiding Support Strategies}

\subsubsection{Roadmap}
\begin{tabular}{p{14cm}l}
	What is the desired outcome? & Outcome \\
	How difficult is it to achieve and why? & Difficulty \\
	How big would the impact be? & Impact \\
	How important is it to achieve this outcome? & Importance
\end{tabular}

\subsubsection{Strategy}
\begin{tabular}{p{14cm}l}
	What exactly is the strategy based on? & Foundation\\
	Why is it expected to work? & Support\\
	What stage(s) in the password life cycle does it target? & Stage \\
	What is the specific context? & Context \\
	Are there other ways to achieve the same effect? & Alternatives \\
	Who is the target user group? & User Group\\
	How does the strategy relate to other frameworks? & Relationships\\
	Is it a one-off effort or continuous support? & Time-Scale \\
\end{tabular}

\subsubsection{Prerequisites}
\begin{tabular}{p{14cm}l}
	What are the dimensions of user needs in this context? & Needs \\
	How can the needs be met by the strategy? & Match \\
	What are the characteristics of the strategy? & Characteristics\\
	What are the minimum abilities that the user needs to have? & Abilities\\
\end{tabular}

\subsubsection{Risks}
\begin{tabular}{p{14cm}l}
	Will users get habituated to the strategy and is this good or bad? & Habituation\\
	How would the strategy interfere with other solutions? & Interference\\
	Does the strategy require too much effort from the users? & Effort\\
\end{tabular}

\subsubsection{Evaluation}
\begin{tabular}{p{14cm}l}
	How is the strategy going to be evaluated? & Method\\
	Is it possible to triangulate methods? & Triangulation\\
	What are the metrics to track impact? & Metrics	
\end{tabular}



\subsection{Status Quo}

\section{Design Lens}



\section{Utilizing the Framework in a Design Exercise}


\section{Summary}