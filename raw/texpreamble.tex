% specify what to compile:
% current chapter:
%\includeonly{glossary, content/influence_strategies/emojipasswords}

% for sarah 07.03.:
% \includeonly{glossary, content/problem_space/pasdjo}
% for lena 08.03.
%\includeonly{glossary, content/passwords/coping-strategies, content/passwords/pw-summary}
% for Mo 08.03.
% \includeonly{glossary, content/passwords/passwords, content/passwords/coping-strategies, content/passwords/pw-summary}
% for judith 08.03.:
% \includeonly{glossary, content/influence_strategies/decoy}
% for julie 08.03.:
%\includeonly{glossary, content/problem_space/mm_pwm}
% for ema 09.03.:
%\includeonly{glossary, content/influence_strategies/pws_and_personality}
% for molly 10.03.
%\includeonly{glossary, content/passwords/passwords, content/passwords/pw-summary}
% for prof hh
%\includeonly{glossary, content/passwords/intro, content/passwords/passwords, content/passwords/coping-strategies, content/passwords/pw-summary, content/problem_space/pasdjo, content/problem_space/policies_reuse, content/influence_strategies/pws_and_personality, content/problem_space/mm_pwm}

% COLORS
\definecolor{pinkaccent}{HTML}{F50057}
\definecolor{darkindigo}{HTML}{303F9F}
\definecolor{lightblue}{HTML}{c5cae9}
\definecolor{primaryblue}{HTML}{3f51b5}
\definecolor{bluegray}{HTML}{607d8b}
\definecolor{secondarygray}{HTML}{757575}
\definecolor{dividergray}{HTML}{bdbdbd}

\newcommand{\mytitle}{Supporting Users in Password Authentication with Persuasive Design}
\newcommand{\mysubtitle}{SUsiPAwiPeDes}
\newcommand{\myname}{Tobias Seitz}
\newcommand{\mykeywords}{Passwords, Usable Security, Authentication, Persuasion, Nudging}
\newcommand{\ie}{i.e.,\ }
\newcommand{\eg}{e.g.,\ }

\author{\myname\\\small tobias.seitz@ifi.lmu.de}
\title{\mytitle\\
\mysubtitle\\}
\date{\today}

%Stats
\newcommand{\statslt}[3]{\textit{F\textsubscript{#1}}~$=$~#2, \textit{p}~$<$~#3}
\newcommand{\stats}[3]{\textit{F\textsubscript{#1}}~$=$~#2, \textit{p}~$=$~#3}
\newcommand{\statsgt}[3]{\textit{F\textsubscript{#1}}~$=$~#2, \textit{p}~$>$~#3}
\newcommand{\pval}[1]{\textit{p}~$=$~#1}
\newcommand{\pvallt}[1]{\textit{p}~$<$~#1}
\newcommand{\pvalgt}[1]{\textit{p}~$>$~#1}
\newcommand{\mean}[2]{\textit{m}~$=$~#1, \textit{sd}~$=$~#2}
\newcommand{\meant}[2]{\textit{m}$=$#1 \textit{sd}$=$#2}
\newcommand{\meanonly}[1]{\textit{m}~$=$~#1}
\newcommand{\median}[1]{\textit{median}~$=$~#1}
\newcommand{\average}{average:\xspace}
\newcommand{\Rsqadj}[1]{$R_{adj}^2$~$=$~#1}
%tables and figures
\newcommand*{\thead}[1]{\multicolumn{1}{c}{\bfseries #1}}
\newunicodechar{✓}{\ding{51}}
\newunicodechar{✘}{\ding{56}}
\newunicodechar{↗}{\ding{246}}
\newunicodechar{↘}{\ding{244}}

% emojis
\newlength\myheight
\newlength\mydepth
\settototalheight\myheight{Xygp}
\settodepth\mydepth{Xygp}
\setlength\fboxsep{0pt}
\newcommand*\inlinegraphics[1]{%
	\settototalheight\myheight{Xygp}%
	\settodepth\mydepth{Xygp}%
	\raisebox{-\mydepth}{\includegraphics[height=\myheight]{#1}}%
}
\newcommand{\emoji}[2][ios]{\inlinegraphics{../latex-emoji/#1/#2}}%
% this works on a more rudimentary level: \newcommand{\emoji}[1]{\includegraphics[width=1em]{../latex-emoji/ios/#1}}%

\newcommand{\citep}[1]{\cite{#1}} %convenience for texmaker, but texstudio doesn't have this problem. 

\graphicspath{{figures/}}




%SHORTCUTS
\label{Shortcuts}
\usepackage{xspace}
\newcommand{\percent}{\%\xspace}
\newcommand{\seefig}{see Figure\xspace}
\newcommand{\red}[1]{\textcolor{red}{#1}}
\newcommand{\access}[1]{\textit{(last accessed #1)}}
\newcommand{\la}[1]{\textit{(last accessed #1)}}
\newcommand{\lastaccess}[1]{\textit{(last accessed #1)}}
\newcommand{\footurl}[2]{\footnote{\url{#1} \lastaccess{#2}}}
%alternative approach: try if time:
%\newcommand{\footurl}[2]{
%	\bibitem{#1}~, \url{http://en.wikipedia.org/wiki/Business_logic_layer}, #2.
%	\cite{#1}
%}


\newcommand{\etal}{et al.\xspace}
\newcommand{\todo}[1]{\textcolor{pinkaccent}{\textbf{@TODO} {#1}}}
\newcommand{\ar}{\textcolor{pinkaccent}{\textbf{@ref}}}
\newcommand{\vspc}{\vspace*{2cm}}

\usepackage[light]{roboto}  %% Option 'sfdefault' only if the base font of the document is to be sans serif


\renewcommand{\floatpagefraction}{.8}% this avoids putting larger images onto separate pages.


\titleformat{\chapter} % command
[display] % shape
{\roboto\bfseries\fontsize{40}{48}\selectfont} % format, %fontsize(size, baselineskip = 1.2 x size)
{\thechapter} % label
{0.5ex} % separation between label and title body and it must be a length and not be empty.
{} % before code
[] % after-code

\titleformat{\section} % command
[block] % shape
{\roboto\bfseries\LARGE} % format, %fontsize(size, baselineskip = 1.2 x size)
{\thesection} % label
{0.5ex} % separation between label and title body and it must be a length and not be empty.
{} % before code
[] % after-code

\titleformat{\subsection} % command
[block] % shape
{\roboto\bfseries\large} % format, %fontsize(size, baselineskip = 1.2 x size)
{\thesubsection} % label
{0.5ex} % separation between label and title body and it must be a length and not be empty.
{} % before code
[] % after-code

\titleformat{\subsubsection} % command
[block] % shape
{\roboto\bfseries\normalsize} % format, %fontsize(size, baselineskip = 1.2 x size)
{} % label
{0.5ex} % separation between label and title body and it must be a length and not be empty.
{} % before code
[] % after-code

\titlespacing*{\section}{0pt}{1cm}{0.75cm}
\titlespacing*{\subsection}{0pt}{0.75cm}{0.5ex}
\titlespacing*{\subsubsection}{0pt}{0.75cm}{0.5ex}

\raggedbottom