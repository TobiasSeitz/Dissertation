% glossary term definitions

\newglossaryentry{WPA}{name={WPA},description={WiFi Protected Access. Protocol used in access control for wireless networks.}}
\newglossaryentry{SP}{name={service provider},description={Entity providing access to a resource through a specific service, e.g. the operator of a news website.},plural={service providers}}
\newglossaryentry{PWM}{name={PWM},description={Password Manager. Software that supports a user in the task of managing credentials. Can be standalone or built into web browsers. Famous examples for standalone password managers: LastPass, 1Password, Dashlane, Keepass, RememBear}}
\newglossaryentry{NIST}{name={NIST},description={National Institute of Standards and Technology.}}
\newglossaryentry{PCFG}{name={PCFG},description={Probabisitic Context-Free Grammar. Statistical grammar model. In password studies, PCFG algorithms can be adapted to measure the guessability of a given password and calculate a guess number.}}
\newglossaryentry{PGS}{name={PGS},description={Password Guessing Service. Service that `estimates plaintext passwords' guessability: how many guesses a particular password-cracking algorithm with particular training data would take to guess a password. \url{https://pgs.ece.cmu.edu/}}}
\newglossaryentry{CHI}{name={CHI},description={ACM CHI Conference on Human Factors in Computing Systems. Largest venue of research in Human-Computer Interaction}}



% ACRONYMS (no definition necessary)
\newacronym{USEC}{USEC}{Usable Security and Privacy}
\newacronym{HCI}{HCI}{Human-Computer Interaction}
