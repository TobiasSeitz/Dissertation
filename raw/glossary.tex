% glossary term definitions

\newglossaryentry{WPA}{name={WPA},description={WiFi Protected Access. Protocol used in access control for wireless networks.}}
\newglossaryentry{SP}{name={service provider},description={Entity providing access to a resource through a specific service, e.g. the operator of a news website.},plural={service providers}}
\newglossaryentry{PWMg}{name={password manager},description={Password Manager. Software that supports a user in the task of managing credentials. Can be standalone or built into web browsers. Famous examples for standalone password managers: LastPass, 1Password, Dashlane, Keepass, RememBear},plural={password managers}}
\newglossaryentry{NIST}{name={NIST},description={National Institute of Standards and Technology.}}
\newglossaryentry{PCFG}{name={PCFG},description={Probablisitic Context-Free Grammar. Statistical grammar model. In password studies, PCFG algorithms can be adapted to measure the guessability of a given password and calculate a guess number.}}
\newglossaryentry{PGS}{name={PGS},description={Password Guessing Service. Service that `estimates plaintext passwords' guessability: how many guesses a particular password-cracking algorithm with particular training data would take to guess a password. \url{https://pgs.ece.cmu.edu/}}}
\newglossaryentry{CHI}{name={CHI},description={ACM CHI Conference on Human Factors in Computing Systems. Largest venue of research in Human-Computer Interaction}}
\newglossaryentry{mTurk}{name={mTurk},description={Amazon Mechanical Turk. Crowd-Sourcing platforms where workers (``turkers'') complete micro tasks and receive a small payment.}}
\newglossaryentry{Unicode}{name={Unicode},description={Standard to consistent encode, represent, and handle digital text}}
\newglossaryentry{persona}{name={persona},description={Fictional character that represents a market or user segment during a user-centered design process},plural={personas}}

% ACRONYMS (no definition necessary)
\newacronym{ASCII}{ASCII}{American Standard Code for Information Interchange}
\newacronym{USEC}{USEC}{Usable Security and Privacy}
\newacronym{HCI}{HCI}{Human-Computer Interaction}
\newacronym{CMU}{CMU}{Carnegie Mellon University (Pittsburgh, Pennsylvania, USA)}
\newacronym{HIT}{HIT}{Human Intelligence Task}
\newacronym[longplural='Institutional Review Boards']{IRB}{IRB}{Institutional Review Board}
\newacronym{ISO}{ISO}{International Organization for Standardization}
\newacronym{ESM}{ESM}{Experience Sampling Method}
\newacronym{SSO}{SSO}{Single-Sign On}
\newacronym{PANAS}{PANAS}{Positive Affect and Negative Affect Scale}
\newacronym{SeBIS}{SeBIS}{Security Behavior Intentions Scale}
\newacronym{LUDS}{LUDS}{lowercase, uppercase, digits, symbols}
\newacronym[plural=PWMs]{PWM}{PWM}{password manager}
\newacronym{PD}{PD}{Persuasive Design}
\newacronym{PAF}{PAF}{Persuasive Authentication Framework}
\newacronym{PSM}{PSM}{password strength meter}
\newacronym{GAM}{GAM}{generalized additive model}
\newacronym{REML}{REML}{Restricted maximum likelihood}
\newacronym{PII}{PII}{personally identifiable information}
\newacronym{UX}{UX}{user experience}
\newacronym[plural=UIs]{UI}{UI}{user interface}
\newacronym{W3C}{W3C}{World Wide Web Consortium}
\newacronym[plural=PINs]{PIN}{PIN}{personal identification number}
\newacronym{P4P}{P4P}{Persuasive Design for Password Support}

